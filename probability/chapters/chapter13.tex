\chapter{Desigualdad de Markov}

\mprop{Desigualdad de Markov}{
  Si $\mcX$ es una variable aleatoria que solo toma valores no negativos, entonces para cualquier $a > 0$ se cumple lo siguiente.
  \[P(\mcX \ge a) \le \frac{\bbE(\mcX)}a \]
}

\begin{myproof}
  \begin{align*}
  \bbE(\mcX) &= \int_{0}^{\infty} x f(x)dx = \int_{0}^{a} x f(x)dx + \int_{a}^{\infty} x f(x)dx \\[2ex]
  &\geq \underbrace{\int_{0}^{a} x f(x)dx}_{\geq 0} + a \int_{a}^{\infty} f(x)dx \geq a \int_{a}^{\infty} f(x)dx \\
  \end{align*}
  \[\therefore \, P(\mcX \ge a) \le \frac{\bbE(\mcX)}a \]
\end{myproof}

\nt{
  Si $\mcX = (\mcY - \mu)^2$ y $a = k^2 \sigma^2$ donde $\sigma^2 = \Var(\mcX) > 0$ se puede deducir el teorema de Chebyshev.
}

Estas desigualdades nos permiten encontrar cotas para las probabilidades cuando solo la esperanza y/o la varianza son conocidas.

\ex{Desigualdad de Markov}{
  Supongamos que sabemos que el número de productos fabricados durante una semana es una variable aleatoria con esperanza de 500.

  1. ¿Qué podemos decir sobre la probabilidad de que la producción de una semana sea al menos de 1000?

  2. Si la varianza de esta producción semanal es 100, ¿qué podemos decir sobre la probabilidad de que la producción semanal esté entre 400  y 600?

  Primero definimos una variable aleatoria.
  \[\mcX = \text{ \# productos fabricados por semana} \quad \text{ con } \quad \bbE(\mcX) = 500 \quad \text{ y } \quad \Var(\mcX) = 100\]

  1. $P(\mcX \ge 1000) = \frac{\bbE(\mcX)}{1000} = \frac{500}{1000} = \frac12$

  2. $\sigma^2 = \bbE(\abs{\mcX - \mu}^2) = 100$

  Queremos calcular $P(400 \le \mcX \le 600)$. Usando el teorema de Chebyshev.
  \[P(\abs{\mcX - \mu} \le k\sigma) \ge 1 - \frac1{k^2}\]
  \[\Longrightarrow P(\abs{\mcX - 500} \le k10) = P(-k10 \le \mcX - 500 \le k10) 
  = P(-k10 + 500 \le \mcX \le k10 + 500) \]

  Si sustituimos $k = 10$ obtenemos lo siguiente.
  \[P(400 \le \mcX \le 600) = 1 - \frac{1}{100} = \frac{99}{100} = 0.99\]

}
