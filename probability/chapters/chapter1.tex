\chapter{Conceptos Básicos}
\section{Experimentos}

\dfn{Experimento}{
  Un \textbf{experimento} es una secuencia de pasos que llevan a un resultado.
}

\subsection{Experimento Determinista}
\dfn{Experimento Determinista}{
    Es un experimento que siempre que se repite bajo las mismas condiciones obtenemos el \textit{mismo} resultado.
}

\subsection{Experimento Aleatorio}
\dfn{Experimento Aleatorio}{
    Es un experimento que siempre que se repite bajo las mismas condiciones obtenemos \textit{diferentes} resultados.
}

\section{Probabilidad de un evento}
\dfn{Probabilidad de un evento}{
  Sea \textit A un evento y $\Omega$ el espacio muestral, la \textbf{probabilidad de A} es $$P(A) = \frac{\mid A \mid}{\mid \Omega \mid}$$
}

\section{Espacio Muestral}

\dfn{Espacio Muestral}{
  Es el conjunto de \textit{todos los posibles resultados} de un experimento aleatorio. Se denota por $\Omega$.
}

\ex{Espacio Muestral}{
  Sea el experimento de lanzar una moneda. Se tiene que el espacio muestral es $\Omega = \{\text{sol}, \text{águila}\}$
}

\ex{Espacio Muestral}{
  Sea el experimento de lanzar un dado de 6 caras. Se tiene que el espacio muestral es $\Omega = \{1,2,3,4,5,6\}$
}

\ex{Espacio Muestral}{
  Sea el experimento de contar la duración de una llamada telefónica. Se tiene que el espacio muestral es $\Omega = [0, \infty]$
}

\subsection{Eventos}

\dfn{Evento}{
  Se le denomina \textbf{evento} a \textit{cualquier subconjunto} de $\Omega$.
}

\nt{
  Los \textit{eventos} suelen denotarse con letras mayúsculas.
}

\qs{Eventos}{
  Supongamos que se lanza un par de dados de 6 caras. Encuentra los elementos del evento:

  \begin{itemize}
    \item \textit{A = "En ambos dados se obtuvo el mismo número"}
    \item \textit{B = "Se obtuvo al menos un 3"}
    \item \textit{C = "La suma de los números obtenidos es 5"}
  \end{itemize}

  \sol{
    \begin{itemize}
      \item $A = \{(1,1), (2,2), (3,3), (4,4), (5,5), (6,6)\}$
      \item $B = \{(1,3), (2,3), (3,3), (4,3), (5,3), (6,3)\}$
      \item $C = \{(1, 4), (2, 3), (3, 2), (4, 1)\}$
    \end{itemize}
  }
}

Decimos que un \textit{evento ocurrió} si al realizar el experimento aleatorio, el elemento $w \in \Omega$ ocurre y $w \in A$, decimos que \textit{A ocurrió}.

\nt{
  $\Omega$ es el evento seguro.
}

\section{Interpretación de las operaciones de conjuntos en probabilidad}

Dado $A, B, C \in \Omega$ eventos.
\subsection{Unión}
Dado $A \cup B$. Decimos que el evento \textit{A} \textit{ocurrió} \textbf{ó} el evento \textit{B} \textit{ocurrió}.

\subsection{Intersección}
Dado $A \cap B$. Decimos que el evento \textit{A} \textit{ocurrió} \textbf{y} el evento \textit{B} \textit{ocurrió}. Dicho de otra manera, \textit{ambos eventos ocurrieron}.

\subsection{Complemento}
Dado $B^c$. Decimos que \textit{B} \textbf{no} \textit{ocurrió}.

\subsection{Diferencia}
Dade $C - B$. Decimos que \textit{C} \textit{ocurrió} pero \textbf{no} ocurrió \textit{B}.

\ex{Eventos con conjuntos}{
  Considerando el ejercicio anterior.

  \begin{itemize}
    \item \textit{A = "En ambos dados se obtuvo el mismo número"} $= \{(1,1), (2,2), (3,3), (4,4), (5,5), (6,6)\}$
    \item \textit{B = "Se obtuvo al menos un 3"} $= \{(1,3), (2,3), (3,3), (4,3), (5,3), (6,3)\}$
    \item \textit{C = "La suma de los números obtenidos es 5"} $= \{(1, 4), (2, 3), (3, 2), (4, 1)\}$
  \end{itemize}
  Se tiene lo siguiente.

  \begin{itemize}
    \item $A - B =$ "En ambos dados se obtuvo el mismo número pero no 3" $= \{(1,1), (2,2), (4,4), (5,5), (6,6)\}$
    \item $B^c = \text{No se obtuvo ningún 3}$
    \item $A \cap C = \text{En ambos dados se obtuvo el mismo número y la suma es 5} = \varnothing$
  \end{itemize}
}

\section{Calcular la probabilidad de un evento}
Sea $\Omega$ el espacio muestral de un Experimento aleatorio. Supongamos que $\Omega$ es un conjunto numerable finito. Si \textit{A} es un evento, entonces \textbf{la probabilidad de A} se calcula como sigue.

\[P(A) = \frac{\mid A \mid}{\mid \Omega \mid}\]

\newpage
\section{Ejercicios}

\qs{Conceptos Básicos}{
  Determina el espacio muestral $\Omega$ de los siguientes Experimentos aleatorios.
  \begin{enumerate}[a)]
    \item Lanzar un dado hasta que se obtiene un \textit{5}.
    \item Observar el número de años que le restan de vida a una persona escogida al azar dentro del conjunto de asegurados de una compañia aseguradora.
  \end{enumerate}
}
\qs{Conceptos Básicos}{
  ¿Cuál es el número de elementos del espacio muestral del experimento aleatorio que consiste en lanzar 3 veces una moneda?
}
\qs{Conceptos Básicos}{
  Se extraen aleatoriamente una carta de una baraja de 52. Sea \textit{A = "se extrae un rey"} y \textit{B = "se extrae un trébol"} eventos. Describe los siguientes sucesos o eventos.
  \begin{enumerate}[a)]
    \item $A \cup B$
    \item $A \cap B$
    \item $A \cup B^c$
    \item $A^c \cup B^c$
    \item $A - B$
    \item $A^c - B^c$
  \end{enumerate}
}
