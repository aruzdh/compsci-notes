\chapter{Variables Aleatorias}
\section{Variable aleatoria}
\dfn{Variable Aleatoria}{
  Una \textbf{variable aleatoria} es una función
  \[\mathcal X : \Omega \longrightarrow \mathbb R\] tal que 
  \[\{w \in \Omega \mid \mathcal X(w) \le a\} \in \mathcal F,~ \forall a \in \mathbb R\]
}

\ex{Variable Aleatoria}{
  Supongamos que lanzamos 3 monedas. Denotamos por \textit{$\mathcal Y$ = "El \# de soles que aparecen"}.\\
  Entonces $\mathcal Y (AAS) = 1$, $\mathcal Y(SSS) = 3$.
}

\section{Tipos de Variables Aleatorias}

\subsection{Variables Aleatorias Discretas}

\dfn{Variable aleatoria discreta}{
  Una \textbf{variable aleatoria discreta} toman un \textit{número finito o infinito numerable} de valores.
}

\subsection{Variables Aleatorias Continuas}

\dfn{Variable aleatoria continuas}{
  Una \textbf{variable aleatoria continua} toma un \textit{número infinito no numerable} de valores.
}

\ex{Variable Aleatoria continua}{
  Sea $\mathcal X$ la variable aleatoria que registra la duración de una llamada telefónica. Entonces \[\mathcal X (w) \in (0, \infty)\]
}

\section{Función de Distribución}
\dfn{}{
  Dada $\mathcal X$ una variable aleatoria \textit{discreta o continua}, su \textbf{función de distribución} se define como
  $$F(a) = P(\mathcal X \le a) = P(\{w \in \Omega \mid \mathcal X(w) \le a\}),~ a \in \mathbb R$$
}

\nt{
  Notemos que $F : \mathbb R \longrightarrow [0,1]$
}

\ex{función de Distribución}{
  Consideremos el ejemplo anterior donde lanzamos 3 monedas justas. Dada $\mathcal Y$ definida anteriormente, se tiene lo siguiente.
  \begin{itemize}
    \item $F_{\mathcal Y} (x) = P(\mathcal Y \le x)$
    \item $F_{\mathcal Y} (-1) = P(\mathcal Y \le -1) = 0$
    \item $F_{\mathcal Y} (0) = P(\mathcal Y \le 0) = \frac18$
    \item $F_{\mathcal Y} (0.3) = P(\mathcal Y \le 0.3) = \frac18$
    \item $F_{\mathcal Y} (2.1) = P(\mathcal Y \le 2.1) = \frac18 + \frac38 + \frac38 = \frac78$
    \item $F_{\mathcal Y} (87080685230) = P(\mathcal Y \le 87080685230) = 1$
  \end{itemize}
}

\nt{
  Notese que esta función es \textbf{acumulativa}.
}

Las siguientes son propiedades de las funciones de distribución.
\begin{enumerate}
  \item $$\text{Es una función no-decreciente.}$$
  \item \[ \lim_{x \to \infty} F(x) = 1 \]
  \item \[ \lim_{x \to -\infty} F(x) = 0 \]
  \item $$\text{Es una función continua por la derecha.}$$
  \item \[ P(\mathcal X < a) = F(a^-) \text{limite por la izquierda}\]
  \item \[ P(\mathcal X = a) = P(\mathcal X \le a) - P(\mathcal X < a) = F(a) - F(a^-) \]
  \item \[ P(a< \mathcal X \le b) = P(\mathcal X \le b) - P(\mathcal X \le a) = F(b) - F(a) \]
  \item \[ P(a \le \mathcal X \le b) = P(\mathcal X \le b) - P(\mathcal X < a) = F(b) - F(a^-) \]
  \item \[ P(a < \mathcal X < b) = P(\mathcal X < b) - P(\mathcal X \le a) = F(b^-) - F(a) \]
  \item \[ P(a \le \mathcal X < b) = P(\mathcal X < b) - P(\mathcal X < a) = F(b^-) - F(a^-) \]
\end{enumerate}

\section{Variables Aleatorias Discretas}
\dfn{Función de probabilidad o Función de probabilidad de masa}{
  Sea $\mathcal X$ una \textit{variable aleatoria discreta}. Entonces su \textbf{función de probabilidad} es
  \[f_{\mathcal X} (a) = P(\mathcal X = a),\] con \textit a en la imagen de $\mathcal X$
}

\nt{
  Notemos que si $\mathcal X$ es una variable aleatoria \textit{continua}, su función de distribución es continua, y por la tanto 
  \[P(\mathcal X = a) = F(a) - F(a^-) = F(a) - F(a) = 0\]
  para cualquier $a \in \mathbb R$. Es decir, no tiene utilidad en el caso continuo.
}

\ex{Función de probabilidad}{
  Consideremos el ejemplo anterior donde lanzamos 3 monedas justas. Dada la $\mathcal Y$ definida anteriormente, se tiene lo siguiente.
  \begin{itemize}
    \item $f_{\mathcal Y} (a) = P(\mathcal Y = a)$
    \item $f_{\mathcal Y} (0) = P(\mathcal Y = 0) = \frac18$
    \item $f_{\mathcal Y} (1) = P(\mathcal Y = 1) = \frac38$
    \item $f_{\mathcal Y} (2) = P(\mathcal Y = 2) = \frac38$
    \item $f_{\mathcal Y} (3) = P(\mathcal Y = 3) = \frac18$
  \end{itemize}
}

\nt{
  Notese que esta función \textbf{no es acumulativa}.
}

Las siguientes son propiedades de las funciones de probabilidad.
\begin{enumerate}
  \item \[f(x) \ge 0\]
  \item \[\sum_x f(x) = 1,\] donde \textit x representa todos los valores que toma la variable aleatoria \textbf{discreta}.
\end{enumerate}

\ex{Función de probabilidad}{
  Sea $f_{\mathcal X} (0) = \frac12 \ge 0$ y $f_{\mathcal X} (1) = \frac12 \ge 0$. Se tiene que 
  \[\sum_{i = 0}^1 f_{\mathcal X}(i) = P(\mathcal X = 0) + P(\mathcal X = 1) = \frac12 + \frac12 = 1\]
  Por tanto cumple las propiedades listadas anteriormente.
}

\section{Variables aleatoria Absolutamente Continuas}

\dfn{Variables Aleatorias Absolutamente Continuas}{
  Sea $\mathcal X$ una \textit{variable Aleatorias continua} con función de distribución $F_{\mathcal X}$, se le llama \textbf{absolutamente continua} \textit{si existe} una función \textit f tal que
  \[F_{\mathcal X}(a) = \int_{-\infty}^a f(x)~dx = P(\mathcal X \le a)\]
}
A la función \textit f se le conoce como \textbf{función de densidad} de la variable aleatoria $\mathcal X$ y satisface las siguientes propiedades.

\begin{enumerate}
  \item \[ f(x) \ge 0, ~ \forall x \in \mathbb R \]
  \item \[ \int_{-\infty}^{\infty}f(x)~dx = 1 \]
\end{enumerate}

\nt{
  Si la función de densidad \textit f es continua, el \textit{Teorema Fundamental del Cálculo} nos dice que
  \[\frac{d}{dx} F(x) = f(x)\]
}

\section{Ejercicios}

\qs{Variables Aleatorias}{
  3 pelotas se eligen aleatoriamente de una caja donde 5 son azules, 3 son rojas, y 3 son amarillas. Supongamos que ganamos \$1 por cada pelota amarilla que seleccionamos y perdemos \$1 por cada pelota roja. Sea $\mathcal X = \text{"Denota el dinero que obtenemos en este experimento"}$. Calcula la función de probabilidad de $\mathcal X$.
}
\qs{Variables Aleatorias}{
  Demuestra que $F_{\mathcal X}(x^-) = P(\mathcal X < x) ~ \forall x \in \mathbb R$, donde \textit{F} es la función de distribución.
}

\qs{Variables Aleatorias}{
  La función de distribución de la variable aleatoria $\mathcal X$ es la siguientes.
  \[
    F_{\mathcal X}(x) = 
    \begin{cases}
      0,  &\text{Si } x <0\\
      \frac x4,  &\text{Si } 0 \le x < 1\\
      \frac 12 + \frac{x-1}4,  &\text{Si } 1 \le x < 2\\
      \frac {11}{12},  &\text{Si } 2 \le x < 3\\
      1,  &\text{Si } 3 \le x\\
    \end{cases}
  \]
  Grafica esta función y encuentra $P(\frac12 < \mathcal X < \frac32)$
}
\qs{Variables Aleatorias}{
  Supongamos que se lanzan dos dados de 6 caras y que la variable aleatoria $\mathcal X$ representa la suma de los números obtenidos. encuentra la función de distribución de la variable aleatoria $\mathcal X$ y graficala.
}
\qs{Variables Aleatorias}{
  Determina la \textit{c} tal que la función pueda servir como la distribución de probabilidad de una variable aleatoria con el intervalo dado.
  \begin{itemize}
    \item $f(x) = cx,~ x = 1, \cdots 5$
    \item $f(x) = c \binom5x,~ x = 0,1, \cdots, 5$
    \item $f(x) = cx^2, ~x = 1, 2, \cdots, k$
  \end{itemize}
}

\qs{Variables Aleatorias}{
  La función de distribución de la variable aleatoria $\mathcal X$ es la siguientes.
  \[
    F_{\mathcal X}(x) = 
    \begin{cases}
      0, &\text{Si } x < -2\\
      \frac {x+2}2,  &\text{Si } -2 \le x < -1\\
      \frac12 , &\text{Si } -1 \le x < 1\\
      \frac x2,  &\text{Si } 1 \le x < 2\\
      1,  &\text{Si } 2 \le x\\
    \end{cases}
  \]
  Encuentra $P(-2 < \mathcal X < 2)$
}

\qs{Variables Aleatorias}{
  La compañía de seguros Acme tiene dos tipos de clientes: cuidadosos(as) e imprudentes. Un cliente cuidadoso(a) tiene un accidente durante el año con probabilidad 0.01. Un(a) cliente imprudente tiene un accidente durante el año con probabilidad 0.04. El 80\% de los (as) clientes son cuidadosos(as) y el 20\% de los(as) clientes son imprudentes. Supongamos que un(a) cliente elegido(a) al azar tiene un accidente este año. ¿Cuál es la probabilidad de que este(a) cliente(a) sea uno(a) de los(as) clientes cuidadosos(as)?
}

\qs{Variables Aleatorias}{
  Para cualesquiera A, $B\in\mathcal{F}$ demuestra que si $P(B|A)>P(B)$ entonces $P(B^{c}|A)<P(B^{c})$.
}

