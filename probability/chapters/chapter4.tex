\chapter{Variables Aleatorias}
\section{Variable Aleatoria}
\subsection{Definición}

\dfn{Variable Aleatoria}{
  Una \textbf{variable aleatoria} es una función
  \[\mcX : \Omega \longrightarrow \bbR\] tal que
  \[\{w \in \Omega \mid \mathcal X(w) \le a\} \in \mathcal F,\, \forall a \in \mathbb R\]
}

\ex{Variable Aleatoria}{
  Supongamos que lanzamos 3 monedas. Denotamos por \textit{$\mathcal Y$ = "El \# de soles que aparecen"} (una variable aleatoria).
  \newpara
  Entonces $\mathcal Y (AAS) = 1$, $\mathcal Y(SSS) = 3$. Donde \textit{A = águila} y \textit{S = sol}
}

\section{Tipos de Variables Aleatorias}
\subsection{Variables Aleatorias Discretas}

\dfn{Variable Aleatoria Discreta}{
  Una \textbf{variable aleatoria discreta} toma un \textit{número finito o infinito numerable} de valores.
}

\subsection{Variables Aleatorias Continuas}

\dfn{Variable Aleatoria Continuas}{
  Una \textbf{variable aleatoria continua} toma un \textit{número infinito no numerable} de valores.
}

\ex{Variable Aleatoria Continua}{
  Sea $\mathcal X$ la variable aleatoria que registra la duración de una llamada telefónica. Entonces \[\mathcal X (w) \in (0, \infty)\]
}

\section{Función de Distribución}
\subsection{Definición}
\dfn{Función de Distribución}{
  Dada $\mathcal X$ una variable aleatoria \textit{discreta o continua}, su \textbf{función de distribución} se define como
  \[F_{\mcX}(a) = P(\mathcal X \le a) = P(\{w \in \Omega \mid \mathcal X(w) \le a\}), \quad \text{con} \quad a \in \bbR\]
}

\nt{
  Cuando se tiene una sola variable aleatoria o cuando es evidente sobre que variable aleatoria se habla, es normal omitir el subíndice de la función $F$.
}

\nt{
  Notemos que $F : \mathbb R \longrightarrow [0,1]$
}

\ex{Función de Distribución}{
  Consideremos el ejemplo anterior donde lanzamos 3 monedas justas. Dada $\mathcal Y$ definida anteriormente, se tiene lo siguiente.
  \begin{itemize}
    \item $F_{\mathcal Y} (x) = P(\mathcal Y \le x)$
    \item $F_{\mathcal Y} (-1) = P(\mathcal Y \le -1) = 0$
    \item $F_{\mathcal Y} (0) = P(\mathcal Y \le 0) = \frac18$
    \item $F_{\mathcal Y} (0.3) = P(\mathcal Y \le 0.3) = \frac18$
    \item $F_{\mathcal Y} (2.1) = P(\mathcal Y \le 2.1) = \frac18 + \frac38 + \frac38 = \frac78$
    \item $F_{\mathcal Y} (87080685230) = P(\mathcal Y \le 87080685230) = 1$
  \end{itemize}
}

\nt{
  Notese que esta función es \textbf{acumulativa}.
}

\subsection{Propiedades}
Las siguientes son propiedades de las funciones de distribución.

\begin{enumerate}
  \item $$\text{Es una función no-decreciente.}$$
    \begin{myproof}
      Sea $x_1 \le x_2$. Queremos probar que $F_{\mcX}(x_1) \le F_{\mcX}(x_2)$.

      Tenemos que $F_{\mcX}(x_1) = P(\mcX \le x_1)$ y $F_{\mcX}(x_2) = P(\mcX \le x_2)$.

      Notemos que si $w_0 \in \{w \in \Omega \mid \mcX(w) \le x_1\}$, entonces $\mcX(w_0) \le \overbrace{x_1 \le x_2}^{\text{hipótesis}}$.

    Así, se tiene que $w_0 \in \{w \in \Omega \mid \mcX(w) \le x_2\}$, implicando que $\{\mcX(w) \le x_1\} \subseteq \{\mcX(w) \le x_2\}$, y por tanto, por los axiomas de probabilidad, se concluye
      \[F_{\mcX}(x_1) = P(\mcX \le x_1) \le P(\mcX \le x_2) = F_{\mcX}(x_2)\]
      \[\therefore \, \text{$F$ es no-decreciente}\]

    \end{myproof}
  \item \[ \lim_{x \to \infty} F(x) = 1 \]
    \begin{myproof}
      Sea $\{a_n\}$ una sucesión de números reales tales que $a_n \le a_{n+1}$ y $\lim_{n\to \infty} a_n = \infty$

      Tenemos que
      \[\lim_{x \to \infty} F_{\mcX}(x) = \lim_{n \to \infty} F_{\mcX}(a_n) = \lim_{n \to \infty} P(\mcX \le a_n)\]

      Observemos que $\{\mcX \le a_n\} \subseteq \{\mcX \le a_{n+1}\}$ es una sucesión de conjuntos monótona creciente. Entonces
      \[\lim_{n\to \infty} P(\mcX \le a_n) = P\parens*{\lim_{n\to \infty}\{\mcX \le a_n\}} = P\parens*{\bigcup_{n=1}^{\infty} \{\mcX \le a_n\}} = P(\Omega) = 1 \]
      \[\therefore \, \lim_{n\to \infty}F_{\mcX}(a_n) = 1\]
    \end{myproof}
  \item \[ \lim_{x \to -\infty} F(x) = 0\]
    \begin{myproof}
      Ejercicio.
    \end{myproof}
  \item $$\text{Es una función continua por la derecha.}$$
    \begin{myproof}
      Ejercicio.
    \end{myproof}
  \item \[ P(\mathcal X < a) = F(a^-) \text{limite por la izquierda}\]
    \begin{myproof}
      Ejercicio.
    \end{myproof}
  \item \[ P(\mathcal X = a) = P(\mathcal X \le a) - P(\mathcal X < a) = F(a) - F(a^-) \]
    \begin{myproof}
      Ejercicio.
    \end{myproof}
  \item \[ P(a< \mathcal X \le b) = P(\mathcal X \le b) - P(\mathcal X \le a) = F(b) - F(a) \]
    \begin{myproof}
      Ejercicio.
    \end{myproof}
  \item \[ P(a \le \mathcal X \le b) = P(\mathcal X \le b) - P(\mathcal X < a) = F(b) - F(a^-) \]
    \begin{myproof}
      Ejercicio.
    \end{myproof}
  \item \[ P(a < \mathcal X < b) = P(\mathcal X < b) - P(\mathcal X \le a) = F(b^-) - F(a) \]
    \begin{myproof}
      Ejercicio.
    \end{myproof}
  \item \[ P(a \le \mathcal X < b) = P(\mathcal X < b) - P(\mathcal X < a) = F(b^-) - F(a^-) \]
    \begin{myproof}
      Ejercicio.
    \end{myproof}
\end{enumerate}

\section{Variables Aleatorias Discretas}
\subsection{Definición}
\dfn{Función de Probabilidad o Función de Probabilidad de Masa}{
  Sea $\mathcal X$ una \textit{variable aleatoria discreta}. Definimos su \textbf{función de probabilidad} como sigue a continuación.
  \[f_{\mathcal X} (a) = P(\mathcal X = a),\] con \textit a en la imagen de $\mathcal X$
}

\nt{
  Notemos que si $\mathcal X$ es una variable aleatoria \textit{continua}, su función de distribución es continua, y por la tanto 
  \[P(\mathcal X = a) = F(a) - F(a^-) = F(a) - F(a) = 0\]
  para cualquier $a \in \mathbb R$. Es decir, no tiene utilidad en el caso continuo.
}

\ex{Variables Aleatorias Discretas}{
  Consideremos el ejemplo anterior donde lanzamos 3 monedas justas. Dada la $\mathcal Y$ definida anteriormente, se tiene lo siguiente.
  \begin{itemize}
    \item $f_{\mathcal Y} (a) = P(\mathcal Y = a)$
    \item $f_{\mathcal Y} (0) = P(\mathcal Y = 0) = \frac18$
    \item $f_{\mathcal Y} (1) = P(\mathcal Y = 1) = \frac38$
    \item $f_{\mathcal Y} (2) = P(\mathcal Y = 2) = \frac38$
    \item $f_{\mathcal Y} (3) = P(\mathcal Y = 3) = \frac18$
  \end{itemize}
}

\nt{
  Notese que esta función \textbf{no es acumulativa}.
}

\section{Propiedades}
Las siguientes son propiedades de las funciones de probabilidad.
\begin{enumerate}
  \item \[f_{\mcX}(k) \ge 0\]
  \item \[\sum_k f(k) = 1,\] donde \textit k representa todos los valores que toma la variable aleatoria.
\end{enumerate}

\ex{Variables Aleatorias Discretas}{
  Sea $f_{\mathcal X} (0) = \frac12 \ge 0$ y $f_{\mathcal X} (1) = \frac12 \ge 0$. Se tiene que 
  \[\sum_{i = 0}^1 f_{\mathcal X}(i) = P(\mathcal X = 0) + P(\mathcal X = 1) = \frac12 + \frac12 = 1\]
  Por tanto cumple las propiedades listadas anteriormente.
}

\section{Variables Aleatorias Absolutamente Continuas}
\subsection{Definición}

\dfn{Variables Aleatorias Absolutamente Continuas}{
  Sea $\mathcal X$ una \textit{variable Aleatorias continua} con función de distribución $F_{\mathcal X}$, se le llama \textbf{absolutamente continua} \textit{si existe} una función \textit f tal que
  \[F_{\mathcal X}(a) = \int_{-\infty}^a f(x)_{\mcX}~dx = P(\mathcal X \le a)\]
}

\subsection{Función de Densidad y sus Propiedades}

A la función $f_{\mcX}$ se le conoce como \textbf{función de densidad} de la variable aleatoria $\mathcal X$.

\begin{enumerate}
  \item \[ f_{\mcX}(x) \ge 0, ~ \forall x \in \mathbb R \]
  \item \[ \int_{-\infty}^{\infty}f_{\mcX}(x)~dx = 1 \]
\end{enumerate}

\nt{
  Si la función de densidad $f_{\mcX}$ es continua, el \textit{Teorema Fundamental del Cálculo} nos dice que
  \[\frac{d}{dx} F_{\mcX}(x) = f_{\mcX}(x)\]
}

\section{Ejercicios}

\qs{Variables Aleatorias}{
  3 pelotas se eligen aleatoriamente de una caja donde 5 son azules, 3 son rojas, y 3 son amarillas. Supongamos que ganamos \$1 por cada pelota amarilla que seleccionamos y perdemos \$1 por cada pelota roja. Sea $\mathcal X =$ "Dinero obtenido en este experimento". Calcula la función de probabilidad de $\mathcal X$.
}
\qs{Variables Aleatorias}{
  Demuestra que $F_{\mathcal X}(x^-) = P(\mathcal X < x) ~ \forall x \in \mathbb R$, donde \textit{F} es la función de distribución.
}

\qs{Variables Aleatorias}{
  La función de distribución de la variable aleatoria $\mathcal X$ es la siguientes.
  \[
    F_{\mathcal X}(x) = 
    \begin{cases}
      0,  &\text{Si } x <0\\
      \sfrac x4,  &\text{Si } 0 \le x < 1\\
      \sfrac 12 + \frac{x-1}4,  &\text{Si } 1 \le x < 2\\
      \sfrac {11}{12},  &\text{Si } 2 \le x < 3\\
      1,  &\text{Si } 3 \le x\\
    \end{cases}
  \]
  Grafica esta función y encuentra $P(\frac12 < \mathcal X < \frac32)$
}
\qs{Variables Aleatorias}{
  Supongamos que se lanzan dos dados de 6 caras y que la variable aleatoria $\mathcal X$ representa la suma de los números obtenidos. encuentra la función de distribución de la variable aleatoria $\mathcal X$ y graficala.
}
\qs{Variables Aleatorias}{
  Determina la \textit{c} tal que la función pueda servir como la distribución de probabilidad de una variable aleatoria con el intervalo dado.
  \begin{itemize}
    \item $f(x) = cx,~ x = 1, \cdots 5$
    \item $f(x) = c \binom5x,~ x = 0,1, \cdots, 5$
    \item $f(x) = cx^2, ~x = 1, 2, \cdots, k$
  \end{itemize}
}

\qs{Variables Aleatorias}{
  La función de distribución de la variable aleatoria $\mathcal X$ es la siguientes.
  \[
    F_{\mathcal X}(x) = 
    \begin{cases}
      0, &\text{Si } x < -2\\
      \frac {x+2}2,  &\text{Si } -2 \le x < -1\\
      \sfrac12 , &\text{Si } -1 \le x < 1\\
      \sfrac x2,  &\text{Si } 1 \le x < 2\\
      1,  &\text{Si } 2 \le x\\
    \end{cases}
  \]
  Encuentra $P(-2 < \mathcal X < 2)$
}

\qs{Variables Aleatorias}{
  La compañía de seguros Acme tiene dos tipos de clientes: cuidadosos(as) e imprudentes. Un cliente cuidadoso(a) tiene un accidente durante el año con probabilidad 0.01. Un(a) cliente imprudente tiene un accidente durante el año con probabilidad 0.04. El 80\% de los (as) clientes son cuidadosos(as) y el 20\% de los(as) clientes son imprudentes. Supongamos que un(a) cliente elegido(a) al azar tiene un accidente este año. ¿Cuál es la probabilidad de que este(a) cliente(a) sea uno(a) de los(as) clientes cuidadosos(as)?
}

\qs{Variables Aleatorias}{
  Para cualesquiera A, $B\in\mathcal{F}$ demuestra que si $P(B|A)>P(B)$ entonces $P(B^{c}|A)<P(B^{c})$.
}


\qs{Variables Aleatorias}{
  Demostrar las propiedades restantes de las funciones de distribución.
}

