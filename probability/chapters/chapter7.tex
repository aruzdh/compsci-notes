\chapter{Varianza}
\section{Definicion caso discreto y continuo}

\dfn{Varianza}{
  La \textbf{varianza} de una variable aleatoria $\mathcal X$, denotada por $\Var(\mathcal X)$, se define de la siguiente forma.
  \[\Var(\mathcal X) = \mathbb E([\mathcal X - \mathbb E(\mathcal X)]^2)\]
  Cuando dicha esperanza existe.
  \newpara
  De este modo, existen dos casos.
  \begin{itemize}
    \item Caso discreto.
      \[\Var(\mathcal X) = \mathbb E([\mathcal X - \mathbb E(\mathcal X)]^2) = \sum_i (i - \mathbb E(\mathcal X))^2 P(\mathcal X = i)\]
    \item Caso continuo.
      \[\Var(\mathcal X) = \mathbb E([\mathcal X - \mathbb E(\mathcal X)]^2) = \int_{-\infty}^\infty (x - \mathbb E(\mathcal X))^2 f_{\mathcal X} (x)dx\]
  \end{itemize}
}

\nt {
  Retomando el ejemplo de una variable aleatoria normal, podemos decir que la varianza ($\sigma ^2$) nos indica que tanto se alejan los datos del valor esperado (esperanza).
}

\section{Propiedades}
Las siguientes son propiedades de la varianza. Sean $\mathcal X, \mathcal Y$ variables aleatorias con varianza finita y $c \in \mathbb R$.

\begin{enumerate}
\item $\Var(\mathcal X) \ge 0$
  \begin{myproof}
    $\Var(\mathcal X) = \mathbb E([\mathcal X - \mathbb E(\mathcal X)]^2)$ donde podemos notar que $(\mathcal X - \mathbb E(\mathcal X))^2 \ge 0$
    \[\therefore \, \mathbb E([\mathcal X - \mathbb E(\mathcal X)]^2) \ge 0 \]
  %
  \end{myproof}
\item $\Var(c) = 0$
  \begin{myproof}
    \[\Var(c) = \mathbb E([c - \mathbb E(c)]^2) = \mathbb E([c - c]^2) = \mathbb E(0) = 0\]
  \end{myproof}
\item $\Var(c \mathcal X) = c^2 \Var(\mathcal X)$
  \begin{myproof}
    \[\Var(c\mathcal X) = \mathbb E([c\mathcal X - \mathbb E(c\mathcal X)]^2) = \mathbb E([c\mathcal X - c\mathbb E(\mathcal X)]^2) = \mathbb E(c^2[\mathcal X - \mathbb E(\mathcal X)]^2) = c^2\mathbb E([\mathcal X - \mathbb E(\mathcal X)]^2) = c^2 \Var(\mathcal X)\]
  \end{myproof}
\item $\Var(\mathcal X + c) = \Var(\mathcal X)$
  \begin{myproof}
    \[\Var(\mathcal X + c) = \mathbb E([\mathcal X + c - \mathbb E(\mathcal X + c)]^2) = \mathbb E([\mathcal X + c - \mathbb E(\mathcal X) - c]^2) = \mathbb E([\mathcal X - \mathbb E(\mathcal X)]^2) = \Var(\mathcal X)\]
  \end{myproof}
\item $\Var(\mathcal X) = \mathbb E(\mathcal X^2) - [\mathbb E(\mathcal X)]^2$
  \begin{myproof}
    \begin{align*}
      \Var(\mathcal X) &= \mathbb E([\mathcal X - \mathbb E(\mathcal X)]^2)\\
                             &= \mathbb E(\mathcal X^2 - 2\mathcal X\mathbb E(\mathcal X) + (\mathbb E(\mathcal X))^2)
                             &= \mathbb E(\mathcal X^2) - 2 \mathbb E(\mathcal X) \mathbb E(\mathcal X) + (\mathbb E(\mathcal X))^2 = \mathbb E(\mathcal X) - (\mathbb E(\mathcal X))^2
    \end{align*}
  \end{myproof}
\item En general, $\Var(\mathcal X + \mathcal Y) \neq \Var(\mathcal X) +\Var(\mathcal Y)$
  \begin{myproof}
    \begin{align*}
      \Var(\mathcal X + \mathcal Y) &= \mathbb E \left([(\mathcal X + \mathcal Y) - \mathbb E(\mathcal X + \mathcal Y)]^2\right) \\
                      &= \mathbb E \left([(\mathcal X + \mathcal Y)]^2 - 2(\mathcal X + \mathcal Y)\mathbb E(\mathcal X + \mathcal Y) + (\mathbb E(\mathcal X + \mathcal Y))^2\right) \\
                      &= \mathbb E \left([\mathcal X^2 + 2\mathcal X \mathcal Y + \mathcal Y^2] - 2(\mathcal X + \mathcal Y)(\mathbb E(\mathcal X) + \mathbb E(\mathcal Y)) + (\mathbb E(\mathcal X)^2 + 2\mathbb E(\mathcal X)\mathbb E(\mathcal Y) + \mathbb E(\mathcal Y)^2)\right) \\
                      &= \mathbb E \left(\mathcal X^2 + 2\mathcal X \mathcal Y + \mathcal Y^2 - [2\mathcal X\mathbb E(\mathcal X) + 2\mathcal X\mathbb E(\mathcal Y) + 2\mathcal Y\mathbb E(\mathcal X) + 2\mathcal Y\mathbb E(\mathcal Y)] \right. \\
                      &\qquad\qquad \left. + [\mathbb E(\mathcal X)^2 + 2\mathbb E(\mathcal X)\mathbb E(\mathcal Y) + \mathbb E(\mathcal Y)^2]\right) \\
                      &= \mathbb E \left([\mathcal X - \mathbb E(\mathcal X)]^2 + [\mathcal Y - \mathbb E(\mathcal Y)]^2 + 2\mathcal X \mathcal Y - 2\mathcal X\mathbb E(\mathcal Y) - 2\mathcal Y\mathbb E(\mathcal X) + 2\mathbb E(\mathcal X)\mathbb E(\mathcal Y)\right) \\
                      &= \Var(\mathcal X) + \Var(\mathcal Y) + 2\mathbb E(\mathcal X \mathcal Y) - 2\mathbb E(\mathcal X)\mathbb E(\mathcal Y) - 2\mathbb E(\mathcal X)\mathbb E(\mathcal Y) + 2\mathbb E(\mathcal X)\mathbb E(\mathcal Y) \\
                      &= \Var(\mathcal X) + \Var(\mathcal Y) + 2(\mathbb E(\mathcal X \mathcal Y) - \mathbb E(\mathcal X)\mathbb E(\mathcal Y)) \neq \Var(\mathcal X) + \Var(\mathcal Y)
    \end{align*}
  \end{myproof}
\end{enumerate}

\ex{Varianza}{
  Calcular $\Var(\mathcal X)$ si $\mathcal X \sim \text{exp}(\lambda)$
  De un ejemplo anterior sabemos que $\mathbb E(\mathcal X) = \frac1\lambda$. Usando la propiedad 5 podemos hacer lo siguiente.
  \[\Var(\mathcal X) = \mathbb E(\mathcal X^2) - [\mathbb E(\mathcal X)]^2 = \mathbb E (\mathcal X^2) = (\frac1\lambda)^2\]

  Por otro lado,
  \[
    \mathbb E(\mathcal X^2) = \int_{-\infty}^\infty x^2 \lambda e^{-\lambda x}\mathbb 1_{(0, \infty)}^{(x)}dx = \lambda \int_0^\infty x^2 e^{-\lambda x}dx
  \]
  Usando la siguiente sustitución
  \[
    u = x^2 \Longrightarrow du = 2xdx \qquad dv = e^{-\lambda x}dx \Longrightarrow v = -\frac{e^{-\lambda x}}\lambda
  \]
  obtenemos
  \begin{align*}
    \mathbb{E}(X^2) &= \lambda \left[-\frac{x^2e^{-\lambda x}}{\lambda}\bigg|_0^\infty + \frac{2}{\lambda}\int_0^\infty xe^{-\lambda x}dx\right] = \lambda\left[\frac{2}{\lambda^2}\int_0^\infty x\lambda e^{-\lambda x}dx\right] \\
    &= \lambda\left[\frac{2}{\lambda^2}\int_0^\infty xf_X(x)dx\right] = \lambda\left[\frac{2}{\lambda^2}\mathbb{E}(X)\right] = \lambda\left[\frac{2}{\lambda^2}\frac{1}{\lambda}\right] = \frac{2}{\lambda^2} = \mathbb{E}(X^2)
  \end{align*}

  \[
  \therefore \Var(\mathcal X) = \mathbb E(\mathcal X^2) - (\frac1\lambda)^2 = \frac2{\lambda^2} - \frac1{\lambda^2} = \frac1{\lambda^2}
  \]
}

\section{Desviación estándar}

\dfn{Desviación estándar}{
  Sea $\mathcal X$ una variable aleatoria con varianza $\Var(\mathcal X)$, entonces se define la \textbf{desviación estándar} de $\mathcal X$ de la siguiente forma.
  \[\sigma = \sqrt{\Var(\mathcal X)}\]
}

\section{Teorema de Chebyshev}
\thm{Teorema de Chebyshev}{
  Sea $\mathcal X$ una variable aleatoria con esperanza $\mu$ y desviación estándar $\sigma$, entonces para cualquier constante $k > 0$ la probabilidad es al menos $1 - \frac1{k^2}$ de que $\mathcal X$ asumirá un valor dentro de $k$ desviaciones estándar de la media (valor esperado), es decir,
  \[P(|\mathcal X - \mu| \le k \sigma) = P(-k \sigma \le \mathcal X - \mu \le k \sigma) \ge 1 - \frac1{k^2}\]
}

\ex{Esperanza y Varianza}{
  Sea $X \sim \Gamma(\alpha, \lambda)$. Calcula $\mathbb{E}(X)$ y $\Var(X)$.
  \begin{itemize}
  \item Por un lado
  \begin{align*}
    \mathbb{E}(X) &= \int_{-\infty}^{\infty} x\frac{\lambda^\alpha}{\Gamma(\alpha)}x^{\alpha-1}e^{-\lambda x}\mathbb 1_{(0,\infty)}^{(x)}dx = \int_0^{\infty} x\frac{\lambda^\alpha}{\Gamma(\alpha)}x^{\alpha-1}e^{-\lambda x}dx \\
  &= \frac{1}{\lambda}\int_0^{\infty} \frac{\lambda^{\alpha+1}}{\Gamma(\alpha)}x^{\alpha+1-1}e^{-\lambda x}dx \quad \text{Usando que } \Gamma(\alpha+1) = \alpha\Gamma(\alpha) \text{ sustituimos,}\\
  &= \frac{\alpha}{\lambda}\int_0^{\infty} \frac{\lambda^{\alpha+1}}{\Gamma(\alpha+1)}x^{\alpha+1-1}e^{-\lambda x}dx = \frac{\alpha}{\lambda}\int_0^{\infty} f_{X'}(x)dx \quad \text{Donde } X' \sim \Gamma(\alpha+1,\lambda)\\
  &= \frac{\alpha}{\lambda}\int_0^{\infty} f_{X'}(x)dx = \frac{\alpha}{\lambda}
  \end{align*}

  \item Por otro lado para calcular $\Var(X)$ primero calculamos lo siguiente,
  \begin{align*}
  \mathbb{E}(X^2) &= \int_0^{\infty} \frac{\lambda^\alpha}{\Gamma(\alpha)}x^{\alpha+2-1}e^{-\lambda x}dx = \frac{1}{\lambda^2}\int_0^{\infty} \frac{\lambda^{\alpha+2}}{\Gamma(\alpha)}x^{\alpha+2-1}e^{-\lambda x}dx
  \end{align*}
  Realizamos una sustitución similar a la del inciso anterior usando que:

  \[\Gamma(\alpha + 2) = (\alpha + 1)\Gamma(\alpha + 1) = (\alpha + 1)(\alpha)\Gamma(\alpha)\]

  \[\mathbb{E}(X^2) = \frac{\alpha(\alpha + 1)}{\lambda^2} \underbrace{\int_0^{\infty} \frac{\lambda^{\alpha+2}}{\Gamma(\alpha+2)}x^{\alpha+2-1}e^{-\lambda x}dx}_{1}\]

  Usando el mismo argumento del inciso anterior: $(X' \sim \Gamma(\alpha + 2, \lambda))$.

  \[ \mathbb E(\mathcal X^2) = \frac{\alpha(\alpha + 1)}{\lambda^2}\]
  Finalmente usamos la propiedad 5 de la varianza.

  \[\Var(X) = \mathbb{E}(X^2) - [\mathbb{E}(X)]^2 = \frac{\alpha(\alpha + 1)}{\lambda^2} - \left(\frac{\alpha}{\lambda}\right)^2 = \frac{\alpha}{\lambda^2}\]

  \end{itemize}
}

