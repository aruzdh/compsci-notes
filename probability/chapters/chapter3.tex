\chapter{Espacio de Probabilidad}

\nt{
  Recordemos que los subconjuntos de $\Omega$ se llaman \textbf{eventos.}
}

\nt{
  Una $\sigma$-algebra es una \textbf{familia de subconjuntos} que satisface lo siguiente.
  \begin{itemize}
    \item $\Omega \in \mathcal F$
    \item $A \in \mathcal F \Rightarrow A^c \in \mathcal F$
    \item $A_1, A_2, \cdots \in \mathcal F \Rightarrow \bigcup_n A_n \in \mathcal F$

  \end{itemize}
}

\section{Axiomas de Probabilidad}

Sean \textit{A, B} eventos,$\Omega$ el espacio muestral, y $\mathcal F$ una $\sigma$-algebra.

\begin{enumerate}
  \item $0 \le P(A) \le 1$
  \item $P(\Omega) = 1$
  \item Sean $A_1, A_2, \cdots \in \mathcal F \text{ (eventos) tales que } A_i \cap A_j = \varnothing, ~\forall i \neq j.$ Entonces $$P\parens*{\bigcup A_i} = \sum_i P(A_i)$$
\end{enumerate}

\section{Proposiciones}
Dado \textit{A, B} eventos.
\begin{enumerate}
  \item $P(A^c) = 1 - P(A)$
    \begin{myproof}
      Sabemos que $\Omega = A \cup A^c$ y $A \cap A^c = \varnothing$. \\
      Usando los \textit{axiomas de probabilidad} se tiene que $P(\Omega) = P(A) + P(A^c)$.\\
      De lo anterior se sigue que $1 = P(A) + P(A^c)$
      \begin{center} \boxed{\therefore \, P(A^c) = 1 - P(A)} \end{center}
    \end{myproof}

    \newpage
  \item Si $A \subseteq B$, entonces $P(A) \le P(B)$
    \begin{myproof}
      Notemos que $B = (B - A) \cup A$.\\ 
      Por los \textit{axiomas} se tiene que $P(B) = P(B-A) + P(A)$ con $P(B - A) \ge 0$.
      \begin{center} \boxed{\therefore \, P(A) \le P(B) } \end{center}
    \end{myproof}
    
  \item $P(A \cup B) = P(A) + P(B) - P(A \cap B)$
    \begin{myproof}
      Se tiene que $A \cup B = (A - B) \cup (A \cap B) \cup (B - A)$ donde cada uniendo es ajeno.\\
      Entonces
      \begin{align*}
         P(A \cup B) &= P((A - B) \cup (A \cap B) \cup (B - A))\\
                  &= P(A-B) + P(A \cap B) + P(B-A)\\
                  &= P(A) + P(B-A)\\
                  &= P(A)  + P(B) - P(B \cap A)\\
      \end{align*}
    \end{myproof}
\end{enumerate}

\ex{Proposiciones}{
  Se sacan 3 pelotas de una urna que contiene 5 pelotas azules, 4 verdes, y 3 rojas. ¿Cuál es la probabilidad de no obtener ni verdes ni rojas?
  \newpara
  Se tienen 12 pelotas en total. Sea \textit{A = "obtener 3 pelotas azules"} un evento.\\
  Entonces 
  \begin{align*}
    P(A) &= \frac{\abs*{A}}{\abs*{\Omega}} = \frac{\binom{5}{3}}{\binom{12}{3}}\\
         &= \frac{ \frac{5!}{3! 2!} }{ \frac{12!}{3! 9!} } = \frac{ \frac{5 \cdot 4 \cdot 3!}{3! \cdot 2 \cdot 1} }{ \frac{12 \cdot 11 \cdot 10 \cdot 9!}{3 \cdot 2 \cdot 1 \cdot 9!} } \\
         &= \frac{ \sfrac{20}{2} }{\sfrac{1320}{6} } = \frac{10}{220} = \frac{1}{22}
  \end{align*}
}

\ex{Proposiciones} {
  Tres pelotas son tomadas aleatoriamente de una caja que contiene 6 pelotas blancas y 5 negras. ¿Cuál es la probabilidad de que una de las pelotas sea blanca y las otras negras?
  \newpara
  Se tienen 11 pelotas en total. Sean \textit{A = "sacar 1 pelota blanca"} y \textit{B = "sacar 2 pelotas negras"} eventos.
  Entonces 
  \begin{align*}
    P(A) &= \frac{\abs*{A} \abs*{B}}{\abs*{\Omega}} = \frac{ \binom61 \binom52} {\binom{11}3} \\ 
        &= \frac{ \frac{6!}{6!} \frac{5!}{2! 3!} } { \frac{11!}{3!8!} } \\ 
        &= \frac{60} {165} = \frac{4}{11}
  \end{align*}
}


\section{Probabilidad Condicional}
\subsection{Definición}
\dfn{Probabilidad Condicional}{
  Sean $A, B \in \mathcal F$. La \textbf{probabilidad condicional} del evento \textit{A} \textit{dado} el evento \textit{B} se define como sigue.
  \[P(A \mid B) = \frac{P(A \cap B)}{P(B)} \]
  donde necesariamente $P(B) > 0$.
}

\nt {
  Lo anterior se puede leer como "la \textbf{probabilidad de $A$ dado $B$"}
}

\nt {
  De la definicón anterior se tiene que 
  \[P(A \cap B) = P(A \mid B)\,P(B)\]
}

\ex{Probabilidad Condicional}{
  Se extraen 2 cartas en sucesión de una baraja de 52. Encuentra la \textit{probabilidad de que ambas cartas sean ases si}
  \begin{itemize}
    \item Se reemplaza la carta (se regresa la carta sacada).
    \item No se reemplaza la carta.
  \end{itemize}

  Sea \textit{$A_1$ = "La primera carta es un As"}, \textit {$A_2$ = "La segunda carta es un As"} eventos. Entonces\\
  $P(A_2 \cap A_1) = P(A_2 \mid A_1) ~P(A_1) = P(A_2 \mid A_1)~ \frac{4}{52}$

  Si se reemplaza la carta tenemos que 
  \[P(A_2 \cap A_1) = (\frac{4}{52})^2 = \frac{1}{169}\]
  Si no se reemplaza tenemos que
  \[P(A_2 \cap A_1) = \frac{3}{51} \cdot \frac{4}{52} = \frac{1}{221} \]
}

\section{Independencia}
\subsection{Definición}

\dfn{Independencia}{
  Sean $A, B \in \mathcal F$. Decimos que \textit{A} y \textit{B} son \textbf{independientes} si
  \[P(A \mid B) = P(A) \qquad \text{ó} \qquad P(B \mid A) = P(B)\]
  dependiendo del contexto.
}

\nt {
  De la definicón anterior se tiene que 
  \[P(A \cap B) = P(A)\,P(B)\]
  por definicón de probabilidad condicional.
}

\nt {
  Dado $P(A \mid B) = P(A)$, se puede decir que "\textbf{la occurrencia de $B$ no afecta en la probabilidad de $B$}".
}

\nt {
  La independencia \textbf{no} implica que $A \cap B = \varnothing$
}

\section{Probabilidad Total}
\subsection{Partición del Espacio Muestral}

\dfn{Partición del Espacio Muestral}{
  Sean $B_1, B_2, \cdots, B_n \in \mathcal F$. Decimos que $\{B_i\}_i$ es una \textbf{partición de $\Omega$} si se satisface lo siguiente.
  \begin{itemize}
    \item $\forall i,j;~ B_i \cap B_j = \varnothing$
    \item $\Omega = \bigcup_i B_i$
  \end{itemize}
}

\subsection{Probabilidad Total}
\dfn{Probabilidad Total}{
  Si $\{B_i\}_i$ es una partición de $\Omega$, y $A \in \mathcal F$ un evento, entonces
  \[P(A) = \sum_{i = 1}^n P(A \cap B_i) = \sum_{i = 1}^n P(A \mid B_i)\,P(B_i)\]
}

\nt {
  A lo anterior también se le llama "\textbf{Ley de probabilidad total}".
}

\nt {
  De la definicón anterior se tiene que \textit{\textit A está distribuido sobre diferentes partes \textbf{disjuntas} de la partición}.
}

\ex{Probabilidad Total}{
  Se tienen dos cajas con pelotas. La caja \textit{I} tiene 2 pelotas azules y 3 rojas. La caja \textit{II} tiene 8 pelotas azules y 2 rojas. Se lanza una moneda justa. Si obtenemos \textit{sol} sacamos una pelota de la caja \textit{I}, y si se obtiene \textit{águila} se saca de la caja \textit{II}. ¿ Cuál es la probabilidad de sacar una pelota roja?
  \newpara
  Sean \textit{A = "se saca una pelota roja"}, \textit{$B_1$ = "se saca la pelota de la caja I"}, \textit{$B_2$= "se saca la pelota de la caja II"} eventos.  Entonces
  \begin{align*}
    P(A) &= P(A \cap B_1) + P(A \cap B_2)\\
         &= P(A \mid B_i)\,P(B_i) + P(A \mid B_2)\,P(B_2)\\
         &= \frac{3}{5} \cdot \frac{1}{2} + \frac{2}{10} \cdot \frac{1}{2} = \frac{3}{10} + \frac{2}{20} = \frac25
  \end{align*}
}

\section{Teorema de Bayes}
\subsection{Motivación}

Siguiendo con el ejemplo anterior. Supongamos que durante el experimento se sacó una pelota azul. ¿Cuál es la probabilidad de que la pelota se haya sacado de la caja \textit{II}? Se tiene que
\[P(B_2 \mid A^c) = \frac{B_2 \cap A^c}{P(A^c)} = \frac{P(A^c \mid B_2)\,P(B_2)}{\sum_{i = 1}^2 P(A^c \mid B_i)\,P(B_i)}\]

Donde 
\[P(A^c) = P(A^c \cap B_1) + P(A^c \cap B_2) = P(A^c \mid B_1)\,P(B_1) + P(A^c \mid B_2)\,P(B_2)\]

\subsection{Definición}
\dfn{Teorema de Bayes}{
  Sea $\{B_i\}_i$ es una partición de $\Omega$ y $A \in \mathcal F$ un evento. Entonces
  \[P(B_j \mid A) = \frac{P(B_j \cap A)}{P(A)}= \frac{P(A \mid B_j)\,P(B_j)}{\sum_{i = 1}^n P(A \mid B_i)\,P(B_i)}\]
}

\nt {
  De la definicón anterior se observa que $P(A)$ es la probabilidad total.
}

\ex{Teorema de Bayes}{
  Una prueba de sagre es 95\% efectiva en detectar cierta enfermedad cuando esta enfermedad está presente. Sin embargo, la prueba también puede dar falsos positivos para el 1\% de quienes se hacen la prueba y están sanos. Si el 0.5\% de la población tiene la enfermedad, ¿Cuál es la probabilidad de que una persona que dió positivo a la prueba tenga esta enfermedad?
  \newpara
  Sea \textit{A = "la persona dió positivo"}, \textit{B = "la persona tiene la enfermedad"} eventos.\\
  Además, $\Omega = B \cup B^c$ donde $P(B) = 0.5\% = 0.005$. $P(A \mid B) = 0.95$. $P(A \mid B^c) = 0.1$.
  Entonces
  \begin{align*}
    P(B \mid A) &= \frac{P(B \cap A)}{P(A)} = \frac{P(A \mid B)~ P(B)}{P(A)}\\
             &= \frac{P(A \mid B)~P(B)}{P(A | B)~P(B) + P(A \mid B^c)~ P(B^c)}\\
             &= \frac{(0.95) (0.005)}{(0.95) (0.005) + (0.1) (1 - 0.005)}\\
             &= \frac{19}{417}
  \end{align*}
}

\ex{Teorema de Bayes}{
  Tengo una bolsa con 3 dados, donde uno es de 4 caras, uno de 6 caras, y una de 12 caras. Tomo aleatoriamente un de ellos y lo lanzo. Dado que se obtuvo un 4, ¿Cuál es la probabilidad de que haya lanzado el dado de 6 caras?
  \newpara
  Sean \textit{A = "lanzar el dado de 4 caras"}, \textit{B = "lanzar el dado de 6 caras"}, \textit{C = "lanzar el dado de 12 caras."}, \textit{D = "obtener un 4"} eventos.

  De lo anterior tenemos que $P(A) = P(B) = P(C) = \frac{1}{3}$. $P(D \mid A) = \frac14,~ P(D \mid B) = \frac16,~ P(D \mid C) = \frac{1}{12}$\\
  Entonces
  \begin{align*}
    P(B \mid D) &= \frac{P(B \cap D)}{P(D)}\\
              &= \frac{P(D \mid B)~P(B)}{P(D)}\\
              &= \frac{P(D \mid B)~P(B)}{ P(D \mid A)~P(A) + P(D \mid B)~P(B) + P(D \mid C)~P(D)}\\
              &= \frac{\sfrac16}{\sfrac16 + \sfrac14 + \sfrac1{12}} = \frac{\sfrac16}{\sfrac12} = \frac13
  \end{align*}
}

\ex{Teorema de Bayes}{
  Una bolsa contiene 4 pelotas blancas y 2 negras. Otra bolsa contiene 3 blancas y 5 negras. Si se lanza un dado y si sale \textit{1, 2, 3 ó 4} se toma la pelota de la bolsa 1, de lo contrario se toma de la bolsa 2. Encuentra la probabilidad de que la pelota extraida sea negra.
  \newpara
  Sean \textit{A = "la pelota es negra"}, \textit{B = "se extrae de I"}, \textit{C = "se extrae de II"} eventos.
  Con lo anterior tenemos que $P(B) = \frac46,~ P(C) = \frac26$\\
  Entonces
  \begin{align*}
    P(A) &= P(A \mid B)~P(B) + P(A \mid C)~P(C)\\
          &= \frac26 \cdot \frac46 + \frac58 \cdot \frac26 = \frac26(\frac46 + \frac58) = 
          \frac{62}{48} \cdot \frac13 = \frac{62}{144} = \frac{31}{72}
  \end{align*}
}

\section{Sucesiones Monótonas}
\subsection{Límite de una Sucesión Creciente}

\dfn{Límite de una Sucesión Creciente}{
  Sea $A_1 \subseteq A_2 \subseteq A_3 \subseteq \cdots A_n \subseteq \cdots$ una sucesión de eventos creciente. Definimos 
  \[\lim_{n \rightarrow \infty} A_n = \bigcup_{i=1}^{\infty} A_i\]
}

\subsection{Límite de una Sucesión Decreciente}
\dfn{Limite de una Sucesión Decreciente}{
Sea $A_1 \supseteq A_2 \supseteq A_3 \supseteq \cdots A_n \supseteq \cdots$ una sucesión de eventos decreciente. Definimos
\[\lim_{n \rightarrow \infty} A_n = \bigcap_{i=1}^{\infty} A_i\]
}

\subsection{Definición}
\dfn{Sucesión Monótona}{
  Si la sucesión $\{A_n\}_{n \ge 1}$ es creciente o decreciente, la llamamos \textbf{sucesión monótona}.
}

\subsection{Teorema}
\thm{}{
  Sea $\{A_n\}_{n \ge 1}$ una sucesión monótona de eventos. Entonces
  \[
    \lim_{n \to \infty} P(A_n) = P \parens*{\lim_{n \to \infty} A_n} =
    \begin{cases}
      P\parens*{\bigcup_{i=1}^{\infty} A_i}& \{A_{n}\} \text{ es creciente}\\[1em]
      P\parens*{\bigcap_{i=1}^{\infty} A_i} & \{A_{n}\} \text{ es decreciente}
    \end{cases}
  \]
}

\begin{myproof}
  Supongamos que $\{A_n\}$ es creciente, es decir, $A_1 \subset A_2 \subset A_3 \subset \cdots$\\
  Consideremos la partición $\{B_n\}$ donde $B_1 = A_1,~ B_2 = A_2 - A_1, ~ B_3 = A_3 - A_2, \cdots,~ B_n = A_n - A_{n-1}$, tal que cumple lo siguiente.
  \begin{itemize}
    \item $\forall i \neq j,~ B_i \cap B_j = \varnothing$
    \item $\bigcup_{i = 1}^n B_i = A_n$
    \item $\bigcup_{i = 1}^{\infty} B_i = \bigcup_{i = 1}^{\infty} A_i$
  \end{itemize}

  Entonces
  \begin{align*}
    \lim_{n \to \infty} P(A_n) &= \lim_{n \to \infty} P\parens*{\bigcup_{i = 1}^n B_i}\\
                               &= \lim_{n \to \infty} \sum_{i = 1}^n P(B_i) \quad \text{(axiomas de probabilidad)}\\
                               &= \sum_{i = 1}^{\infty} P(B_i)\\
                               &= P\parens*{\bigcup_{i = 1}^{\infty}B_i}
                               \quad \text{(axiomas de probabilidad)}\\ 
                               &= P\parens*{\bigcup_{i = 1}^{\infty}A_i}= P\parens*{\lim_{n \to \infty} A_n)}
  \end{align*}
  Ahora supongamos que $\{A_n\}$ es decreciente, es decir, $A_1^c \subset A_2^c \subset A_3^c \cdots$. Como $\{A_n\}$ es decreciente, entonces $\{A^c\}$ es creciente. Así que
  \begin{align*}
    \lim_{n \to \infty} P(A_n^c) &= P\parens*{\lim_{n \to \infty}A_n^c}\\
                        &= P\parens*{\bigcup_{i = 1}^{\infty}A_i^c}\\
                        &= \lim_{n \to \infty} 1 - P(A_n) \\
                        &= 1 - P\parens*{\bigcap^{\infty}A_n}\\
  \end{align*}

  De la anterior se obtiene que 
  \[1 - \lim_{n \to \infty} P(A_n) = 1 - P\parens*{\bigcap^{\infty}A_n}\]

  Por tanto
  \[\lim_{n \to \infty}P(A_n) = P\parens*{\bigcap^{\infty}A_n} = P\parens*{\lim_{n \to \infty}A_n}\]
\end{myproof}

\newpage
\section{Ejercicios}

\qs{Espacio de Probabilidad}{
  Demuestra que:
  \[P(A^{c}|B)=1-P(A|B)\]
}

\qs{Espacio de Probabilidad}{
  Demuestra que si A y B son independientes, entonces $A^{c}$ y B también son independientes.
}

\qs{Espacio de Probabilidad}{
  Demuestra que:
  \[P(A\cap B)\ge P(A)+P(A\cap B)-1\]
}

\qs{Espacio de Probabilidad}{
  Cuatro personas han respondido a una solicitud de un banco de sangre para efectuar donaciones. Se desconoce sus tipos de sangre. Solo se desea el tipo de sangre $O+$ y solo una de ellas tiene este tipo de sangre. Si se seleccionan de forma aleatoria para determinar su tipo de sangre, ¿cuál es la probabilidad de que por lo menos 3 tengan que ser examinadas para así obtener a la persona donante O+?
}

\qs{Espacio de Probabilidad}{
  Benito tiene 3 cuentas distintas de correo electrónico. El 70\% de sus correos llegan a la cuenta 1, el 20\% a la cuenta 2 y el resto a la cuenta 3. El 1\% de los mensajes que llegan a la cuenta 1 es spam, mientras que el 2\% y el 5\% son spam en las cuentas $2$ y $3,$ respectivamente. ¿Cuál es la probabilidad de que abra al azar un correo de spam de cualquiera de las 3 cuentas?
}

\qs{Espacio de Probabilidad}{
  En una gasolinera el 40\% de los clientes utilizan gasolina regular, el 35\% gasolina plus y el 25\% premium. De los que utilizan regular solo el 30\% llenan sus tanques, de los que utilizan plus el 60\% llenan sus tanques, mientras que de los que utilizan premium el 50\% los llenan.
  \begin{itemize}
    \item ¿Cuál es la probabilidad de que el siguiente cliente pida gasolina plus y llene el tanque?
    \item ¿Cuál es la probabilidad de que el siguiente cliente llene el tanque?
    \item Si el siguiente cliente llena el tanque, ¿cuál es la probabilidad de que pida gasolina regular, gasolina plus o gasolina premium?
  \end{itemize}
}

\qs{Espacio de Probabilidad}{
  Demuestre o proporcione un contraejemplo:
  \begin{itemize}
    \item Si $A\cap B=\emptyset$ entonces $P(A)\le P(B^{\epsilon})$.
    \item Si $P(A)=P(B)=1$, entonces $P(A\cap B)=0$.
  \end{itemize}
}

\newpage
\qs{Espacio de Probabilidad}{
  10 personas tienen el mismo tipo y marca de teléfono celular. Suponga que antes de presentar un examen estas personas ponen su celular en una caja. Los teléfonos son mezclados dentro de la caja. Al final del examen cada persona toma aleatoriamente un celular ¿Cuál es la probabilidad de que al menos una persona elijan su propio celular? ¿Cuál es la probabilidad de que nadie seleccione su propio celular?
}

\qs{Espacio de Probabilidad}{
  Las garrapatas de venados pueden ser portadoras de la enfermedad de Lyme o de la Ehrlichiosis Granulocítica Humana (EGH). En cierto lugar el 16\% de todas las garrapatas portan la enfermedad de Lyme, 10\% portan EGH y 10\% de las garrapatas que portan por lo menos una de estas enfermedades en realidad portan ambas. Si una garrapata seleccionada al azar ha sido portadora de EGH, ¿cuál es la probabilidad de que la garrapata seleccionada también porte la enfermedad de Lyme?
}

\qs{Espacio de Probabilidad}{
  Una ciudad tiene dos carros de bomberos que operan de forma independiente. La probabilidad de que un carro específico esté disponible cuando se necesite es 0.96.
  \begin{itemize}
    \item ¿Cuál es la probabilidad de que ningún carro esté disponible cuando se le necesite?
    \item ¿Cuál es la probabilidad de que sólo un carro de bomberos esté disponible cuando se le necesite?
  \end{itemize}
}

\qs{Espacio de Probabilidad}{
  Una de cada diez personas en una población determinada sufre de enfisema pulmonar. Si se eligen 15 personas al azar, ¿cuál es la probabilidad de que al menos una de estas personas tenga enfisema pulmonar?
}
