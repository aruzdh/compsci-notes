\chapter{Técnicas de Conteo}

\section{Principios de Conteo}
Si tenemos \textit m objetos de tipo $A$, y \textit n objetos de tipo $B$ ($A$ y $B$ tipos arbitrarios y distintos), entonces tenemos $m + n$ objetos en total.

Si tenemos \textit m objetos diferentes de tipo $A$, y \textit n objetos diferentes de tipo $B$ ($A$ y $B$ tipos arbitrarios y distintos), entonces tenemos \textit{mn} distintas maneras de tener un objeto de tipo $A$ y uno de tipo $B$.

\ex{Principios de Conteo}{
  Supongamos que tenemos 4 dados. ¿De cuantas maneras se pueden obtener \textit{4 números diferentes} al lanzar los 4 dados?\\
  Notemos que en el primer lanzamiento tenemos 6 números diferente ($1,2, \cdots, 6$), en el segundo tendremos 5 opciones, y así sucesivamente. \\
  Por tanto, tenemos $6 \cdot 5 \cdot 4 \cdot 3 = 360$ diferentes formas.
}

Dado un problema como el anterior, podemos analizarlo de la siguiente manera.\\
Supongamos que 2 etapas de un experimento se van a desarrollar. Entonces, si la etapa 1 puede resultar en cualquiera de \textit{m} posibilidades, y para cada salida de la etapa 1 hay \textit{n} posibles salidas de la etapa 2, entonces hay \textit{mn} posibles salidas de las dos etapas.

\ex{Principios de Conteo}{
  Una familia se acaba de cambiar a una ciudad nueva y requiere los servicios tanto de un obstetra como de un pediatra. Existen dos clínicas médicas fácilmente accesibles y cada una tiene 2 obstetras y 3 pediatras. La familia obtendrá los máximos beneficios del seguro de salud si se une a una clínica y selecciona a ambos especialistas de dicha clínica. ¿De cuántas maneras se puede hacer lo anterior?\\
  El experimento consta de seleccionar 1 obstetra y 1 pediatra de una clínica. Entonces podemos considerar las siguientes etapas del experimento.
  \begin{enumerate}
    \item Elegir una clínica: 2 posibilidades.
    \item Elegir el obstetra: 2 posibilidades.
    \item Elegir el pediatra 3 posibilidades.
  \end{enumerate}
  Por tanto, hay $2 \cdot 2 \cdot 3 = 12$ maneras.\\
  Ahora, si suponemos que no importa la clínica elegida, tenemos $4 \cdot 6$ formas.
}

\ex{Principios de Conteo}{
  ¿Cuántas placas son posibles formar si las primeras 3 posiciones deben estar ocupadas por letras y las últimas 4 por dígitos?\\
  Se tienen las siguientes etapas.
  \begin{itemize}
    \item Primera posición: 27 posibilidades.
    \item Segunda posición: 27 posibilidades.
    \item Tercera posición: 27 posibilidades.
    \item Cuarta posición: 10 posibilidades.
    \item Quinta posición: 10 posibilidades.
    \item Sexta posición: 10 posibilidades.
    \item Séptima posición: 10 posibilidades.
  \end{itemize}
  Así, son posibles $27 \cdot 27 \cdot 27 \cdot 10 \cdot 10 \cdot 10 \cdot 10 = 27^3 \cdot 10^4$ placas.
}

\ex{Principios de Conteo}{
  Se lanza un dado y una moneda. ¿Cuántos elementos tiene $\Omega$?\\
Tenemos 6 posibilidades para el dado, y 2 para la moneda. Por tanto $\abs*{\Omega} = 12$
}

\section{Permutaciones}

Las permutaciones son las distintas formas de ordenar un conjunto de objetos. Existen distintos tipos de ordenamientos dependiendo de las condiciones dadas o requeridas.

\subsection{Permutaciones lineales}

Como el nombre indica, no se establece ninguna restricción sobre la manera en que los objetos deben ser ordenados, por lo que buscamos \textbf{todas} los posibilidades. Para lograrlo usamos el factorial de $n$, donde $n$ es el total de objetos a ordenar.

\[n! = n \cdot (n - 1) \cdot (n - 2) \cdot \cdots \cdot 2 \cdot 1\]

Existen situaciones donde debemos ordenar $n$ objetos en $r$ distintas posiciones, con $r < n$. Para estos casos usamos la siguiente formula.
\[\frac{n!}{(n-r)!}\]

\nt{
  Notemos que si $n = r$, $(n-r)! = 0! = 1$. Por tanto se obtiene la primera formula.
}

\ex{Permutaciones}{
  ¿De cuántas maneras se pueden ordenar las letras \textit{a, b, c}?\\
  Considemos 3 etapas. En la primera, tenemos 3 posibilidades (todas las  letras disponibles); en la sugunda tenemos solo 2 posibilidades; y en la última solo 1 posibilidad. Es decir, $3 \cdot 2 \cdot 1 = 3! = 6$
}

\ex{Permutaciones}{
  Si un equipo de baseball tiene 9 jugadores, ¿Cuántos alineamientos son posible?\\
  Usando el mismo procedimiento que en el ejemplo pasado tenemos $9 \cdot 8 \cdot \cdots 2 \cdot 1 = 9!$ alineamientos.
}

\ex{Permutaciones}{
  En una clase de probabilidad hay 6 mujeres y 4 hombres. Se les asigma un examen y son rakeados sugún su desempeño en el examen. Supongamos que no hay 2 personas con la misma calificación. 
  \begin{itemize}
    \item ¿Cuántos rakings son posible?

      Dado que hay $6 + 4 = 10$ personas en total, se tienen $10!$ rakings.
    \item ¿Cuántos rankings son posibles si los hombres son rankeados entre ellos y lo mismo sucede para las mujeres?

      Para los hombres hay $6!$ rakings, y para las mujeres $4!$. Por tanto, hay $6! \cdot 4!$ rakings totales.
  \end{itemize}
}

\ex{Permutaciones}{
  De entre los 24 miembros de un club se sacan 4 nombres para los puestos de presidente, vicepresidente, tesorero, y secretario. ¿De cuántas maneras diferentes se puede hacer esto?\\
  Se tienen $24 \cdot 23 \cdot 22 \cdot 21 = \frac{24!}{20!}$
}

\nt {
  En este último ejemplo se usa la formula donde $n \neq r$ dada al comienzo de la sección.
}

\nt {
  Si tenemos \textit{A, B, C}, los ordenamientos \textit{ABC} y \textit{CBA} son distintos y ambos deben considerse.
  Esta observación toma más sentido en el subsección de \textit{combinaciones}.
}

\subsection{Permutaciones Circulares}

A diferencia de las permutaciones lineales (donde hay un "principio" y un "fin"), en las permutaciones circulares los elementos se disponen alrededor de un círculo. La característica clave que las distingue es que las rotaciones de un mismo arreglo se consideran idénticas.

Para calcular el número de permutaciones circulares de n objetos distintos, se utiliza la siguiente fórmula
\[(n - 1)!\]

\nt{
  La lógica detrás de esta fórmula es que, para evitar contar los arreglos rotados como distintos, se "fija" la posición de uno de los elementos. Una vez que un elemento está fijo, los n−1 elementos restantes se pueden ordenar de (n−1)! maneras diferentes, como si estuvieran en una línea recta.
}

\subsection{Permutaciones Indistinguibles}
En este caso, deseamos contar el número de ordenamientos cuando existen elementos repetidos indistinguibles.\\
Si tenemos \textit{n} objetos en total, y de los \textit{n} objetos, $n_1,n_2, \cdots, n_r$ son indistinguibles con $n_1 + n_2 + \cdots + n_r \le n$, entonces las permutaciones son
\[\frac{n!}{n_1!n_2!\cdots n_r!}\]

\ex{Permutaciones Indistinguibles}{
  Sea \textit{PEPPER} una palabra. ¿De cuántas formas se puede ordenar las letras de la palabra?\\
  Dada la información que precede al ejemplo, se tiene que
  \[\frac{6!}{3! \cdot 2! \cdot 1!} = \frac{6 \cdot 5 \cdot 4}{2} = 60\] es el número de ordenamientos, donde $3!$ representa las 3 \textit{P's}, $2!$ las 2 \textit{E's}, y $1!$ la \textit{R}.

  Notemos que los factores $1!$ carecen de importancia ya que no aportan nada.
}

\section{Combinaciones}

\subsection{Coeficiente Binomial}

Supongamos que estamos interesados en contar cuantos subconjuntos de \textit{r} elementos podemos tomar de \textit{n} objetos totales ($r \le n)$. Este número es el \textbf{coeficiente binomial}.
\[\binom{n}{r} = \frac{n!}{r!(n - r)!}\]

\nt{
  Cuando hablamos de combinaciones, el orden de la selección \textbf{no} importa. Es decir, si eliges un grupo de objetos, y ese mismo grupo de objetos es reordenado, sigue siendo la misma combinación.
}

\ex{Coeficiente Binomial}{
  ¿Cuántos subconjuntos de 3 letras se pueden formar de las letras \textit{A,B,C,D,E}?\\
  Podemos observar las siguientes etapas del experimento.
  \begin{enumerate}
    \item Tenemos cinco posibilidades (todas las letras disponibles).
    \item Tenemos cuatro posibilidades.
    \item Tenemos tres posibilidades.
  \end{enumerate}
  Dado que dentro de estas tres etapas existen repeticiones (el conjunto \textit{ABC} es igual que \textit{BCA}), tenemos lo siguiente.
  \[\frac{5 \cdot 4 \cdot 3}{3 \cdot 2 \cdot 1} \cdot \frac{2 \cdot 1}{2 \cdot 1} = \frac{5!}{3! \cdot 2!} = \binom53\]
}

\ex{Coeficiente Binomial}{
  Si tenemos 5 mujeres y 7 hombres. ¿Cuántos comités de 2 mujeres y 3 hombres se pueden formar?\\
  Por un lado tenemos $\binom52$ combinaciones de 2 mujeres. Por otro, $\binom73$ de hombres.\\ Por tanto tenemos \[\binom52 \cdot \binom73\] comités totales.
}

\ex{Coeficiente Binomial}{
  ¿Cuántos subconjuntos tiene un conjunto de 5 elementos?\\
  Dado que buscamos \textbf{cada} posible subconjunto, tenemos
  \begin{align*}
    \sum_{i = 0}^5 \binom5i\ &= \binom50 + \binom51 + \binom52 + \binom53 + \binom54 + \binom55\\
    &= \frac{5!}{0! 5!} + \frac{5!}{1!4!} + \frac{5!}{2!3!} + \frac{5!}{3!2!} + \frac{5!}{4! 1!} + \frac{5!}{5! 0!}\\
    &= 1 + 5 + 10 + 10 + 5 + 1 = 32 \\
  \end{align*}
}

\subsection {Coeficiente Multinomial}
Un conjunto de \textit{n} distintos objetos va a ser dividido en \textit{r} distintos subgrupos de tamaño $n_1, n_2, \cdots, n_r$ donde $n_1 + n_2 + \cdots + n_r = n$. ¿De cuántas formas podemos hacer esto?
\[\binom{n}{n_1,n_2,\cdots,n_r} = \frac{n!}{n_1!n_2!\cdots n_r!}\]

\ex{Coeficiente Multinomial}{
  Tenemos el conjunto de letras \textit{A,B,C,D,E} y queremos dos grupos con 3 y 2 elementos cada uno.\\
  Entonces, tenemos lo siguiente para el primer grupo.
  \[\frac{5 \cdot 4 \cdot 3}{3!}\]
  Para el segundo grupo se tiene lo siguiente.
  \[\frac{2 \cdot 1}{2!}\]
  Por tanto, el total es 
  \[\frac{5!}{3! \cdot 2!}\]
}

\thm{}{
  \[(x_1 + x_2 + \cdots + x_r)^n = \sum_{n_1,\cdots, n_r} \binom{n}{n_1, \cdots,n_r} x_1^{n_1}x_2^{n_2}\cdots x_r^{n_r} \]
  Es decir, sumamos sobre todos los vectores cuyas entradas son enteras no negativas tales que $n_1 + \cdots + n_r = n$
}

\section{Ejercicios}

\qs{Conteo}{
  10 personas se van a dividir en equipo \textit{A} y \textit{B}, cada una con 5 integrantes. El equipo \textit{A} jugará en una liga y el \textit{B} en otra. ¿Cuántos diferentes equipos son posibles?
}

\qs{Conteo}{
  Para jugar basketball, 10 niñas se dividen en 2 grupas de 5 integrantes. ¿Cuántas divisiones son posibles?
}

\qs{Conteo}{
  Consideremos un conjunto de 11 antenas de las cuales 3 están defectuosas. Defectuosas y funcionales son consideradas Indistinguibles. ¿Cuántos ordenamientos en linea obtenemos si no hay 2 defectuosas consecutivas?
}

\qs{Conteo}{
  \textit{n} pelotas diferentes van a ser distribuidas en \textit{r} urnas. ¿De cuántas formas se pueden distribuir?
}

\qs{Conteo}{
  Supongamos que tenemos 8 pelotas iguales que van a ser distribuidas en 3 urnas. Si todas las urnas deben tener al menos una pelota ¿De cuántas formas se pueden distribuir?
}

\qs{Conteo}{
  ¿Cuántos diferentes arreglos se pueden formar de las letras de las siguientes palabras?
  \begin{itemize}
    \item FLUKE
    \item PROPOSE
    \item MISSISSIPPI
    \item ARRANGE
  \end{itemize}
}

\qs{Conteo}{
  Una niña tiene 12 bloques de lego de los cuales 6 son negros, 4 rojos, 1 blanco, y 1 azul. Si la niña pone los bloques en línea, ¿cuántos arreglos son posibles?
}

\qs{Conteo}{
  De cuántas maneras se pueden sentar 8 personas una junto a la otra en cada escenario.
  \begin{itemize}
    \item No hay restricciones.
    \item Las personas \textit{A} y \textit{B} deben estar sentadas una junto a la otra.
    \item Hay 4 hombres y 4 mujeres, y 2 hombres o 2 mujeres no pueden sentarse uno al lado del otro.
    \item Hay 5 hombres y ellos deben sentarse uno al lado de otro.
    \item Hay 4 parejas de casados y cada pareja debe sentarse junta.
  \end{itemize}
}
