\chapter{Técnicas de Conteo}

\section{Principios de Conteo}
\begin{itemize}
  \item Si tenemos \textit m objetos y \textit n objetos (diferente tipo respecto a \textit m), entonces tenemos $m + n$ objetos en total.
  \item Si tenemos \textit m objetos diferentes y \textit n objetos diferentes (diferente tipo respecto a \textit m), entonces tenemos \textit{mn} distintas maneras de tener un objeto de tipo \textit m y uno de tipo \textit n.
\end{itemize}

\ex{Principios de Conteo}{
  Supongamos que tenemos 4 dados. ¿De cuantas maneras se pueden obtener \textit{4 números diferentes} al lanzar los 4 dados?\\
  Notemas que en el primer lanzamiento tenemos 6 números diferente ($1,2, \cdots 6$), en el segundo tendremos 5 opciones, y así sucesivamente. \\
  Por tanto, tenemos $6 \cdot 5 \cdot 4 \cdot 3$ diferentes formas.
}

Dado un problema como el anterior, podemos analizarlo de la siguiente manera.\\
Supongamos que 2 etapas de un experimento se van a desarrollar. Entonces, si la etapa 1 puede resultar en cualquiera de \textit{m} posibilidades, y para cada salida de la etapa 1 hay \textit{n} posibles salidas de la etapa 2, entonces hay \textit{mn} posibles salidas de las dos etapas.

\ex{Principios de Conteo}{
  Una familia se acaba de cambiar a una ciudad nueva y requiere los servicios tante de un obstetra como de un pediatra. Existen dos clínicas médicas fácilmente accesibles y cada una tiene 2 obstetras y 3 pediatras. La familia obtendrá los máximos beneficios del seguro de salud si se une a una clínica y selecciona a ambos especialistas de dicha clínica. ¿De cuántas maneras se puede hacer lo anterior?\\
  El experimento consta de seleccionar 1 obstetra y 1 pediatra de una clínica. Entonces podemos considerar las siguientes etapas del experimento.
  \begin{enumerate}
    \item Elegir una clínica: 2 posibilidades.
    \item Elegir el obstetra: 2 posibilidades.
    \item Elegir el pediatra 3 posibilidades.
  \end{enumerate}
  Por tanto, hay $2 \cdot 2 \cdot 3 = 12$ maneras.\\
  Ahora, si suponemos que no importa la clínica elegida, tenemos $4 \cdot 6$ formas.
}

\ex{Principios de Conteo}{
  ¿Cuántas placas son posibles formar si las primeras 3 posiciones deben estar ocupadas por letras y las últimas 4 por dígitos?\\
  Se tienen las siguientes etapas.
  \begin{enumerate}
    \item Primera posición: 27 posibilidades.
    \item Sugunda posición: 27 posibilidades.
    \item Tercera posición: 27 posibilidades.
    \item Cuarta posición: 10 posibilidades.
    \item Quinta posición: 10 posibilidades.
    \item Sexta posición: 10 posibilidades.
    \item Septima posición: 10 posibilidades.
  \end{enumerate}
  Son posibles $27 \cdot 27 \cdot 27 \cdot 10 \cdot 10 \cdot 10 \cdot 10 = 27^3 \cdot 10^4$ placas.
}

\ex{Principios de Conteo}{
  Se lanza un dado y una moneda. ¿Cuántos elementos tiene $\Omega$?\\
  Tenemos 6 posibilidades para el dado, y 2 para la moneda. Por tanto $\mid \Omega \mid = 12$
}

\section{Permutaciones}

\subsection{Permutaciones sin restricciones}
\[n!~~~(\text{o hasta donde sea necesario)}\]

\ex{Permutaciones}{
  ¿De cuántas maneras se pueden ordenar las letras \textit{a, b, c}?\\
  Considemos 3 etapas. En la primera, tenemos 3 posibilidades (todas las  letras disponibles); en la sugunda tenemos solo 2 posibilidades; y en la última solo 1 posibilidad. Es decir, $3 \cdot 2 \cdot 1 = 3! = 6$
}

\ex{Permutaciones}{
  Si un equipo de baseball tiene 9 jugadores, ¿Cuántos alineamientos son posible?\\
  Usando el mismo approach que en los ejemplos pasados se tiene que hay $9 \cdot 8 \cdot \cdots 2 \cdot 1 = 9!$ alineamientos.
}

\ex{Permutaciones}{
  En una clase de probabilidad hay 6 mujeres y 4 hombres. Se les asigma un examen y son rakeados sugún su desempeño en el examen. Supongamos que hay 2 personas con la misma calificación. 
  \begin{itemize}
    \item ¿Cuántos rakings son posible?
      Dado que hay $6 + 4 = 10$ personas en total, se tienen $10!$ rakings.
    \item ¿Cuántos rankings son posibles si los hombres son rankeados entre ellos y lo mismo sucede para las mujeres?
      Para los hombres hay $6!$ rakings, y para las mujeres $4!$. Por tanto, hay $6! 4!$ rakings totales.
  \end{itemize}
}

\ex{Permutaciones}{
  De entre los 24 miembros de un club se sacan 4 nombres para los puestos de presidente, vicepresidente, tesorero, y secretario. ¿De cuántas maneras diferentes se puede hacer esto?\\
  Se tienen $24 \cdot 23 \cdot 22 \cdot 21 = \frac{24!}{20!}$
}

\nt {
  Dado el ejemplo pasado. Se observa que \textit{no siempre} se "necesita" todo el factorial. Es decir, dependiendo del contexto, necesitaremos o no eliminar datos.
}

\subsection{Permutaciones Circulares}

\[(n - 1)!\] es el número de Permutaciones circulares de \textit{n} elementos.

\subsection{Permutaciones Indistinguibles}
En este caso, deseamos contar el número de ordenamientos cuando existen elementos repetidos indistinguibles.\\
Tenemos \textit{n} objetos en total. Supongamos que de los \textit{n} objetos, $n_1,n_2, \cdots, n_r$ son indistinguibles con $n_1 + n_2 + \cdots + n_r \le n$. Entonces las permutaciones son
\[\frac{n!}{n_1!n_2!\cdots n_r!}\]

\ex{Permutaciones Indistinguibles}{
  Sea \textit{PEPPER} una palabra. ¿De cuántas formas se puede ordenar las letras de la palabra?\\
  Dada la observación anterior. Se tiene que 
  \[\frac{6!}{3! \cdot 2! \cdot 1!} = \frac{6 \cdot 5 \cdot 4}{2} = 60\] es el número de ordenamientos. Donde $3!$ representa las 3 \textit{p's}, $2!$ las \textit{E's}, y $1!$ la \textit{R}.
}

\section{Combinaciones}

\subsection{Coeficiente Binomial}

Supongamos que estamos interesados en contar cuantos subconjuntos de \textit{r} elementos podemos tomar de \textit{n} objetos totales ($r \le n)$. Este número es el \textbf{coeficiente binomial}.
\[\binom{n}{r} = \frac{n!}{r!(n - r)!}\]

\ex{Coeficiente Binomial}{
  ¿Cuántos subconjuntos de 3 letras se pueden formar de las letras \textit{A,B,C,D,E}?\\
  Podemos observar las siguientes etapas del experimento.
  \begin{enumerate}
    \item Tenemos cinco posibilidades (todas las letras disponibles).
    \item Tenemos cuatro posibilidades.
    \item Tenemos tres posibilidades.
  \end{enumerate}
  Dado que dentro de estas tres etapas existen repeticiones, tenemos lo siguiente.
  \[\frac{5 \cdot 4 \cdot 3}{3 \cdot 2 \cdot 1} \frac{2 \cdot 1}{2 \cdot 1} = \frac{5!}{3! \cdot 2!} = \binom53\]
}

\ex{Coeficiente Binomial}{
  Si tenemos 5 mujeres y 7 hombres. ¿Cuántos comités de 2 mujeres y 3 hombres se pueden formar?\\
  Por un lado tenemos $\binom52$ combinaciones de 2 mujeres. Por otro, $\binom73$ de hombres.\\ Por tanto tenemos \[\binom52 \cdot \binom73\] comités totales.
}

\ex{Coeficiente Binomial}{
  ¿Cuántos subconjuntos tiene un conjunto de 5 elementos?\\
  Dado que buscamos \textbf{cada} posible subconjunto, tenemos
  \begin{align*}
    \sum_{i = 0}^5 \binom5i\ &= \binom50 + \binom51 + \binom52 + \binom53 + \binom54 + \binom55\\
    &= \frac{5!}{0! 5!} + \frac{5!}{1!4!} + \frac{5!}{2!3!} + \frac{5!}{3!2!} + \frac{5!}{4! 1!} + \frac{5!}{5! 0!}\\
    &= 1 + 5 + 10 + 10 + 5 + 1 = 32 \\
  \end{align*}
}

\subsection {Coeficiente Multinomial}
Un conjunto de \textit{n} distintos objetos va a ser dividido en \textit{r} distintos subgrupos de tamaño $n_1, n_2, \cdots, n_r$ donde $n_1 + n_2 + \cdots + n_r = n$. ¿De cuántas formas podemos hacer esto?
\[\binom{n}{n_1,n_2,\cdots,n_r} = \frac{n!}{n_1!n_2!\cdots n_r!}\]

\ex{Coeficiente Multinomial}{
  Tenemos el conjunto de letras \textit{A,B,C,D,E} y queremos dos grupos con 3 y 2 elementos cada uno.\\
  Entonces, tenemos lo siguiente para el primer grupo.
  \[\frac{5 \cdot 4 \cdot 3}{3!}\]
  Para el segundo grupo se tiene lo siguiente.
  \[\frac{2 \cdot 1}{2!}\]
  Por tanto, el total es 
  \[\frac{5!}{3! \cdot 2!}\]
}

\thm{}{
  \[(x_1 + x_2 + \cdots + x_r)^r = \sum_{n_1,\cdots, n_r} \binom{n}{n_1\cdots,n_r} x_1^{n_1}x_2^{n_2}\cdots x_r^{n_r} \]
  Es decir, sumamos sobre todos los vectores cuyas entradas son enteras no negativas tales que $n_1 + \cdots + n_r = n$
}

\section{Ejercicios}

\qs{Conteo}{
  10 personas se van a dividir en equipo \textit{A} y \textit{B}, cada una con 5 integrantes. El equipo \textit{A} jugará en una liga y el \textit{B} en otra. ¿Cuántos diferentes equipos son posibles?
}

\qs{Conteo}{
  Para jugar basketball, 10 niñas se dividen en 2 grupas de 5 integrantes. ¿Cuántas divisiones son posibles?
}

\qs{Conteo}{
  Consideremos un conjunto de 11 antenas de las cuales 3 están defectuosas. Defectuosas y funcionales son consideradas Indistinguibles. ¿Cuántos ordenamientos en linea obtenemos si no hay 2 defectuosas consecutivas?
}

\qs{Conteo}{
  \textit{n} pelotas diferentes van a ser distribuidas en \textit{r} urnas. ¿De cuántas formas se pueden distribuir?
}

\qs{Conteo}{
  Supongamos que tenemos 8 pelotas iguales que van a ser distribuidas en 3 urnas. Si todas las urnas deben tener al menos una pelota ¿De cuántas formas s pueden distribuir?
}

\qs{Conteo}{
  ¿Cuántos diferentes arreglos se pueden formar de las letras de las siguientes palabras?
  \begin{itemize}
    \item FLUKE
    \item PROPOSE
    \item MISSISSIPPI
    \item ARRANGE
  \end{itemize}
}

\qs{Conteo}{
  Una niña tiene 12 bloques de lego de los cuales 6 son negros, 4 rojos, 1 blanco, y 1 azul. Si la niña pone los bloques en línea, ¿cuántos arreglos son posibles?
}

\qs{Conteo}{
  De cuántas maneras se pueden sentar 8 personas una junto a la otra.
  \begin{itemize}
    \item No hay restricciones.
    \item Las personas \textit{A} y \textit{B} deben estar sentadas una junto a la otra.
    \item Hay 4 hombres y 4 mujeres, y 2 hombres o 2 mujeres no pueden sentarse uno al lado del otro.
    \item Hay 5 hombres y ellos deben sentarse uno al lado de otro.
    \item Hay 4 parejas de casados y cada pareja debe sentarse junta.
  \end{itemize}
}
