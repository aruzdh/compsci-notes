\chapter{Vectores aleatorios y función de probabilidad conjunta}
\section{Definición}

Se sabe que una \textit{variable aleatoria es una función} que tiene \textit{dominio en el espacio muestral} $\Omega$ y \textit{toma valores en} $\mathbb R$. Sin embargo, este concepto se puede expandir a $\mathbb R^n$, y por tanto, definir una función $f: \Omega \longrightarrow \mathbb R^n$. Visto de otra manera:
\[w \mapsto (\mathcal X_1(w), \mathcal X_2(w), \cdots, \mathcal X_n(w)) = \mathcal X(w)\]
A esta nueva función se le conoce como \textit{vector aleatorio}.

\dfn{Vector aleatorio}{
  Decimos que $\mathcal X$ es un \textbf{vector aleatorio} si
  \[\mathcal X = (\mathcal X_1(w), \mathcal X_2(w), \cdots, \mathcal X_n(w))\]
  donde cada $\mathcal X_i$ es una variable aleatoria.
}

\ex{Vector aleatorio}{
  Se lanza un dado 2 veces. Definimos las siguientes variables aleatorias.
  \[\mathcal X_1 := \text{Valor máximo de los lanzamientos}\]
  \[\mathcal X_2 := \text{Valor mínimo de los lanzamientos}\]
  \[\mathcal X_3 := \text{La suma de los valores en los lanzamientos}\]

  Con lo anterior, nuestro espacio muestral es el siguiente.
\[
  \Omega = \left\{
  \begin{matrix}
  (1,1) & (1,2) & (1,3) & (1,4) & (1,5) & (1,6) \\
  (2,1) & (2,2) & (2,3) & (2,4) & (2,5) & (2,6) \\
  (3,1) & (3,2) & (3,3) & (3,4) & (3,5) & (3,6) \\
  (4,1) & (4,2) & (4,3) & (4,4) & (4,5) & (4,6) \\
  (5,1) & (5,2) & (5,3) & (5,4) & (5,5) & (5,6) \\
  (6,1) & (6,2) & (6,3) & (6,4) & (6,5) & (6,6)
  \end{matrix}
  \right\}
\]
Supongamos que en los dados obtuvimos los valores de 4 y 3. En este caso nuestro vector $\mathcal X$ se ve como:
\[\mathcal X((4,3)) = (\mathcal X_1((3,4)), \, \mathcal X_2((3,4)), \, \mathcal X_3((3,4))) = (4,3,7)\]
Notese que se obtuvo un vector en $\mathbb R^3$.
}

Al igual que las variables aleatorias, los vectores aleatorios se dividen en \textit{discretos} y \textit{continuos} de acuerdo con las variables que los definen. 

\subsection{Caso discreto}
\dfn{Vector aleatorio discreto}{
  Decimos que un \textit{vector aleatorio} $\mathcal X = (\mathcal X_1, \mathcal X_2, \cdots, \mathcal X_n)$ es \textbf{discreto} si $\forall i = 1, \cdots, n, \, \mathcal X_i$ es una variable aleatoria discreta.
}

\subsection{Caso continuo}
\dfn{Vector aleatorio continuo}{
  Decimos que un \textit{vector aleatorio} $\mathcal X = (\mathcal X_1, \mathcal X_2, \cdots, \mathcal X_n)$ es \textbf{continuo} si $\forall i = 1, \cdots, n, \, \mathcal X_i$ es una variable aleatoria continua.
}

\section{Función de probabilidad conjunta}
De la misma manera, podemos calcular la probabilidad de que el vector aleatorio tome ciertos valores con la siguiente Definición.

\dfn{Función de probabilidad conjunta}{
  La \textbf{función de probabilidad conjunta} de dos variables aleatorias $\mathcal X$ y $\mathcal Y$, denotada $f_{\mathcal X, \mathcal Y}(x,y)$, es la función que asigna a cada par de valores $(x,y)$ la probabilidad de que las variables aleatorias $\mathcal X$ y $\mathcal Y$ tomen esos valores simultáneamente.

  \[
    f_{\mathcal X, \mathcal Y}(x,y) = P(\mathcal X = x, \, \mathcal Y = y)
  \]
  donde $(x,y)$ está en la imagen de $(\mathcal X, \, \mathcal Y)$. En caso contrario, la función vale 0.
}

\nt{
  La definición anterior está dada para 2 variables aleatorias, pero se puede extender para \textit{n} variables con $P(\mathcal X_1 = x_1, \, \mathcal X_2 = x_2, \, \cdots, \, \mathcal X_n = x_n)$
}

\ex{Función de probabilidad conjunta}{
  Se lanza una moneda justa tres veces consecutivas y deseamos analizar el comportamiento de las siguientes variables aleatorias.

  \[\mathcal Y = \text{Número de soles obtenidos en las primeros dos lanzamientos}\]
  \[\mathcal X = \text{Número de soles obtenidos en las últimos dos lanzamientos}\]

  El espacio muestral se puede ver como
  \[
    \Omega = \left\{
      \begin{matrix}
        AAA & ASA \\
        AAS & ASS \\
        SAA & SAS \\
        SSA & SSS \\
      \end{matrix}
    \right\}
  \]
  Con esto tenemos que nuestras variables pueden tomar los valores $\{0,1,2\}$, y entonces podemos obtener las siguientes probabilidades.

  \begin{align*}
    P(\mathcal X = 0, \, \mathcal Y = 0) = \frac18 \\
    P(\mathcal X = 0, \, \mathcal Y = 1) = \frac18 \\ 
    P(\mathcal X = 0, \, \mathcal Y = 2) = 0 \\ 
    P(\mathcal X = 1, \, \mathcal Y = 0) = \frac18 \\ 
    P(\mathcal X = 1, \, \mathcal Y = 1) = \frac14 \\ 
    P(\mathcal X = 1, \, \mathcal Y = 2) = \frac18 \\ 
    P(\mathcal X = 2, \, \mathcal Y = 0) = 0 \\ 
    P(\mathcal X = 2, \, \mathcal Y = 1) = \frac18 \\ 
    P(\mathcal X = 2, \, \mathcal Y = 2) = \frac18 \\ 
  \end{align*}
}
