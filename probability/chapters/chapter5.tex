\chapter{Distribuciones}
\section{Distribuciones Aleatorias Discretas}
\subsection{Destribución Bernoulli}

\dfn{}{
  Sea $\mathcal X$ una \textit{variable aleatoria discreta}. Diremos que $\mathcal X$ tiene \textbf{distribución Bernoulli} ($\mathcal X \sim Ber(p)$) con parámetro \textit p, donde $p \in [0,1]$ si satisface lo siguiente
  \[
    \mathcal X = 
    \begin{cases}
    0 ~~~~~ \text{con probabilidad } (1 - p) & \\
    1 ~~~~~ \text{con probabilidad } p & \\
    \end{cases}
  \]
}

\noindent Generalmente decimos que si $\mathcal X(w) = 1$, entonces "\textit{ocurrió un éxito}", y si $\mathcal X(w) = 0$, "\textit{ocurrió un fracaso"}.
\nt{
  A \textit p se le conoce como la \textbf{probabilidad de éxito}
}

\dfn{}{
  Si $\mathcal X \sim Ber(p)$. Definimos su \textbf{función de probabilidad} como
  \[f(k) = P(\mathcal X = k) = 
 \begin{cases}
   P ~~~~~~~~\text{si } k = 1 \\
   1-P ~~~\text{si } k = 0\\
 \end{cases}\]
}

\dfn{}{
  Si $\mathcal X \sim Ber(p)$. Definimos su \textbf{función de distribución} como
  \[
  F_{\mathcal X}(k) = P(\chi \le k)= 
  \begin{cases}
  0 ~~~~~~~~~~~~~~~ \text{si } k < 0 & \\
  1 - p ~~~~~~~~~~ \text{si } 0 \le k < 1 & \\
  1 ~~~~~~~~~~~~~~~ \text{si } 1 \le k & \\
  \end{cases}
\]
}

\subsection{Variable Aleatoria Binomial}
\dfn{}{
  Sea $\mathcal X$ una \textit{variable aleatoria} tal que cuenta el número de éxitos en \textit {n experimentos Bernoulli independientes}. Entonces $\mathcal X$ tiene \textbf{distribución Binomial} con parámetros \textit{p, n}. Se denota como
  \[\mathcal X \sim Bin(n,p)\]
  donde \textit n es el \textit{número de experimentos Bernoulli}, y \textit p es la \textit{probabilidad de éxito de cada experimento}.
}

\dfn{}{
  Si $\mathcal X \sim Bin(n,p)$. Definimos la \textbf{función de probabilidad de masa binomial} como
  \[f(k) = P(\mathcal X = k) = \binom{n}{k}p^n(1-p)^{n - k}\]
}

\dfn{}{
  Si $\mathcal X \sim Bin(n,p)$. Definimos su \textbf{función de distribución} como
  \[F_{\mathcal X}(k) = P(\mathcal X \le k) = \sum_{i = 0}^k \binom{n}{i}p^i(1-p)^{n - i}\]
}

\subsection{Variable Aleatoria Poisson}

\dfn{}{
  Sea $Y \sim Bin(n, \frac{\lambda}{n})$ con $\lambda > 0$, es decir, \textit {Y es una variable aleatorial binomial} donde la probabilidad de éxito es $\frac{\lambda}{n}$.
  \[ \lim_{n \to \infty}P(Y = k) = \frac{\lambda^k}{k!}e^{-\lambda} = P(Z = k) = f(k)\]
  donde \textit f es la \textbf{función de probabilidad de masa} de una variable aleatoria $Z \sim Poi(\lambda)$
}

\nt{
  Si $Y \sim Bin(n,p)$ y \textit {p es pequeño} y \textit {n suficientemente grande}. Entonces se puede calcular la distribución de \textit Y unsando la \textbf{distribución de Poisson}.
}

\dfn{}{
  Una \textit{variable aleatoria con distribución poisson} $\mathcal X \sim Poi(\lambda)$ con $\lambda > 0$ tiene función de probabilidad de masa
  \[P(\mathcal X = k) = f(k) = \frac{e^{-\lambda}\lambda^k}{k!},~ k = 0, 1, \cdots\]
}

\subsection{Variable Aleatoria Geométrica}
Supongamos que van a desarrollarse experimentos Bernoulli independientes con parámetro de éxito \textit p.

\dfn{}{
  La \textbf{variable aleatoria geométrica} se define como
  \[\mathcal X = \text{"\# de fracasos antes de obtener un el primer éxito"}\]
}

Entonces
\begin{align*}
  &f(0) = P(\mathcal X = 0) = P(\text{"Éxito"}) = p \\
  &f(1) = P(\mathcal X = 1) = P(\text{F~E}) = (1 - p) p \\
  &f(2) = P(\mathcal X = 2) = P(\text{F~F~E}) = (1 - p)^2 p \\
  &\cdots \\
  &f(k) = P(\mathcal X = k) = (1-p)^k p \\
\end{align*}

\dfn{}{
  Si $\mathcal X$ en una \textit{variable aleatoria geométrica}. Definimos la \textbf{función de probabilidad} como

  \[f(k) = P(\mathcal X = k) = (1-p)^k p\]
}

\noindent \textit f satisface las propiedades de una función de probabilidad
\begin{enumerate}
  \item \[f(k) \le 0\]
  \item \[\sum_{k = 0}^{\infty} f(k) = 1\]
\end{enumerate}

\dfn{}{
  Si $\mathcal X$ en una \textit{variable aleatoria geométrica}. Definimos su \textbf{función de distribución} como

  \[F_{\mathcal X}(k) = P(\mathcal X \le k) = \sum_{i = 0}^k(1-p)^i p\]
}


\subsection{Variable Aleatoria Binomial Negativa}

Supongamos que van a desarrollarse experimentos Bernoulli independientes con parámetro de éxito \textit p.
\dfn{}{
  La \textbf{variable aleatoria binomial negativa} se define como
  \[\mathcal X = \text{\# de ensayos hasta obtener r éxitos}\]
}

\dfn{}{
  Si $\mathcal X$ en una \textit{variable aleatoria binomial negativa}. Definimos la \textbf{función de probabilidad} como

  \[f(k) = P(\mathcal X = k) = \binom{k-1}{r-1}p^r(1-p)^{k-r} = \frac{(k-1)!}{(k-r)!(r-1)!}p^r(1-p)^{k-r} \]
  donde \textit r es el número de éxitos, y \textit k el número de ensayos.
}

\dfn{}{
  Si $\mathcal X$ en una \textit{variable aleatoria binomial negativa}. Definimos su \textbf{función de distribución} como

  \[F_{\mathcal X}(k) = P(\mathcal X \le k) =\sum_{i = 0}^k \binom{i-1}{r-1}p^r(1-p)^{i-r} = \sum_{i = 0}^k \frac{(i-1)!}{(i-r)!(r-1)!}p^r(1-p)^{i-r} \]
  donde \textit r es el número de éxitos, y \textit k el número de ensayos.
}

\subsection{Variable Aleatoria Uniforme}
\dfn{}{
  Una \textit{variable aleatoria $\mathcal X$} se dice que es \textbf{uniforme} sobre $\{a_1, a_2, \cdots, a_n\}$ donde $a_i \in \mathbb R$ si
  \[P(\mathcal X = a_i) = \frac{1}{n},~ \forall i \in \{1, 2, \cdots n\}\]
}

\subsection{Variable Aleatoria Hipergeométrica}

Supongamos que en una caja hay \textit N pelotas, \textit m azules y \textit{N - m} amarillas. Se escogen \textit n pelotas al azar.

\[P(\mathcal X = k) = \frac{\binom{m}{k} \binom{N - m}{n-k} }{\binom{N}{n}}\]

\noindent Todos los experimentos que tengan un contexto similar se modelan con una variable aleatoria hipergeométrica.

\section{Distribuciones Absolutamente Continuas}
Dada una variable aleatoria \textit{continua}, su correspondiente función de distribución es \textbf{continua}.\\
Se tienen los siguientes casos:
\begin{itemize}
  \item Singulares: La función de densidad \textbf{no existe}.
  \item Absolutamente continuas: \[F'(x) = f(x)\] \[P(a < \mathcal X < b) = \int_a^b f(x)~dx\]
\end{itemize}

\subsection{Variable Aleatoria Uniforme Continua}
\dfn{}{
  Una \textit{variable aleatoria $\mathcal X$} tiene \textbf{distribución Uniforme Continua} y se denota $\mathcal X \sim unif(a,b)$ si su \textbf{función de densidad} es la siguiente
  \[f(x) = 
    \begin{cases} 
      \frac{1}{b-a} ~~~\text{si } x \in (a,b)\\
      0 ~~~~~\text{ otro caso}
    \end{cases} 
    = f(x) = \frac{1}{b-a}\mathbb 1_{(a,b)}^{(x)}
    \]
}

\nt{
  Se observa que $f(x) \ge 0$ y $\int_{-\infty}^{\infty}f(x) ~dx = 1$
}

\dfn{}{
  Dada $\mathcal X \sim unif(a,b)$. Su \textbf{función de distribución} está dada por
  \[
    F_{\mathcal X}(c) =
    \begin{cases}
      0~~~~~~~ \text{si } c \le a\\
      \frac{c-a}{b-a}~~~~ \text{si } c \in (a,b)\\
      1~~~~~~~ \text{si } c \ge b
    \end{cases}
  \]
}

\subsection{Variable Aleatoria Simétrica}

\dfn{Función de densidad simétrica}{
  Una función de densidad \textit{f} es \textbf{simétrica} si \textit{f} es una función \textit{par}, es decir, \[f(x) = f(-x),~ \forall x \in \mathbb R\]
}

\dfn{}{
  Una \textit{variable aleatoria $\mathcal X$} es \textbf{simétrica} si $\mathcal X$ y $-\mathcal X$ tienen la misma función de distribución, es decir,
  \[F_{\mathcal X}(a) = F_{-\mathcal X} (a),~ \forall a \in \mathbb R\]
  Lo anterior se denota como ($\mathcal X \stackrel{\text{d}}{\sim} \mathcal {-X}$)
}

\thm{}{
  Una \textit{variable aleatoria $\mathcal X$} \textbf{simétrica si y solo si su función de densidad es simétrica} (su función de densidad es par).
}

\subsection{Variable Aleatoria Normal o Gaussiana}

\dfn{}{
  Decimos que una \textit{variable aleatoria $\mathcal X$} tiene \textbf{distribución Normal Estándar} si su función de densidad es \[\phi(x) = \frac{e^{-\frac{x^2}{2}}}{\sqrt{2\pi}},~ x \in \mathbb R\]
  Observemos que $\mathcal X$ es una \textit{variable aleatoria simétrica} porque $\phi(x) = \phi(-x)$
}

\nt{
  Se tiene que
  \begin{itemize}
    \item \[\phi(x) \ge 0\]
    \item \[\int_{-\infty}^{\infty}\phi(x)~dx = 1\]
  \end{itemize}
}

\mlemma{}{
  Si $\mathcal X$ es una \textit{variable aleatoria simétrica}, entonces \[F_{\mathcal X}(0) = \frac{1}{2}\]
}

\mlemma{}{
  Si $\mathcal X$ es una \textit{variable aleatoria simétrica}, entonces \[F_{\mathcal X}(-a) = 1 - F_{\mathcal X}(a)\]
}

\nt{
  Recordemos que la función de densidad Normal
  \[\phi(x) = \frac{e^{-\frac{x^2}{2}}}{\sqrt{2\pi}}\]
  no tiene antiderivada.
}

\thm{Teorema de Cambio de Variable}{
  Sea $\mathcal X$ una \textit{variable aleatoria absolutamente continua} con valores en $(a,b) \subseteq \mathbb R$. Sea $g : (a,b) \longrightarrow \mathbb R$ una función continua estrictamente creciente o decreciente con inversa diferenciable. Entocnes la variable aleatoria $\mathcal Y= g(\mathcal X) = g \circ \mathcal X : \Omega \longrightarrow \mathcal R$ tinee la función de densidad

  \[f_{\mathcal Y}(y) = 
    \begin{cases}
      f_{\mathcal X} (g^{-1}(y)) \mid \frac{d}{dy} g^{-1}(y)\mid,~~~~ y \in g(a,b)\\
      0, ~~~~~~~~~~~~~~~~~~~~~~~~~~~~~~~~~ \text{otro caso}
    \end{cases}
  \]
}

\ex{}{
  Sea $\mathcal X$ una \textit{variable aleatoria} Normal Estándar, y definimos $\mathcal Y = y + \sigma \mathcal X$ con $y \in \mathbb R, ~ \sigma \neq 0,~ \sigma > 0$.
  Calculemos la función de densidad de $\mathcal Y$

  Se tiene $\mathcal Y = g(\mathcal X)$ donde $\mathcal X$ es Normal Estándar
  \[f_{\mathcal X}(x) = \frac{e^{-\frac{x^2}{2}}}{\sqrt{2\pi}}\]

  Notemos que
  \[
    g(s) = y + \sigma s \Longrightarrow g^{-1}(s) = \frac{s-y}{\sigma}
  \]

  Aplicando el teorema de cambio de variable.
  \begin{align*}
    f_{\mathcal Y}(x) &= f_{\mathcal X}(g^{-1}(x)) \mid \frac{d}{dx} g^{-1}(x) \mid\\
             &= \frac{e^{-\frac{g^{-1}(x)^2}{2}}}{\sqrt{2\pi}}\mid \frac{d}{dx} g^{-1}(x) \mid\\
             &= \frac{e^{-\frac{1}{2} (\frac{s - y}{\sigma})^2}}{\sqrt{2\pi}}\mid \frac{d}{dx} (\frac{x-y}{\sigma}) \mid\\
             &= \frac{e^{-\frac{1}{2}(\frac{s-y}{\sigma})^2}}{\sqrt{2\pi}\sigma}
  \end{align*}

  Entonces, la función de densidad de $\mathcal Y = y + \sigma \mathcal X, ~ y \in \mathbb R,~ \sigma > 0$ es 
  \[f_{\mathcal Y}(x) = \frac{e^{-\frac{1}{2}(\frac{x-y}{\sigma})^2}}{\sqrt{2\pi}\sigma} \]

  Por tanto, $\mathcal Y$ es una variable aleatoria con distribución Normal
}

\dfn{}{
  Si $\mathcal X \sim N(0,1)$, entonces $\mathcal X$ es \textbf{variable aleatoria normal estándar}.

  Si $\mathcal X \sim N(y,\sigma)$, entonces $\mathcal X$ es \textbf{variable aleatoria normal con parámetros} $y \in \mathbb R$ y $\sigma \neq 0$.
}
\nt{
  Observemos que si $\mathcal X \sim N(0,1)$ entonces $y \equiv 0,~ \sigma \equiv 0$
}

Usualmente se trabaja con una \textit{Tabla De La Distribución Normal}

\ex{}{
  Sea $\mathcal X \sim N(-1,3)$, donde $y = -1, ~ \sigma = 3$
  Entonces $\mathcal X = -1 + 3 \mathcal Z$, con $\mathcal Z\sim N(0,1)$

  Se tiene
  \begin{align*}
    P(\mathcal X \le -1.06) &= P(-1 + 3 \mathcal Z \le -1.06)\\
                 &= P(3 \mathcal Z \le -1.06 + 1)\\
                 &= P(\mathcal Z \le \frac{-1.06+1}{3})\\
                 &= P(\mathcal Z \le 0.02)\\
                 &= 0.5080\\
  \end{align*}
}


