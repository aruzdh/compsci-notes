\section{Ejercicios}

\qs{Conteo}{
  10 personas se van a dividir en equipo \textit{A} y \textit{B}, cada una con 5 integrantes. El equipo \textit{A} jugará en una liga y el \textit{B} en otra. ¿Cuántos diferentes equipos son posibles?
}

\qs{Conteo}{
  Para jugar basketball, 10 niñas se dividen en 2 grupas de 5 integrantes. ¿Cuántas divisiones son posibles?
}

\qs{Conteo}{
  Consideremos un conjunto de 11 antenas de las cuales 3 están defectuosas. Defectuosas y funcionales son consideradas Indistinguibles. ¿Cuántos ordenamientos en linea obtenemos si no hay 2 defectuosas consecutivas?
}

\qs{Conteo}{
  \textit{n} pelotas diferentes van a ser distribuidas en \textit{r} urnas. ¿De cuántas formas se pueden distribuir?
}

\qs{Conteo}{
  Supongamos que tenemos 8 pelotas iguales que van a ser distribuidas en 3 urnas. Si todas las urnas deben tener al menos una pelota ¿De cuántas formas se pueden distribuir?
}

\qs{Conteo}{
  ¿Cuántos diferentes arreglos se pueden formar de las letras de las siguientes palabras?
  \begin{itemize}
    \item FLUKE
    \item PROPOSE
    \item MISSISSIPPI
    \item ARRANGE
  \end{itemize}
}

\qs{Conteo}{
  Una niña tiene 12 bloques de lego de los cuales 6 son negros, 4 rojos, 1 blanco, y 1 azul. Si la niña pone los bloques en línea, ¿cuántos arreglos son posibles?
}

\qs{Conteo}{
  De cuántas maneras se pueden sentar 8 personas una junto a la otra en cada escenario.
  \begin{itemize}
    \item No hay restricciones.
    \item Las personas \textit{A} y \textit{B} deben estar sentadas una junto a la otra.
    \item Hay 4 hombres y 4 mujeres, y 2 hombres o 2 mujeres no pueden sentarse uno al lado del otro.
    \item Hay 5 hombres y ellos deben sentarse uno al lado de otro.
    \item Hay 4 parejas de casados y cada pareja debe sentarse junta.
  \end{itemize}
}
