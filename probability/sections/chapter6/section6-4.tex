\newpage
\section{Teorema de Chebyshev}
\thm{Teorema de Chebyshev}{
  Sea $\mathcal X$ una variable aleatoria con esperanza $\mu$ y desviación estándar $\sigma$, entonces para cualquier constante $k > 0$ la probabilidad es al menos $1 - \frac1{k^2}$ de que $\mathcal X$ asumirá un valor dentro de $k$ desviaciones estándar de la media (valor esperado), es decir,
  \[P(|\mathcal X - \mu| \le k \sigma) = P(-k \sigma \le \mathcal X - \mu \le k \sigma) = P(\mu -k \sigma \le \mathcal X \le \mu + k \sigma) \ge 1 - \frac1{k^2}\]
}

