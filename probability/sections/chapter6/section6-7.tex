\section{Ejercicios}

\qs{Momentos de Variables Aleatorias}{
  Calcula la Función Generadora de Momentos y la Función Generadora de Probabilidad de $\mcX$ si:
  \begin{itemize}
    \item $\mcX \sim Bin(n,p)$
    \item $\mcX \sim Geo(p)$
  \end{itemize}
}

\qs{Momentos de Variables Aleatorias}{
  Sea $\mcX$ una variable aleatoria con función de densidad $f_{\mcX}(x) = \frac{e^{- \abs*{x}}}2$ para $x \in \bbR$. Muestra que
  \[M_{\mcX}(s) = \frac{1}{1-s^2}\]
  para $s \in (-1, 1)$
}

\qs{Momentos de Variables Aleatorias}{
  Los ingresos de una compañia por semana se representan con una variable aleatoria $\mcX$ tal que
  \[M_{\mcX}(t) = (1-2500t)^{-4}\]
  Encuentra la varianza de dichos ingresos semanales.
}

\qs{Momentos de Variables Aleatorias}{
  Sea $\mcZ \sim N(0,1)$
  \begin{itemize}
    \item Calcula $M_{\mcZ}(s)$
    \item Usa la expansión en series de Taylor de $M_{\mcZ}$ para demostrar que
      \[
        \E({\mcZ}^n) = 
        \begin{cases}
          0 & \text{si $n$ es impar}\\
          \frac{n!}{2^{\sfrac{n}{2}}(n/2)!} & \text{si $n$ es par}
        \end{cases}
      \]
    \item Sea $\mcX = \sigma \mcZ + \mu,\, \sigma > 0$. Calcula $M_{\mcX}$
    \item Usa la función $M_{\mcX}$ para calcular $\Var(\mcX)$
  \end{itemize}
}

\qs{Momentos de Variables Aleatorias}{
  Sean $\mcX_1, \mcX_2,\cdots, \mcX_n$ variables aleatorias independientes e identicamente distribuidas tales que $M_{\mcX_i}(r)$ es finita para toda $r$. Calcula $\E(S^3_n)$
}

