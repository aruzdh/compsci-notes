\section{Función Generadora de Probabilidad}
\subsection{Definición}

\dfn{Función generadora de probabilidad}{
  Sea $\mathcal X$ una variable aleatoria que toma valores enteros positivos $\mathbb Z^+$. La \textbf{función generadora de probabilidad} se define de la siguiente manera.
  \[G(t) = \E(t^{\mathcal X})\]
}

\ex{Función generadora de probabilidad}{
  Calculemos la función generadora de probabilidad para $\mathcal X \sim \text{Poi}(\lambda)$

  \begin{align*}
    G(t) &= \E(t^{\mathcal X}) = \sum_{n=0}^{\infty} t^n P(X = n) = \sum_{n=0}^{\infty} t^n \frac{e^{-\lambda}\lambda^n}{n!} = e^{-\lambda} \sum_{n=0}^{\infty} \frac{(t\lambda)^n}{n!} = e^{-\lambda}e^{\lambda t} = e^{-\lambda(1-t)}
  \end{align*}
}
Esta función nos sirve principalmente para conocer los momentos factoriales de una variable aleatoria $\mathcal X$, es decir:
  \[
  \lim_{t \to 1} \frac{d^n}{dt^n}G(t) = \E(\mathcal X(\mathcal X-1)(\mathcal X-2)...(\mathcal X-n+1))
\]
A este último se le conoce como el n-ésimo momento factorial. Notemos que:
\begin{itemize}
    \item El primer momento factorial es la esperanza $(n=1)$ $\E(\mathcal X)$
    \item Segundo momento factorial $(n=2)$ $\E(\mathcal X(\mathcal X-1))=\E(\mathcal X^2)-\E(\mathcal X)$
    \item Tercer momento factorial $(n=3)$ $\E(\mathcal X(\mathcal X-1)(\mathcal X-2))=\E((\mathcal X^2-X)(\mathcal X-2))=\E(\mathcal X^3-3\mathcal X^2+2\mathcal X)$
\end{itemize}

\subsection{Variables Aleatorias Iguales en Distribución}
Algo que también es importante distinguir es cuando las variables aleatorias tienen la misma distribución como en el siguiente ejemplo.\\
Definimos los siguientes experimentos:
\begin{itemize}
    \item El lanzamiento de una moneda justa, $\mathcal X=\begin{cases}
        -1 & \text{si sale sol}\\
        1 & \text{si sale águila}
    \end{cases}$

    \item Al tomar una carta en una baraja de 52 cartas, $\mathcal Y=\begin{cases}
        100 & \text{si sale pica ó corazón}\\
        0 & \text{si sale diamante ó trébol}
    \end{cases}$

    \item Al lanzar un dado justo, $\mathcal Z=\begin{cases}
        1 & \text{si sale un número impar}\\
        0 & \text{si sale un número par}
    \end{cases}$
\end{itemize}
En este caso observamos que $\mathcal X \neq \mathcal Y \neq \mathcal Z$, sin embargo, $\mathcal X \sim \text{Ber}(1/2)$, $\mathcal Y \sim \text{Ber}(1/2)$, $\mathcal Z \sim \text{Ber}(1/2)$. Lo anterior significa que las variables son iguales en distribución y lo denotamos por $\mathcal X \stackrel{d}{\sim} \mathcal Y \stackrel{d}{\sim} \mathcal Z$, es decir $F_{\mathcal X}(y)=F_{\mathcal Y}(y)=F_{\mathcal Z}(y) \quad \forall y \in \mathbb{R}$.

\dfn{Variables Aleatorias Iguales en Distribución}{
  Sean $\mathcal X$ y $\mathcal Y$ dos variables aleatorias. Se dice que son \textbf{iguales en distribución} si
  \[P(\mathcal X \in B) = P(\mathcal Y \in B),\, \forall B \subset \bbR\]
  Esto último también se denota como $\mathcal X \stackrel{d}{=} \mathcal Y$.
}

\ex{Igualdad en distribución}{
  Uno de los ejemplos más comunes de esta propiedad es que las variables aleatorias $\mathcal X \sim \Gamma(1, \lambda)$ y $\mathcal Y \sim \text{Exp}(\lambda)$ son iguales en distribución ($\mathcal X \stackrel{d}{=} \mathcal Y$)
}

\subsection{Propiedad de Igualdad en Distribución}
\mprop{Igualdad en Distribución}{
  Sean $\mathcal X$ y $\mathcal Y$ dos variables aleatorias con función generadora de momentos $M_{\mathcal X}(t)$ y $M_{\mathcal Y} (t)$, respectivamente, y además se cumple que $M_{\mathcal X}(t) = M_{\mathcal Y}(t)$ para todo $t \in (-s, s), s > 0$. Entonces
  \[\mathcal X \stackrel{d}{=} \mathcal Y\]
}

\nt{
  La proposición anterior también se cumple para la función generadora de probabilidad, es decir, si en lugar de usar $M_{\mathcal X}(t)$ y $M_{\mathcal Y}(t)$ usamos $G_{\mathcal X}(t)$ y $G_{\mathcal Y}(t)$ para todo $t \in (-s, s), s > 0$. Entonces $\mathcal X \stackrel{d}{=} \mathcal Y$.
}

