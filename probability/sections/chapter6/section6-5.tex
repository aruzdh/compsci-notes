\section{Momentos de una Variable Aleatoria}

Los momentos de una variable aleatoria son medidas que describen diferentes aspectos de su comportamiento. Nos ayudan a entender cosas como el promedio, la dispersión, y la forma de la distribución de los valores. En general, los momentos nos ofrecen una forma matemática de capturar información clave sobre cómo se comportan los datos o resultados asociados a esa variable.

\subsection{Definición}
\dfn{Momentos de una Variable Aleatoria}{
  Sea $\mathcal X$ una variable aleatoria, definimos el \textbf{$n$-ésimo momento} de $\mathcal X$ como:
  \[\E(\mathcal X^n), \quad n \in \mathbb N\]
}

\subsection{Momentos Importantes de una Variable Aleatoria}

Algunos de los momentos más importantes son los siguientes.
\begin{itemize}
  \item $\E(\mathcal X)$. La esperanza es el primer momento.
  \item $\E([\mathcal X - \mu]^2)$. El segundo momento central.
  \item $\E([\mathcal X - \mu]^n)$. El \textit{n}-ésimo momento central.
  \item $\E(\abs*{\mathcal X}^n)$. El \textit{n}-ésimo momento absoluto.
  \item $\E(\abs*{\mathcal X - \mu}^n)$. El \textit{n}-ésimo momento absoluto central.
\end{itemize}

\subsection{Función Generadora de Momentos}

\dfn{Función Generadora de Momentos}{
  La \textbf{función generadora de momentos} de una variable aleatoria $\mathcal X$ está definida por
  \[M_{\mathcal X} = \E(e^{t\mathcal X})\]
Lo que implica que $M_{\mathcal X} (t)$ se encuentra bien definida para valores de $t$ tales que $\E(|e^{t \mathcal X}|) < \infty$
}

\ex{Función Generadora de Momentos}{
  Calculemos $M_{\mathcal X}(t)$ si $\mathcal X \sim \Gamma(\alpha, \lambda)$ con $\alpha, \lambda > 0$.
  \[
    f_{\mathcal X}(x) =
    \begin{cases}
      \frac{\lambda^\alpha}{\Gamma(\alpha)}x^{\alpha -1}e^{-\lambda x} & x > 0 \\
      0 &\text{otro caso}
    \end{cases}
  \]
  Entonces
  \begin{align*}
    M_{\mathcal X}(t) &= \E(e^{t \mathcal X}) = \int_{-\infty}^{\infty} e^{tx}\frac{\lambda^\alpha}{\Gamma(\alpha)}x^{\alpha -1}e^{-\lambda x} \mathbb 1_{(0, \infty)}^{(x)}dx = \int_0^{\infty} e^{tx}\frac{\lambda^\alpha}{\Gamma(\alpha)}x^{\alpha -1}e^{-\lambda x} dx\\
                    &= \lambda^\alpha \int_0^{\infty} \frac1{\Gamma(\alpha)}x^{\alpha -1}e^{tx-\lambda x} dx
  \end{align*}
  Multiplicando por $\frac{(\lambda - t)^\alpha}{(\lambda - t)^\alpha}$
  \begin{align*}
    M_{\mathcal X} (t) &= \frac{\lambda^\alpha}{(\lambda - t)^\alpha} \int_0^{\infty} \frac{(\lambda - t)^\alpha}{\Gamma(\alpha)}x^{\alpha -1}e^{-(\lambda - t)x} dx = \frac{\lambda^\alpha}{(\lambda - t)^\alpha} \int_0^{\infty} f_{\mathcal X'}(x)dx = \frac{\lambda^\alpha}{(\lambda - t)^\alpha}
  \end{align*}
  Donde $\mathcal X' \sim \Gamma(\alpha, \lambda - t)$ y además $\lambda - t > 0 \Rightarrow \lambda > t$. Por lo tanto, la función generadora de momentos de $\mathcal X$ es
  \[\therefore \,M_{\mathcal X} (t) = \parens*{\frac{\lambda}{\lambda - t}}^{\alpha}, \quad \lambda > t\]
}

\ex{Función Generadora de Momentos}{
  Calcular $M_{\mathcal X}(t)$ donde $\mathcal X \sim \text{Poi}(\lambda)$
  \newpara
  Primera recordemos que
  \[f_{\mathcal X}(x) = \frac{e^{-\lambda} \lambda^x}{x!}, \quad x = 0, 1, 2, \cdots\]
  Entonces
  \[M_{\mathcal X}(t) = \E(e^{-\lambda \mathcal X}) = \sum_{i = 0}^\infty \frac{e^{ti}e^{-\lambda}\lambda^i}{i!} = e^{-\lambda} \sum_{i = 0}^\infty \frac{e^{ti}\lambda^i}{i!} = e^{-\lambda} \sum_{i = 0}^\infty \frac{(e^t \lambda)^i}{i!} = e^{-\lambda}e^{e^t\lambda}\]
  Esto último se obtuvo usando la serie de la exponencial: 
  \[e^x = \sum_{k = 0}^\infty \frac{x^k}{k!}\]
  \[\therefore \,M_{\mathcal X}(t) = e^{-\lambda}e^{e^t \lambda} = e^{-\lambda(1 - e^t)}\]
}

Consideremos $\mcX$ una variable aleatoria discreta que toma valores en $\{0,1,\cdots,n\}$. Tenemos lo siguiente.
\[M_{\mcX}(t) = E(e^{t\mcX}) = \sum_{i=0}^n e^{ti}P(\mcX = i) \Longrightarrow \frac{d}{dt}M_{\mcX}(t) = \sum_{i=0}^n P(\mcX = i) \frac{d}{dt}e^{ti} = \sum_{i=0}^n P(\mcX = i) ie^{ti} = \E(e^{t\mcX}\mcX)\]
\[M'_{\mcX}(t) = \E(e^{t\mcX}\mcX) \Longrightarrow M'_{\mcX}(0) = \E(e^{0 \cdot\mcX}\mcX) = \E(\mcX)\]

Usando un procedimiento análogo se tiene que
\[M''_{\mcX}(t) = \E(e^{t\mcX}\mcX^2) \Longrightarrow M''_{\mcX}(0) = \E(e^{0 \cdot\mcX}\mcX^2) = \E(\mcX^2)\]

Así, por medio de inducción se puede probar que
\[M^{(n)}_{\mcX}(0) = \E(\mcX^n)\]

\subsection{Propiedades de la Función Generadora de Momentos}

\mprop{}{
  Sea $\mathcal X$ una variable aleatoria con función generadora de momentos $M_{\mathcal X} (t)$. Entonces se cumple lo siguiente.
  \begin{enumerate}
    \item \[\frac{d^n}{dt^n} M_{\mathcal X}(t) \bigg |_{t = 0} = \E(\mathcal X^n)\]
    \item \[\E(\mathcal X^n) < \infty \text{ cuando } t \in (-s, s) \text{ con } s > 0 \text{ suficientemente pequeño}\]
    \item \[M_{\mathcal X}(t) = \sum_{n = 0}^\infty \frac{t^n}{n!} \E(\mathcal X^n) \leftarrow \quad \text{Serie de Taylor} \]
  \end{enumerate}
}

\ex{Propiedades de la Función Generadora de Momentos}{
  Sea $\mcX \sim \Gamma(\alpha, \lambda)$. Sabemos que
  \[M_{\mcX}(t) = \parens*{\frac{\lambda}{\lambda - t}}^{\alpha}\]
  Entonces
  \[M'_{\mcX}(t) = \alpha \parens*{\frac{\lambda}{\lambda - t}}^{\alpha-1} \lambda(\lambda - t)^{-2} \Longrightarrow
  M'_{\mcX}(0) = \alpha \parens*{\frac{\lambda}{\lambda - 0}}^{\alpha-1} \lambda(\lambda - 0)^{-2} = \alpha \lambda^{-1} = \frac{\alpha}{\lambda} = \E(\mcX)\]
}

\subsection{Función Característica}
Aunque la función generadora de momentos no siempre existe, la siguiente existe para todo $t \in \mathbb R$.

\dfn{Función característica}{
  Sea $\mathcal X$ una variable aleatoria, definimos la \textbf{función característica} de $\mathcal X$ como sigue.
  \[\Phi_{\mathcal X} (t)=\E(e^{it \mathcal X}) = \E(\cos (t\mathcal X) + i \sin(t \mathcal X))\]
}

