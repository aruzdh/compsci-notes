\section{Ejercicios}

\qs{Conceptos Básicos}{
  Determina el espacio muestral $\Omega$ de los siguientes experimentos aleatorios.
  \begin{enumerate}[a)]
    \item Lanzar un dado hasta que se obtiene un \textit{5}.
    \item Observar el número de años que le restan de vida a una persona escogida al azar dentro del conjunto de asegurados de una compañia aseguradora.
  \end{enumerate}
}
\qs{Conceptos Básicos}{
  ¿Cuál es el número de elementos del espacio muestral del experimento aleatorio que consiste en lanzar 3 veces una moneda?
}
\qs{Conceptos Básicos}{
  Se extraen aleatoriamente una carta de una baraja de 52. Sea \textit{A = "se extrae un rey"} y \textit{B = "se extrae un trébol"} eventos. Describe los siguientes sucesos o eventos.
  \begin{enumerate}[a)]
    \item $A \cup B$
    \item $A \cap B$
    \item $A \cup B^c$
    \item $A^c \cup B^c$
    \item $A - B$
    \item $A^c - B^c$
  \end{enumerate}
}
