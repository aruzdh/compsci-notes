\section{Eventos}

\dfn{Evento}{
  Se le denomina \textbf{evento} a \textit{cualquier subconjunto} de $\Omega$.
}

\nt{
  Los \textit{eventos} suelen denotarse con letras mayúsculas.
}

\ex{Eventos}{
  Supongamos que se lanza un par de dados de 6 caras. Encuentra los elementos del evento:

  \begin{itemize}
    \item \textit{A = "En ambos dados se obtuvo el mismo número"}
    \item \textit{B = "Se obtuvo al menos un 3"}
    \item \textit{C = "La suma de los números obtenidos es 5"}
  \end{itemize}

  Dado lo anterior, se tienen los siguientes eventos con sus respectivos elementos.
  \begin{itemize}
    \item $A = \{(1,1), (2,2), (3,3), (4,4), (5,5), (6,6)\}$
    \item $B = \{(1,3), (2,3), (3,3), (4,3), (5,3), (6,3)\}$
    \item $C = \{(1, 4), (2, 3), (3, 2), (4, 1)\}$
  \end{itemize}
}

Decimos que un \textit{evento ocurrió} si al realizar el experimento aleatorio, el elemento $w \in \Omega$ ocurre y $w \in A$. Entonces decimos que \textit{A ocurrió}.

\nt{
  $\Omega$ es el evento seguro (dado que todo elemento pertecene a $\Omega$).
}
