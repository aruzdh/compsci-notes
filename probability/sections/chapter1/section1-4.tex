\section{Operaciones de Conjuntos en Probabilidad}

Dados $A, B, C \subseteq \Omega$ eventos.
\begin{itemize}
  \item Unión

    Dado $A \cup B$. Decimos que el evento \textit{A} \textit{ocurrió} \textbf{ó} el evento \textit{B} \textit{ocurrió}.
  \item Intersección

    Dado $A \cap B$. Decimos que el evento \textit{A} \textit{ocurrió} \textbf{y} el evento \textit{B} \textit{ocurrió} (\textbf{ambos eventos ocurrieron}).
  \item Complemento

    Dado $B^c$. Decimos que \textit{B} \textbf{no} \textit{ocurrió}.
  \item Diferencia

    Dado $C - B$. Decimos que \textit{C} \textit{ocurrió} pero \textbf{no} ocurrió \textit{B}.
\end{itemize}

\ex{Operaciones de Conjuntos en Probabilidad}{
  Considerando el ejemplo anterior, se tiene lo siguiente.
  \begin{itemize}
    \item $A - B =$ "En ambos dados se obtuvo el mismo número pero no 3" $= \{(1,1), (2,2), (4,4), (5,5), (6,6)\}$
    \item $B^c = \text{No se obtuvo ningún 3}$
    \item $A \cap C = \text{En ambos dados se obtuvo el mismo número y la suma es 5} = \varnothing$
  \end{itemize}
}
