\section{Función de Probabilidad}
\subsection{Definición}
\dfn{Función de Probabilidad o Función de Probabilidad de Masa}{
  Sea $\mathcal X$ una \textit{variable aleatoria discreta}. Definimos su \textbf{función de probabilidad} como sigue a continuación.
  \[f_{\mathcal X} (a) = P(\mathcal X = a),\] con \textit a en la imagen de $\mathcal X$
}

\nt{
  Notemos que si $\mathcal X$ es una variable aleatoria \textit{continua}, su función de distribución es continua, y por la tanto 
  \[P(\mathcal X = a) = F(a) - F(a^-) = F(a) - F(a) = 0\]
  para cualquier $a \in \mathbb R$. Es decir, no tiene utilidad en el caso continuo.
}

\ex{Variables Aleatorias Discretas}{
  Consideremos el ejemplo anterior donde lanzamos 3 monedas justas. Dada la $\mathcal Y$ definida anteriormente, se tiene lo siguiente.
  \begin{itemize}
    \item $f_{\mathcal Y} (a) = P(\mathcal Y = a)$
    \item $f_{\mathcal Y} (0) = P(\mathcal Y = 0) = \frac18$
    \item $f_{\mathcal Y} (1) = P(\mathcal Y = 1) = \frac38$
    \item $f_{\mathcal Y} (2) = P(\mathcal Y = 2) = \frac38$
    \item $f_{\mathcal Y} (3) = P(\mathcal Y = 3) = \frac18$
  \end{itemize}
}

\nt{
  Notese que esta función \textbf{no es acumulativa}.
}

\subsection{Propiedades}
Las siguientes son propiedades de las funciones de probabilidad.
\begin{enumerate}
  \item \[f_{\mcX}(k) \ge 0\]
  \item \[\sum_k f(k) = 1,\] donde \textit k representa todos los valores que toma la variable aleatoria.
\end{enumerate}

\ex{Variables Aleatorias Discretas}{
  Sea $f_{\mathcal X} (0) = \frac12 \ge 0$ y $f_{\mathcal X} (1) = \frac12 \ge 0$. Se tiene que 
  \[\sum_{i = 0}^1 f_{\mathcal X}(i) = P(\mathcal X = 0) + P(\mathcal X = 1) = \frac12 + \frac12 = 1\]
  Por tanto cumple las propiedades listadas anteriormente.
}

