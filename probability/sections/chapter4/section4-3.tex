\section{Función de Distribución}
\subsection{Definición}
\dfn{Función de Distribución}{
  Dada $\mathcal X$ una variable aleatoria \textit{discreta o continua}, su \textbf{función de distribución} se define como
  \[F_{\mcX}(a) = P(\mathcal X \le a) = P(\{w \in \Omega \mid \mathcal X(w) \le a\}), \quad \text{con} \quad a \in \bbR\]
}

\nt{
  Cuando se tiene una sola variable aleatoria o cuando es evidente sobre que variable aleatoria se habla, es normal omitir el subíndice de la función $F$.
}

\nt{
  Notemos que $F : \mathbb R \longrightarrow [0,1]$
}

\ex{Función de Distribución}{
  Consideremos el ejemplo anterior donde lanzamos 3 monedas justas. Dada $\mathcal Y$ definida anteriormente, se tiene lo siguiente.
  \begin{itemize}
    \item $F_{\mathcal Y} (x) = P(\mathcal Y \le x)$
    \item $F_{\mathcal Y} (-1) = P(\mathcal Y \le -1) = 0$
    \item $F_{\mathcal Y} (0) = P(\mathcal Y \le 0) = \frac18$
    \item $F_{\mathcal Y} (0.3) = P(\mathcal Y \le 0.3) = \frac18$
    \item $F_{\mathcal Y} (2.1) = P(\mathcal Y \le 2.1) = \frac18 + \frac38 + \frac38 = \frac78$
    \item $F_{\mathcal Y} (87080685230) = P(\mathcal Y \le 87080685230) = 1$
  \end{itemize}
}

\nt{
  Notese que esta función es \textbf{acumulativa}.
}

\subsection{Propiedades}
Las siguientes son propiedades de las funciones de distribución.

\begin{enumerate}
  \item $$\text{Es una función no-decreciente.}$$
    \begin{myproof}
      Sea $x_1 \le x_2$. Queremos probar que $F_{\mcX}(x_1) \le F_{\mcX}(x_2)$.

      Tenemos que $F_{\mcX}(x_1) = P(\mcX \le x_1)$ y $F_{\mcX}(x_2) = P(\mcX \le x_2)$.

      Notemos que si $w_0 \in \{w \in \Omega \mid \mcX(w) \le x_1\}$, entonces $\mcX(w_0) \le \overbrace{x_1 \le x_2}^{\text{hipótesis}}$.

    Así, se tiene que $w_0 \in \{w \in \Omega \mid \mcX(w) \le x_2\}$, implicando que $\{\mcX(w) \le x_1\} \subseteq \{\mcX(w) \le x_2\}$, y por tanto, por los axiomas de probabilidad, se concluye
      \[F_{\mcX}(x_1) = P(\mcX \le x_1) \le P(\mcX \le x_2) = F_{\mcX}(x_2)\]
      \[\therefore \, \text{$F$ es no-decreciente}\]

    \end{myproof}
  \item \[ \lim_{x \to \infty} F_{\mcX}(x) = 1 \]
    \begin{myproof}
      Sea $\{a_n\}$ una sucesión de números reales tales que $a_n \le a_{n+1}$ y $\lim_{n\to \infty} a_n = \infty$

      Tenemos que
      \[\lim_{x \to \infty} F_{\mcX}(x) = \lim_{n \to \infty} F_{\mcX}(a_n) = \lim_{n \to \infty} P(\mcX \le a_n)\]

      Observemos que $\{\mcX \le a_n\} \subseteq \{\mcX \le a_{n+1}\}$ es una sucesión de conjuntos monótona creciente. Entonces
      \[\lim_{n\to \infty} P(\mcX \le a_n) = P\parens*{\lim_{n\to \infty}\{\mcX \le a_n\}} = P\parens*{\bigcup_{n=1}^{\infty} \{\mcX \le a_n\}} = P(\Omega) = 1 \]
      \[\therefore \, \lim_{n\to \infty}F_{\mcX}(a_n) = 1\]
    \end{myproof}
  \item \[ \lim_{x \to -\infty} F_{\mcX}(x) = 0\]
    \begin{myproof}
      Ejercicio.
    \end{myproof}
  \item $$\text{Es una función continua por la derecha.}$$
    \begin{myproof}
      Ejercicio.
    \end{myproof}
  \item \[ P(\mathcal X < a) = F_{\mcX}(a^-) \text{ limite por la izquierda}\]
    \begin{myproof}
      Ejercicio.
    \end{myproof}
  \item \[ P(\mathcal X = a) = P(\mathcal X \le a) - P(\mathcal X < a) = F_{\mcX}(a) - F_{\mcX}(a^-) \]
    \begin{myproof}
      Ejercicio.
    \end{myproof}
  \item \[ P(a< \mathcal X \le b) = P(\mathcal X \le b) - P(\mathcal X \le a) = F_{\mcX}(b) - F_{\mcX}(a) \]
    \begin{myproof}
      Ejercicio.
    \end{myproof}
  \item \[ P(a \le \mathcal X \le b) = P(\mathcal X \le b) - P(\mathcal X < a) = F_{\mcX}(b) - F_{\mcX}(a^-) \]
    \begin{myproof}
      Ejercicio.
    \end{myproof}
  \item \[ P(a < \mathcal X < b) = P(\mathcal X < b) - P(\mathcal X \le a) = F_{\mcX}(b^-) - F_{\mcX}(a) \]
    \begin{myproof}
      Ejercicio.
    \end{myproof}
  \item \[ P(a \le \mathcal X < b) = P(\mathcal X < b) - P(\mathcal X < a) = F_{\mcX}(b^-) - F_{\mcX}(a^-) \]
    \begin{myproof}
      Ejercicio.
    \end{myproof}
\end{enumerate}

