\section{Ejercicios}

\qs{Variables Aleatorias}{
  3 pelotas se eligen aleatoriamente de una caja donde 5 son azules, 3 son rojas, y 3 son amarillas. Supongamos que ganamos \$1 por cada pelota amarilla que seleccionamos y perdemos \$1 por cada pelota roja. Sea $\mathcal X =$ "Dinero obtenido en este experimento". Calcula la función de probabilidad de $\mathcal X$.
}
\qs{Variables Aleatorias}{
  Demuestra que $F_{\mathcal X}(x^-) = P(\mathcal X < x) ~ \forall x \in \mathbb R$, donde \textit{F} es la función de distribución.
}

\qs{Variables Aleatorias}{
  La función de distribución de la variable aleatoria $\mathcal X$ es la siguientes.
  \[
    F_{\mathcal X}(x) = 
    \begin{cases}
      0,  &\text{Si } x <0\\
      \sfrac x4,  &\text{Si } 0 \le x < 1\\
      \sfrac 12 + \frac{x-1}4,  &\text{Si } 1 \le x < 2\\
      \sfrac {11}{12},  &\text{Si } 2 \le x < 3\\
      1,  &\text{Si } 3 \le x\\
    \end{cases}
  \]
  Grafica esta función y encuentra $P(\frac12 < \mathcal X < \frac32)$
}
\qs{Variables Aleatorias}{
  Supongamos que se lanzan dos dados de 6 caras y que la variable aleatoria $\mathcal X$ representa la suma de los números obtenidos. encuentra la función de distribución de la variable aleatoria $\mathcal X$ y graficala.
}
\qs{Variables Aleatorias}{
  Determina la \textit{c} tal que la función pueda servir como la distribución de probabilidad de una variable aleatoria con el intervalo dado.
  \begin{itemize}
    \item $f(x) = cx,~ x = 1, \cdots 5$
    \item $f(x) = c \binom5x,~ x = 0,1, \cdots, 5$
    \item $f(x) = cx^2, ~x = 1, 2, \cdots, k$
  \end{itemize}
}

\qs{Variables Aleatorias}{
  La función de distribución de la variable aleatoria $\mathcal X$ es la siguientes.
  \[
    F_{\mathcal X}(x) = 
    \begin{cases}
      0, &\text{Si } x < -2\\
      \frac {x+2}2,  &\text{Si } -2 \le x < -1\\
      \sfrac12 , &\text{Si } -1 \le x < 1\\
      \sfrac x2,  &\text{Si } 1 \le x < 2\\
      1,  &\text{Si } 2 \le x\\
    \end{cases}
  \]
  Encuentra $P(-2 < \mathcal X < 2)$
}

\qs{Variables Aleatorias}{
  La compañía de seguros Acme tiene dos tipos de clientes: cuidadosos(as) e imprudentes. Un cliente cuidadoso(a) tiene un accidente durante el año con probabilidad 0.01. Un(a) cliente imprudente tiene un accidente durante el año con probabilidad 0.04. El 80\% de los (as) clientes son cuidadosos(as) y el 20\% de los(as) clientes son imprudentes. Supongamos que un(a) cliente elegido(a) al azar tiene un accidente este año. ¿Cuál es la probabilidad de que este(a) cliente(a) sea uno(a) de los(as) clientes cuidadosos(as)?
}

\qs{Variables Aleatorias}{
  Para cualesquiera A, $B\in\mathcal{F}$ demuestra que si $P(B \mid A)>P(B)$ entonces $P(B^{c}|A)<P(B^{c})$.
}


\qs{Variables Aleatorias}{
  Demostrar las propiedades restantes de las funciones de distribución.
}

