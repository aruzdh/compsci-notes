\nt{
  Recordemos que los subconjuntos de $\Omega$ se llaman \textbf{eventos.}
}

\nt{
  Una $\sigma$-algebra es una \textbf{familia de subconjuntos} que satisface lo siguiente.
  \begin{itemize}
    \item $\Omega \in \mathcal F$
    \item $A \in \mathcal F \Rightarrow A^c \in \mathcal F$
    \item $A_1, A_2, \cdots \in \mathcal F \Rightarrow \bigcup_n A_n \in \mathcal F$

  \end{itemize}
}

\section{Axiomas de Probabilidad}

Sea $A$ un evento, $\Omega$ el espacio muestral, y $\mathcal F$ una $\sigma$-algebra.

\begin{enumerate}
  \item $0 \le P(A) \le 1$
  \item $P(\Omega) = 1$
  \item Sean $A_1, A_2, \cdots \in \mathcal F \text{ (eventos) tales que } A_i \cap A_j = \varnothing, ~\forall i \neq j.$ Entonces $$P\parens*{\bigcup A_i} = \sum_i P(A_i)$$
\end{enumerate}
