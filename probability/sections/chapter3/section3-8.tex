\section{Ejercicios}

\qs{Espacio de Probabilidad}{
  Demuestra que:
  \[P(A^{c}|B)=1-P(A|B)\]
}

\qs{Espacio de Probabilidad}{
  Demuestra que si A y B son independientes, entonces $A^{c}$ y B también son independientes.
}

\qs{Espacio de Probabilidad}{
  Demuestra que:
  \[P(A\cap B)\ge P(A)+P(A\cap B)-1\]
}

\qs{Espacio de Probabilidad}{
  Cuatro personas han respondido a una solicitud de un banco de sangre para efectuar donaciones. Se desconoce sus tipos de sangre. Solo se desea el tipo de sangre $O+$ y solo una de ellas tiene este tipo de sangre. Si se seleccionan de forma aleatoria para determinar su tipo de sangre, ¿cuál es la probabilidad de que por lo menos 3 tengan que ser examinadas para así obtener a la persona donante O+?
}

\qs{Espacio de Probabilidad}{
  Benito tiene 3 cuentas distintas de correo electrónico. El 70\% de sus correos llegan a la cuenta 1, el 20\% a la cuenta 2 y el resto a la cuenta 3. El 1\% de los mensajes que llegan a la cuenta 1 es spam, mientras que el 2\% y el 5\% son spam en las cuentas $2$ y $3,$ respectivamente. ¿Cuál es la probabilidad de que abra al azar un correo de spam de cualquiera de las 3 cuentas?
}

\qs{Espacio de Probabilidad}{
  En una gasolinera el 40\% de los clientes utilizan gasolina regular, el 35\% gasolina plus y el 25\% premium. De los que utilizan regular solo el 30\% llenan sus tanques, de los que utilizan plus el 60\% llenan sus tanques, mientras que de los que utilizan premium el 50\% los llenan.
  \begin{itemize}
    \item ¿Cuál es la probabilidad de que el siguiente cliente pida gasolina plus y llene el tanque?
    \item ¿Cuál es la probabilidad de que el siguiente cliente llene el tanque?
    \item Si el siguiente cliente llena el tanque, ¿cuál es la probabilidad de que pida gasolina regular, gasolina plus o gasolina premium?
  \end{itemize}
}

\qs{Espacio de Probabilidad}{
  Demuestre o proporcione un contraejemplo:
  \begin{itemize}
    \item Si $A\cap B=\emptyset$ entonces $P(A)\le P(B^{\epsilon})$.
    \item Si $P(A)=P(B)=1$, entonces $P(A\cap B)=0$.
  \end{itemize}
}

\qs{Espacio de Probabilidad}{
  10 personas tienen el mismo tipo y marca de teléfono celular. Suponga que antes de presentar un examen estas personas ponen su celular en una caja. Los teléfonos son mezclados dentro de la caja. Al final del examen cada persona toma aleatoriamente un celular ¿Cuál es la probabilidad de que al menos una persona elijan su propio celular? ¿Cuál es la probabilidad de que nadie seleccione su propio celular?
}

\qs{Espacio de Probabilidad}{
  Las garrapatas de venados pueden ser portadoras de la enfermedad de Lyme o de la Ehrlichiosis Granulocítica Humana (EGH). En cierto lugar el 16\% de todas las garrapatas portan la enfermedad de Lyme, 10\% portan EGH y 10\% de las garrapatas que portan por lo menos una de estas enfermedades en realidad portan ambas. Si una garrapata seleccionada al azar ha sido portadora de EGH, ¿cuál es la probabilidad de que la garrapata seleccionada también porte la enfermedad de Lyme?
}

\qs{Espacio de Probabilidad}{
  Una ciudad tiene dos carros de bomberos que operan de forma independiente. La probabilidad de que un carro específico esté disponible cuando se necesite es 0.96.
  \begin{itemize}
    \item ¿Cuál es la probabilidad de que ningún carro esté disponible cuando se le necesite?
    \item ¿Cuál es la probabilidad de que sólo un carro de bomberos esté disponible cuando se le necesite?
  \end{itemize}
}

\qs{Espacio de Probabilidad}{
  Una de cada diez personas en una población determinada sufre de enfisema pulmonar. Si se eligen 15 personas al azar, ¿cuál es la probabilidad de que al menos una de estas personas tenga enfisema pulmonar?
}

