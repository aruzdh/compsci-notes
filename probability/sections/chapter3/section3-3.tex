\section{Probabilidad Condicional}
\dfn{Probabilidad Condicional}{
  Sean $A, B \in \mathcal F$. La \textbf{probabilidad condicional} del evento \textit{A} \textit{dado} el evento \textit{B} se define como sigue.
  \[P(A \mid B) = \frac{P(A \cap B)}{P(B)} \]
  donde necesariamente $P(B) > 0$.
}

\nt {
  Lo anterior se puede leer como la \textbf{probabilidad de $A$ dado $B$}
}

\nt {
  De la definicón anterior se tiene que 
  \[P(A \cap B) = P(A \mid B)\,P(B)\]
}

\ex{Probabilidad Condicional}{
  Se extraen 2 cartas en sucesión de una baraja de 52. Encuentra la \textit{probabilidad de que ambas cartas sean ases si}
  \begin{itemize}
    \item Se reemplaza la carta (se regresa la carta sacada).
    \item No se reemplaza la carta.
  \end{itemize}

  Sea \textit{$A_1$ = "La primera carta es un As"}, \textit {$A_2$ = "La segunda carta es un As"} eventos. Entonces\\
  $P(A_2 \cap A_1) = P(A_2 \mid A_1) ~P(A_1) = P(A_2 \mid A_1)~ \frac{4}{52}$

  Si se reemplaza la carta tenemos que 
  \[P(A_2 \cap A_1) = \parens*{\frac{4}{52}}^2 = \frac{1}{169}\]
  Si no se reemplaza tenemos que
  \[P(A_2 \cap A_1) = \frac{3}{51} \cdot \frac{4}{52} = \frac{1}{221} \]
}

