\section{Teorema de Bayes}
\subsection{Motivación}

Siguiendo con el ejemplo anterior. Supongamos que durante el experimento se sacó una pelota azul. ¿Cuál es la probabilidad de que la pelota se haya sacado de la caja \textit{II}? Se tiene que
\[P(B_2 \mid A^c) = \frac{B_2 \cap A^c}{P(A^c)} = \frac{P(A^c \mid B_2)\,P(B_2)}{\sum_{i = 1}^2 P(A^c \mid B_i)\,P(B_i)}\]

Donde 
\[P(A^c) = P(A^c \cap B_1) + P(A^c \cap B_2) = P(A^c \mid B_1)\,P(B_1) + P(A^c \mid B_2)\,P(B_2)\]

\subsection{Definición}
\dfn{Teorema de Bayes}{
  Sea $\{B_i\}_i$ es una partición de $\Omega$ y $A \in \mathcal F$ un evento. Entonces
  \[P(B_j \mid A) = \frac{P(B_j \cap A)}{P(A)}= \frac{P(A \mid B_j)\,P(B_j)}{\sum_{i = 1}^n P(A \mid B_i)\,P(B_i)}\]
}

\nt {
  En la definicón anterior se observa que $P(A)$ es la probabilidad total.
}

\ex{Teorema de Bayes}{
  Una prueba de sagre es 95\% efectiva en detectar cierta enfermedad cuando esta enfermedad está presente. Sin embargo, la prueba también puede dar falsos positivos para el 1\% de quienes se hacen la prueba y están sanos. Si el 0.5\% de la población tiene la enfermedad, ¿Cuál es la probabilidad de que una persona que dió positivo a la prueba tenga esta enfermedad?
  \newpara
  Sea \textit{A = "la persona dió positivo"}, \textit{B = "la persona tiene la enfermedad"} eventos.\\
  Además, $\Omega = B \cup B^c$ donde $P(B) = 0.5\% = 0.005$. $P(A \mid B) = 0.95$. $P(A \mid B^c) = 0.1$.
  Entonces
  \begin{align*}
    P(B \mid A) &= \frac{P(B \cap A)}{P(A)} = \frac{P(A \mid B)~ P(B)}{P(A)}\\
             &= \frac{P(A \mid B)~P(B)}{P(A | B)~P(B) + P(A \mid B^c)~ P(B^c)}\\
             &= \frac{(0.95) (0.005)}{(0.95) (0.005) + (0.1) (1 - 0.005)} = \frac{19}{417}
  \end{align*}
}

\ex{Teorema de Bayes}{
  Tengo una bolsa con 3 dados, donde uno es de 4 caras, uno de 6 caras, y una de 12 caras. Tomo aleatoriamente un de ellos y lo lanzo. Dado que se obtuvo un 4, ¿Cuál es la probabilidad de que haya lanzado el dado de 6 caras?
  \newpara
  Sean \textit{A = "lanzar el dado de 4 caras"}, \textit{B = "lanzar el dado de 6 caras"}, \textit{C = "lanzar el dado de 12 caras."}, \textit{D = "obtener un 4"} eventos.

  De lo anterior tenemos que $P(A) = P(B) = P(C) = \frac{1}{3}$. $P(D \mid A) = \frac14,~ P(D \mid B) = \frac16,~ P(D \mid C) = \frac{1}{12}$\\
  Entonces
  \begin{align*}
    P(B \mid D) &= \frac{P(B \cap D)}{P(D)} = \frac{P(D \mid B)~P(B)}{P(D)}\\
              &= \frac{P(D \mid B)~P(B)}{ P(D \mid A)~P(A) + P(D \mid B)~P(B) + P(D \mid C)~P(D)}\\
              &= \frac{\sfrac16}{\sfrac16 + \sfrac14 + \sfrac1{12}} = \frac{\sfrac16}{\sfrac12} = \frac13
  \end{align*}
}

\ex{Teorema de Bayes}{
  Una bolsa contiene 4 pelotas blancas y 2 negras. Otra bolsa contiene 3 blancas y 5 negras. Si se lanza un dado y si sale \textit{1, 2, 3 ó 4} se toma la pelota de la bolsa 1, de lo contrario se toma de la bolsa 2. Encuentra la probabilidad de que la pelota extraida sea negra.
  \newpara
  Sean \textit{A = "la pelota es negra"}, \textit{B = "se extrae de I"}, \textit{C = "se extrae de II"} eventos.
  Con lo anterior tenemos que $P(B) = \frac46,~ P(C) = \frac26$\\
  Entonces
  \begin{align*}
    P(A) &= P(A \mid B)~P(B) + P(A \mid C)~P(C)\\
          &= \frac26 \cdot \frac46 + \frac58 \cdot \frac26 = \frac26(\frac46 + \frac58) = 
          \frac{62}{48} \cdot \frac13 = \frac{62}{144} = \frac{31}{72}
  \end{align*}
}
