\section{Proposiciones}
Dados \textit{A, B} eventos.
\begin{enumerate}
  \item $P(A^c) = 1 - P(A)$
    \begin{myproof}
      Sabemos que $\Omega = A \cup A^c$ y $A \cap A^c = \varnothing$. \\
      Usando los \textit{axiomas de probabilidad} se tiene que $P(\Omega) = P(A) + P(A^c)$.\\
      De lo anterior se sigue que $1 = P(A) + P(A^c)$
      \begin{center} \boxed{\therefore \, P(A^c) = 1 - P(A)} \end{center}
    \end{myproof}

    \newpage
  \item Si $A \subseteq B$, entonces $P(A) \le P(B)$
    \begin{myproof}
      Notemos que $B = (B - A) \cup A$.\\ 
      Por los \textit{axiomas} se tiene que $P(B) = P(B-A) + P(A)$ con $P(B - A) \ge 0$.
      \begin{center} \boxed{\therefore \, P(A) \le P(B) } \end{center}
    \end{myproof}
    
  \item $P(A \cup B) = P(A) + P(B) - P(A \cap B)$
    \begin{myproof}
      Se tiene que $A \cup B = (A - B) \cup (A \cap B) \cup (B - A)$ donde cada uniendo es ajeno.\\
      Entonces
      \begin{align*}
         P(A \cup B) &= P((A - B) \cup (A \cap B) \cup (B - A))\\
                  &= P(A-B) + P(A \cap B) + P(B-A)\\
                  &= P(A) + P(B-A)\\
                  &= P(A)  + P(B) - P(B \cap A)\\
      \end{align*}
    \end{myproof}
\end{enumerate}

\ex{Proposiciones}{
  Se sacan 3 pelotas de una urna que contiene 5 pelotas azules, 4 verdes, y 3 rojas. ¿Cuál es la probabilidad de no obtener ni verdes ni rojas?
  \newpara
  Se tienen 12 pelotas en total. Sea \textit{A = "obtener 3 pelotas azules"} un evento.\\
  Entonces 
  \begin{align*}
    P(A) &= \frac{\abs*{A}}{\abs*{\Omega}} = \frac{\binom{5}{3}}{\binom{12}{3}}\\
         &= \frac{ \frac{5!}{3! 2!} }{ \frac{12!}{3! 9!} } = \frac{ \frac{5 \cdot 4 \cdot 3!}{3! \cdot 2 \cdot 1} }{ \frac{12 \cdot 11 \cdot 10 \cdot 9!}{3 \cdot 2 \cdot 1 \cdot 9!} } \\
         &= \frac{ \sfrac{20}{2} }{\sfrac{1320}{6} } = \frac{10}{220} = \frac{1}{22}
  \end{align*}
}

\ex{Proposiciones} {
  Tres pelotas son tomadas aleatoriamente de una caja que contiene 6 pelotas blancas y 5 negras. ¿Cuál es la probabilidad de que una de las pelotas sea blanca y las otras negras?
  \newpara
  Se tienen 11 pelotas en total. Sean \textit{A = "sacar 1 pelota blanca"} y \textit{B = "sacar 2 pelotas negras"} eventos.
  Entonces 
  \begin{align*}
    P(A) &= \frac{\abs*{A} \abs*{B}}{\abs*{\Omega}} = \frac{ \binom61 \binom52} {\binom{11}3} \\ 
        &= \frac{ \frac{6!}{6!} \frac{5!}{2! 3!} } { \frac{11!}{3!8!} } \\ 
        &= \frac{60} {165} = \frac{4}{11}
  \end{align*}
}

