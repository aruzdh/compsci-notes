\section{Independencia y Distribuciones Marginales}

\subsection{Variables Aleatorias Independientes}
Recordemos que si dos eventos $A, B \in \mathcal F$ son independientes, entonces
\[P(A \cap B) = P(A)P(B)\]

\dfn{Variables Aleatorias Independientes}{
  Decimos que las variables aleatorias $\mcX$ y $\mcY$ son \textbf{variables aleatorias independientes} si 
  \[\forall B, C \subseteq \bbR,\, P(\mcX \in B, \, \mcY \in C) = P(\mcX \in B)\,P(\mcY \in C)\]
}

Ademas, podemos extenderlo de la siguiente manera.
\[f_{\mcX, \mcY}(x,y) = P(\mcX = x, \, \mcY = y) = P(\mcX = x)P(\mcY = y) = f_{\mcX}(x) f_{\mcY}(y)\]

\nt{
  Si $\mcX$ y $\mcY$ son variables aleatorias \textit{discretas}, entonces $f_{\mcX}$, y $f_{\mcY}$ son funciones de probabilidad y $f_{\mcX, \mcY}$ es la función de probabilidad conjunta.
  \newpara
  Si $\mcX$ y $\mcY$ son variables aleatorias \textit{continuas}, entonces $f_{\mcX}$, y $f_{\mcY}$ son funciones de densidad y $f_{\mcX, \mcY}$ es la función de densidad conjunta.
}

\subsection{Distribuciones Marginales}

\dfn{Distribuciones Marginales (caso discreto)}{
  Sea ($\mcX, \mcY$) un vector aleatorio \textit{discreto}. Definimos la \textbf{función de probabilidad marginal} de $\mcX$ y $\mcY$ de la siguiente forma.

  \[f_{\mcX}(x) = P(\mcX = x) = \sum_j f(x,j) = \sum_j P(\mcX = x, \, \mcY = j)\]
  \[f_{\mcY}(y) = P(\mcY = y) = \sum_i f(i,y) = \sum_i P(\mcX = i, \, \mcY = y)\]

  Donde $j$ son todos los valores que toma $\mcY$, mientras que $i$ son todos los valores que toma $\mcX$.
}

\dfn{Distribuciones marginales (caso continuo)}{
  Sea ($\mcX, \mcY$) un vector aleatorio \textit{continuo}. Definimos la \textbf{función de probabilidad marginal} de $\mcX$ y $\mcY$ de la siguiente forma.

  \[f_{\mcX}(x) = \int_{-\infty}^{\infty} f_{\mcX, \mcY}(x,y) \,dy \]
  \[f_{\mcY}(y) = \int_{-\infty}^{\infty} f_{\mcX, \mcY}(x,y) \,dx \]

}

\ex{Distribuciones Marginales}{
  Consideremos el mismo experimento del ejemplo anterior (lanzar 3 monedas justas) y las mismas variables aleatorias.
  \newpara
  Las probabilidades marginales se calculan como sigue.

  \begin{align*}
    f_{\mcX}(0) &= \sum_{i=0}^2 f_{\mcX}(0,i)\\ 
                &= \sum_{i=0}^2 P(\mcX = 0, \, \mcY = i)\\
                &= P(\mcX = 0, \, \mcX = 0) + P(\mcX = 0, \, \mcX = 1) + P(\mcX = 0, \, \mcX = 2)\\
                &= \frac18 + \frac18 + 0 = \frac14
  \end{align*}
  El resultado anterior nos dice que la probabilidad de obtener 0 soles en los últimos dos lanzamientos es de $\frac14$.

  Usando el mismo método se obtiene lo siguiente.
  \[
    P(\mcX = 0) = P(\mcX = 0, \, \mcX = 0) + P(\mcX = 0, \, \mcX = 1) + P(\mcX = 0, \, \mcX = 2) = \frac18 + \frac18 + 0 = \frac14
  \]
  \[
    P(\mcX = 1) = P(\mcX = 1, \, \mcX = 0) + P(\mcX = 1, \, \mcX = 1) + P(\mcX = 1, \, \mcX = 2) = \frac18 + \frac14 + \frac18 = \frac12
  \]
  \[
    P(\mcX = 2) = P(\mcX = 2, \, \mcX = 0) + P(\mcX = 2, \, \mcX = 1) + P(\mcX = 2, \, \mcX = 2) = 0 + \frac18 + \frac18 = \frac14
  \]
  El caso para la variables aleatorias $\mcY$ es análogo.
}

Así, podemos afirmar que, dadas dos variables aleatorias independientes $\mcX$ y $\mcY$, se tiene que

\begin{itemize}
  \item Caso discreto: $P(\mcX = x, \, \mcY = y) = P(\mcX = x)P(\mcY = y)$
  \item Caso continuo: $f_{\mcX, \mcY}(x,y) = f_{\mcX}(x) f_{\mcY}(y)$
\end{itemize}

