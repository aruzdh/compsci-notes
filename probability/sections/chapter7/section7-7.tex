\section{Ejercicios}

\qs{Teorema del Límite Central}{
  Supón que 3 pelotas son sacadas sin remplazarlas de una caja que contiene 5 pelotas blancas y 8 rojas. Sea $\mcX_i$ la variable aleatoria que toma el valor de $1$ si la i-énesima pelota elegida es blanca y 0 si es roja. Encuentra la función de probabilidad conjunta de $\mcX_1$ y $\mcX_2$, así como la de $\mcX_1, \mcX_2$ y $\mcX_3$.
}


\qs{Teorema del Límite Central}{
  Se lanza 40 veces una moneda. Encuentra $P(\mcX = 20)$ usando el Teorema del Límite Central.
}


\qs{Teorema del Límite Central}{
  El número de estudiantes de se inscriben en un curso de cálculo es una variable aleatoria Poisson con $\lambda = 100$. Quien organiza los horarios ha decidido que si el número de estudiantes matriculados es de 120 o más, se abrirán 2 grupos. ¿Cuál es la probabilidad de abrir 2 grupos?
}

\qs{Teorema del Límite Central}{
  Sea $\mcX$ una variable aleatoria absolutamente continua con función generadora de momentos $M_{\mcX}(t)$ para toda $t$. Demuestra que
  \[P(\mcX \ge x) \le e^{-t\mcX}M_{\mcX}(t)\]
  para toda $t \ge 0$
}

\qs{Teorema del Límite Central}{
  Si $\mcX \sim exp(\lambda_1)$ y $\mcY \sim exp(\lambda_2)$, ¿$\mcX + \mcY \sim exp(\lambda_1 + lambda_2)$
}

\qs{Teorema del Límite Central}{
  Sea $\mcX_1, \cdots, \mcX_n, \cdots$ una sucesión de variables aleatorias independientes e idénticamente distribuidas con valor esperado común $\mu$. Considera la media muestral $\bar{\mcX} = \frac{1}{n}\sum_{i = 1}^n \mcX_i$

  \begin{itemize}
    \item Calcula $\E(\bar{\mcX})$ y $\Var(\bar{\mcX})$
    \item Sea $\epsilon > 0$. Usa la desigualdad de Markov o el Teorema de Chebyshev para demostrar lo siguiente.
      \[\lim_{n \to \infty} P(\abs*{\bar{\mcX}-\mu} > \epsilon) = 0\]
  \end{itemize}
}

\qs{Teorema del Límite Central}{
  Usa el Teorema del Límite Central para resolver lo siguiente.
  Si $\mcX$ es el número de soles que se obtiene en los 60 lanzamientos de una maneda, encuentra la probabilidad $P(\mcX = 20)$.
}

\qs{Teorema del Límite Central}{
  Usa el Teorema del Límite Central para calcular $P(\mcX \ge 20)$, donde $\mcX \sim Poi(100)$
}

\qs{Teorema del Límite Central}{
Para la variable aleatoria $\mcX$ cuya $M_{\mcX}(t) = (1-2t)^{-\frac{1}{2}}$ para $t < \frac{1}{2}$, encuentra $\E(\mcX^3)$
}

\qs{Teorema del Límite Central}{
  Sean $\{\mcZ_i\}_{1\le i \le n}$ variables aleatorias independientes tales que $\mcZ_i \sim N(0,1)$. Calcula $M_{\mcY}$ donde
  \[\mcY = \sum_{i=1}^n \mcZ_i^2\]
}
