
\documentclass{article}
\usepackage[utf8]{inputenc}
\usepackage{amssymb}
\usepackage[fleqn]{amsmath}
\usepackage[top=2cm, bottom=2cm, left=2.5cm, right=2.5cm]{geometry}

\title{Autom\'atas y Lenguajes Formales 25-2 \\ Tarea 1: Cadenas y Lenguajes}
\author{Hernández Vázquez Carlos Arturo }
\date{Martes 11 de Febrero del 2025}

\begin{document}

\maketitle

\begin{enumerate}
    \item (6pts) Dado el siguiente alfabeto $\Sigma = \{a, b, k\}$ y las siguientes palabras sobre el alfabeto $\Sigma$, $u = baba$, $v = kk$ y $w = kab$. Define las siguientes palabras.
    \begin{enumerate}
        \item $(uv)^2w^R = (babakk)^2(kab)^R = babakkbabakkbak$
        \item $w^R u^0 v^1 = (kab)^R(baba)^0(kk)^1 = bakkk$ 
        \item $(((uv)^R)^3)^0 = (((babakk)^R)^3)^0 = \varepsilon$ 
    \end{enumerate}
    \item (8pts) Considera la cadena $w = \text{unam}$
    \begin{enumerate}
        \item Lista todos los prefijos de $w$. \\
        $\varepsilon, u, un, una, unam$
        \item Lista todos los prefijos propios de $w$. \\
        $\varepsilon, u, un, una$
        \item Lista todos los sufijos de $w$. \\
        $\varepsilon, m, am, nam, unam$
        \item Lista todos los sufijos propios de $w$. \\
        $\varepsilon, m, am, nam$
    \end{enumerate}
    \item (18pts) Dado el alfabeto $\Sigma = \{a, b, c\}$ y los lenguajes $L_1 = \{ca, bb, ab\}$, $L_2 = \{\varepsilon, ca, ba, ac\}$, $L_3 = \{\varepsilon\}$, $L_4 = \varnothing$, define los siguientes lenguajes.
    \begin{enumerate}
        \item $L_1 \cap L_2^R = \{ca, bb, ab\} \cap \{\varepsilon, ac, ab, ca\} = \{ca, ab\}$
        \item $L_2 L_1 = \{ca, bb, ab, caca, cabb, caab, baca, babb, baab, acca, acbb, acab\} $
        \item $L_1 (L_2 \cup L_3) = \{ca, bb, ab\}(\{\varepsilon, ca, ba, ac\}) = \{ca, caca, caba, caac, bb, bbca, bbba, bbac, ab, abca, abba, abac\}$
        \item $L_1 (L_2 \cap L_4) = \{ca, bb, ab\}(\varnothing) = \varnothing$
        \item $L_2^+$ Escribe hasta la potencia 2 del lenguaje.

        \begin{itemize}
        \item $L^1_2 = \{\varepsilon, ca, ba, ac\}$
        \item $L^2_2 = \{\varepsilon, ca, ba, ac, caca, caba, caac, baca, baba, baac, acca, acba, acac\}$
        \item $L^*_2 = \{\varepsilon, ca, ba, ac, caca, caba, caac, baca, baba, baac, acca, acba, acac, ...\}$
        \end{itemize}
        \item $L_1^*$ Escribe hasta la potencia 2 del lenguaje.
        \begin{itemize}
            \item $L^0_1 = \{\varepsilon\}$
            \item $L^1_1 = \{ca, bb, ab\}$
            \item $L^2_1 = \{caca, cabb, caab, bbca, bbbb, bbab, abca, abbb, abab\}$
            \item $L^*_1 = \{\varepsilon, ca, bb, ab, caca, cabb, caab, bbca, bbbb, bbab, abca, abbb, abab, ...\}$
        \end{itemize}
    \end{enumerate}
    
    \item (8pts) Sea $L$ el lenguaje de todas las cadenas sobre el alfabeto $\Sigma = \{0, 1\}$ con longitud igual a 4 y que tienen un número par de 0's.

    $L = \{1111,0011, 1100, 0101, 1010, 0110, 1001, 0000\}$
    \begin{enumerate}
        \item Calcula $L^0$, $L^1$, $L^2$ y $L^3$.
        \begin{itemize}
            \item $L^0 = \{\varepsilon\}$
            \item $L^1 = \{1111, 0011, 1100, 0101, 1010, 0110, 1001, 0000\}$
            \item $L^2 = \{ 11111111, 11110011, 11111100, 11110101, 11111010, 11110110, 11111001, 11110000, \\
                            00111111, 00110011, 00111100, 00110101, 00111010, 00110110, 00111001, 00110000, \\ 
                            11001111, 11000011, 11001100, 11000101, 11001010, 11000110, 11001001, 11000000, \\
                            01011111, 01010011, 01011100, 01010101, 01011010, 01010110, 01011001, 01010000, \\
                            10101111, 10100011, 10101100, 10100101, 10101010, 10100110, 10101001, 10100000, \\
                            01101111, 01100011, 01101100, 01100101, 01101010, 01100110, 01101001, 01100000, \\
                            10011111, 10010011, 10011100, 10010101, 10011010, 10010110, 10011001, 10010000, \\
                            00001111, 00000011, 00001100, 00000101, 00001010, 00000110, 00001001, 00000000\}$
            \item $L^3 = \{
                            001100110011, 001100111100, 001100110101, 001100111010, 001100110110, 001100111001, \\ 001100110000, 001111000011, 001111001100, 001111000101, 001111001010, 001111000110, \\
                            001111001001, 001111000000, ...\}$
        \end{itemize}
        \item Determina si $L^+ = L^*$. Justifica tu respuesta. \\
        Teniendo en cuenta que $\varepsilon \notin L$, y por tanto $\varepsilon \notin L^1$, sabemos que $\varepsilon \notin L^+$ ya que en la unión definida por $L^+$ solo tomamos los elementos de $L^n$ con $n>0$. Caso contrario ocurre con $L^*$, donde tomamos los elementos de $L^n$ con $n \geq 0$, y en particular $\varepsilon \in L^0$ y por ende $\varepsilon \in L^*$. \\  
        Así que $L^+ \neq L^*$
    \end{enumerate}
    
    \item (10pts) Encuentra lenguajes que cumplan con cada una de las siguientes condiciones, argumentando porque tu lenguaje cumple con la condición.
    \begin{enumerate}
        \item Un lenguaje $L$ en $\{a, b\}$ que no sea $\{\varepsilon\}$ ni $\{a, b\}^*$ y se satisfaga $L = L^*$.
        \begin{itemize}
            \item $L = \{a^n \mid n \geq 0\} = \{\varepsilon, a, aa, aaa, aaaa, ...\}$
            \item $L^0 = \{\varepsilon\}$
            \item $L^1 = \{a^n \mid n \geq 0\} = \{\varepsilon, a, aa, aaa, aaaa, ...\}$
            \item $L^2 = \{a^n \mid n \geq 0\} = \{\varepsilon, a, aa, aaa, aaaa, ...\}$ \\
            Lo anterior se sostiene para cualquier $L^n$ con $n \geq 1$ porque cada vez que concatenemos una cadena de $L^1$ con cualquiera de algún $L^n$, obtendremos una cadena compuesta de a's (o $\varepsilon$ en un caso particular) que por la definición de $L$ ya se encuentra en el conjunto.\\
            Por tanto, la unión de todas las potencias de $L$, es decir $L^*$, es igual a $L$.
        \end{itemize}
        \item Un lenguaje $L$ en $\{a, b\}$ que no sea $\{\varepsilon\}$ ni $\{a, b\}^*$ y se satisfaga $L^0 = L^*$.
        \begin{itemize}
            \item $L = \varnothing$
            \item $L^0 = \{\varepsilon\}$
            \item $L^1 = \varnothing$
            \item $L^2 = \varnothing$ \\
            Lo anterior continua para cualquier $n \geq 3$ debido a que $L$ es vacío. Así, al calcular $L^*$, el único elemento presente en la unión es $\varepsilon$. Por tanto $L^0 = L^*$
        \end{itemize}
    \end{enumerate}
    
    \item (10pts) Demuestra que para cualquier lenguaje $L$ se cumple $(L^+)^* = (L^*)^+$.
    
    Por demostrar: $(L^+)^* = (L^*)^+$ 

    $\boxed{\Rightarrow} (L^+)^* \subseteq (L^*)^+$

    Sea $w \in (L^+)^*$. Por definición de $*$, $w \in \bigcup_{i \geq 0} (L^+)^i$, es decir, $w \in (L^+)^0 \cup (L^+)^1 \cup (L^+)^2 \cup ...$

    Lo anterior es igual a $w \in \{\varepsilon\} \cup (L^+)^1 \cup (L^+)^2 \cup ...$, y a su vez $w \in \{\varepsilon\} \cup (L^+)^+.$
    Ademas sabemos que $(L^+)^+ = L^+$, asi que $w \in \{\varepsilon\} \cup (L^+)$, y por definición de *, $w \in L^*$

    Por otro lado, para que $w \in (L^*)^+$, se debe cumplir que $w \in \bigcup_{i > 0}(L^*)^i$, es decir, $w \in (L^*)^1 \cup (L^*)^2 \cup (L^*)^3 \cup ...$ Pero sabemos que $w \in L^* = (L^*)^1$, y por definición de unión, $w \in \bigcup_{i > 0}(L^*)^i$, por tanto $w \in (L^*)^+$
    
    $\boxed{\Rightarrow} (L^*)^+ \subseteq (L^+)^*$

    Sea $w \in (L^*)^+$. Por definición de $+$, $w \in \bigcup_{i > 0}(L^*)^i$, es decir, $w \in (L^*)^1 \cup (L^*)^2 \cup (L^*)^3 \cup ...$ Sin perdida de generalidad, supongamos que $w \in (L^*)^i$, con $i > 0$ \\
    Por otro lado, para que $w \in (L^+)^*$, debe ocurrir que $w \in \bigcup_{i \geq 0}(L^+)^i = (L^+)^0 \cup (L^+)^1 \cup (L^+)^2 \cup ... =\{\varepsilon\} \cup L^+ \cup (L^+)^2 \cup ... = L^* \cup (L^+)^2 \cup ...$ Pero sabemos que $w \in (L^*)^i = L^*$ (demostración hecha en clase), entonces $w$ est\'a en algún uniendo, y por tanto $w \in \bigcup_{i \geq 0} (L^+)^i = (L^+)^*$ \\
    $\therefore ~(L^+)^* =  (L^*)^+$
    
    \item (10pts) Da un ejemplo de lenguajes $L_1$ y $L_2$ que satisfagan que $L_1L_2 = L_2L_1$ y la condición planteada en cada inciso, por separado.
    \begin{enumerate}
        \item Ninguno de los dos lenguajes es un subconjunto propio del otro, ni es $\{\varepsilon\}$.
        \begin{itemize}
            \item $L_1 = \{s, m\}$
            \item $L_2 = \{s, m\}$
            \item $L_1L_2 = \{ss, sm, ms, mm\}$
            \item $L_2L_1 = \{ss, sm, ms, mm\}$
        \end{itemize}
        \item $L_1$ es un subconjunto no vacío propio de $L_2$ y $L_1 \neq \{\varepsilon\}$.
        \begin{itemize}
            \item $L_1 = \{s\}$
            \item $L_2 = \{\varepsilon, s\}$
            \item $L_1L_2 = \{s, ss\}$
            \item $L_2L_1 = \{s, ss\}$
        \end{itemize}
    \end{enumerate}
    
    \item (20pts) Sean $L_1$, $L_2$ y $L_3$ lenguajes sobre $\Sigma$. Determina la relación que existe entre los dos lenguajes de cada inciso, son iguales o uno es subconjunto del otro. Argumenta tu respuesta o da un contraejemplo.
    \begin{enumerate}
        \item $L_1(L_2 \cap L_3)$; $L_1L_2 \cap L_1L_3$: Consideremos $L_1 = \{\varepsilon, a\},~L_2 = \{\varepsilon\},~ L_3 = \{a\}$  \\
            Por un lado $\{\varepsilon, a\} (\{\varepsilon\}\cap \{a\}) = \{\varepsilon, a\}(\varnothing) = \varnothing$ \\
            Por otro $(\{\varepsilon, a\}\{\varepsilon\})\cap(\{\varepsilon, a\}\{a\}) = \{\varepsilon, a\} \cap \{a, aa\} = \{a\}$ \\
            Con lo anterior, queda mostrado que no se cumple la igualdad.
            
            Ahora, sea $w \in L_1(L_2 \cap L_3)$, $w = sm$, donde $s \in L_1 \land m \in L_2 \cap L_3$. A su vez, $m \in L_2 \land m \in L_3$ \\
            Con lo anterior, $sm \in L_1L_2 \land sm \in L_1L_3$
            Entonces $sm = w \in L_1L_2 \cap L_1L_3$

            $ \therefore ~L_1(L_2 \cap L_3) \subseteq L_1L_2 \cap L_1L_2$
        \item $L_1^* \cap L_2^*$; $(L_1 \cap L_2)^*$: Consideremos $L_1 = \{a\},~ L_2 = \{aa\}$\\
        Tenemos que $L_1^* = \{a^n \mid n \ge 0\}$ y $L_2^* = \{(aa)^m \mid m \ge 0\}$. Considerando la intersección de los dos conjuntos anteriores, $L_1^* \cap L_2^* = \{a^{2n} \mid n \ge 0\}$ \\
        Por otro lado, $L_1 \cap L_2 = \varnothing$, y entonces $(L_1 \cap L_2)^* = \varnothing$. Asi, $L_1^* \cap L_2^* \neq (L_1 \cap L_2)^*$\\
        
        Ahora, sea $w \in (L_1 \cap L_2)^*$. Por *, $w = w_1w_2...w_n$ con $w_i \in L_1 \cap L_2$ tal que $w_i \in L_1 \land w_i \in L_2$. \\
        Usando la definición de *, $w \in L_1^* \land w \in L_2^*$. Por tanto, $w \in L_1^* \cap L_2^*$ \\

        $\therefore ~ (L_1 \cap L_2)^* \subseteq L_1^*L_2^*$
        
        \item $L_1^* L_2^*$; $(L_1 L_2)^*$: Consideremos $L_1 = \{a\}, ~L_2 = \{b\}$\\
        Tenemos que $L_1^* = \{a^n \mid n \ge 0\}, ~L_2^* = \{b^m \mid m \ge 0\}$ \\
        Ahora, $L_1^*L_2^* = \{sm \mid s \in L_1^* \land m \in L_2^*\} = \{a^nb^m \mid n,m \ge 0\} = \{\varepsilon, a, b, ab, a^2, b^2, a^2b, ab^2, a^2b^2, ...\}$ \\
        Por otro lado, $L_1L_2 = \{ab\}, ~(L_1L_2)^* = \bigcup_{i \ge 0}(L_1L_2)^i = \{\varepsilon\} \cup L_1L_2 \cup (L_1L_2)^2 \cup ...$ \\
        Esto ultimo es $(L_1L_2)^* = \{\varepsilon, ab, abab, (ab)^3, (ab)^4, ...\}$ \\
        En particular $abab \notin L_1^*L_2^* \land abab \in (L_1L_2)^*$, y por ende la igualdad no se cumple. \\

        Dado que se pide determinar si se cumple la igualdad \'o uno es subconjunto del otro, y acabamos de mostrar que $(L_1L_2)^* \nsubseteq L_1^*L_2^*$, la única opción que resta es que $L_1^*L_2^* \subseteq (L_1L_2)^*$\\
        
        $\therefore ~ L_1^*L_2^* \subseteq (L_1L_2)^*$
        
        \item $L_1^*(L_2L_1^*)^*; (L_1^*L_2)^*L_1^*$
        \begin{align*}
            L_1^*(L_2L_1^*)^* &= (L_1)^*((L_2L_1^*)^0 \cup (L_2L_1^*)^1 \cup (L_2L_1^*)^2 \cup ...) ~\mbox{--- por *}\\
                          &= ((L_1)^*(L_2L_1^*)^0 \cup (L_1)^*(L_2L_1^*)^1 \cup (L_1)^*(L_2L_1^*)^2 \cup ...) ~\mbox{--- distributiva}\\
                          &= ((L_1)^*(\{\varepsilon\}) \cup (L_1)^*(L_2L_1^*) \cup (L_1)^*(L_2L_1^*)(L_2L_1^*) \cup ...) ~\mbox{--- desarrollando potencias}\\
                          &= ((\{\varepsilon\})(L_1^*)^* \cup (L_1^*L_2)(L_1^*) \cup (L_1^*L_2)(L_1^*L_2)L_1^* \cup ...) ~\mbox{--- conmutatividad y asociatividad}\\
                          &= ((L_1^*L_2)^0(L_1^*)^* \cup (L_1^*L_2)(L_1^*) \cup (L_1^*(L_2)(L_1^*L_2)L_1^* \cup ...) ~\mbox{--- sustituyendo }\\
                          &= ((L_1^*L_2)^0\cup (L_1^*L_2)^1\cup (L_1^*L_2)^2 \cup ...)L_1^* ~\mbox{--- distributiva} \\
                          &= (L_1^*L_2)^*L_1^* ~\mbox{--- por *}
        \end{align*}

        $\therefore ~ L_1^*(L_2L_1^*) = (L_1^*L_2)^*L_1^*$
    \end{enumerate}
    
    \item (10pts) Demuestra por inducción estructural que para cualquier cadena $w$:
    \[
        \textit{rev} (\textit{Pref}(w)) = \textit{Suf} (\textit{rev} (w))
    \]
    donde $\textit{Pref}(w) = \{ x \mid x \text{ es prefijo de } w \}$ y $\textit{Suf}(w) = \{ y \mid y \text{ es sufijo de } w \}$. \\

    Por demostrar: $rev(Pref(w)) = Suf(rev(w))$ \\
    
    Case base: Sea $w = \varepsilon$, $rev(Pref(w) = Suf(rev(w))$ \\
    Tengamos en cuenta que $\varepsilon$ solo tiene como único prefijo y sufijo a $\varepsilon$. (1) \\
    \begin{align*}
      rev(Pref(w)) &= rev(Pref(\varepsilon)) ~\mbox{--- sustituyendo}\\
                      &= rev(\{\varepsilon\}) ~\mbox{--- por (1)}\\
                      &= \{\varepsilon\} ~\mbox{--- aplicando rev}\\
                      &= Suf(\varepsilon) ~\mbox{--- por (1)}\\
                      &= Suf(rev(\varepsilon)) ~\mbox{--- aplicando rev}\\
                      &= Suf(rev(w)) ~\mbox{--- sustituyendo}\\
    \end{align*}

    Hipótesis de inducción: Sea $ w \in \Sigma^*,~ rev(Pref(w)) = Suf(rev(w))$ \\

    Paso inductivo: Sea $w \in \Sigma^*, a \in \Sigma, \mbox{y sea}~ w' =wa$, $rev(Pref(w')) = Suf(rev(w'))$
    \begin{align*}
      rev(Pref(w')) &= rev(Pref(wa)) ~\mbox{--- sustituyendo} \\
                      &= rev(Pref(w) \cup \{wa\}) ~\mbox{--- definicion de prefijo} \\
                      &= rev(Pref(w)) \cup rev(\{wa\}) ~\mbox{--- propiedad de reversa} \\
                      &= Suf(rev(w)) \cup rev(\{wa\}) ~\mbox{--- Hipotesis de induccion} \\
                      &= Suf(rev(w)) \cup \{a(rev(w))\} ~\mbox{--- reversa de un conjunto} \\
                      &= Suf(a (rev(w))) ~\mbox{--- definicion de sufijo} \\
                      &= Suf(rev(wa)) ~\mbox{--- reversa} \\
                      &= Suf(rev(w')) ~\mbox{--- sustitucion}
    \end{align*}

    $\therefore ~rev(Pref(w)) = Suf(rev(w))$ para cualquier cadena $w$
\end{enumerate}
\end{document}
