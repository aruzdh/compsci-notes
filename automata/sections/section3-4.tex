\section{Autómata Mínimo}
Dado que la construcción de autómatas puede generar instancias demasiado grandes y complejas, se busca \textit{minimizar} los autómatas para obtener una versión de ellas más eficientes.

\subsection{Estados inalcanzables}
Anteriormente se hizo mención de la posibilidad de que un autómata posea \textit{estados inalcanzables}. De manera formal se tiene lo siguiente.

\dfn{Estados Inalcanzables}{
  Sea \textit{M} un \textit{AFD} definido de la forma usual. Decimos que un estado \textit{q} es \textbf{alcanzable} \textit{si y solo si existe} $w \in \Sigma^*$ tal que $\delta^*(q_0, w) = q$. Es decir, \textit{q} es \textit{alcanzable} si, una vez procesada una cadena, el autómata termina en \textit{q}.

  Si un estado \textbf{no} es alcanzable, se dice que es \textbf{inalcanzable}.
}

\nt {
  Existen definiciones más formales que la anterior, e incluso algoritmos para conocer el conjunto de estados alcanzables de un autómata finito. Por cuestiones de practicidad se asume que dado un AF con \textit{Q} su conjunto de estados, $\forall q \in Q$, \textit{q es alcanzable}.
  Además, fuera del campo educativo, no se común trabajar con autómatas que contienen estados inalcanzables, ya que esto solo generiaría información inútil.
}

\section{Lenguaje de un estado}

\dfn{Lenguaje de un estado}{
  Sea $M = (Q, \Sigma, \delta, q_0, F)$ un \textit{AFD}. Se define el \textbf{lenguaje de un estado} $q \in Q$ como
  \[\mathcal L(q) = \{w \in \Sigma^* \mid \delta^*(q,w) \in F\}\]
}

\ex{Lenguaje de un estado}{
  Dado el siguiente AFD.
  \begin{center}
    \begin{tikzpicture}[shorten >=1pt, node distance=2.5cm, on grid, auto]
        \tikzset{initial text={}}
        
        % Estados
        \node[state, initial] (q0) {$q_0$}; 
        \node[state] (q1) [above right=of q0] {$q_1$}; 
        \node[state, accepting] (q2) [right=of q1] {$q_2$}; 
        \node[state] (q3) [below right=of q0] {$q_3$}; 
        \node[state, accepting] (q4) [right=of q3] {$q_4$}; 
        
        % Transiciones
        \path[->]
            (q0) edge node {1} (q3)
            (q1) edge node {0} (q2)
            (q1) edge [loop above] node {1} (q1)
            (q2) edge [loop above] node {1} (q2)
            (q3) edge node {0} (q4)
            (q3) edge [loop above] node {1} (q3)
            (q4) edge [loop above] node {1} (q4)
            (q0) edge node {0} (q1);
    \end{tikzpicture}
  \end{center}

  Se tiene lo siguiente.
  \begin{itemize}
    \item $\mathcal L(q_0) = 01^*01^* + 11^*01^* = (0 + 1)1^*01^*$
    \item $\mathcal L(q_1) = 1^*01^*$
    \item $\mathcal L(q_2) = 1^*$
    \item $\mathcal L(q_3) = 1^*01^*$
    \item $\mathcal L(q_4) = 1^*$
  \end{itemize}
}

\subsection{Estados equivalentes}

\dfn{Estados equivalentes}{
  Sea $M = (Q, \Sigma, \delta, q_0, F)$ un \textit{AFD}. Dado $q, p \in Q$, decimos que son \textbf{equivalentes} si
  \[\mathcal L(q) = \mathcal L(p)\]
  De manera alterna, se dice que
  \[q \equiv p\]
}

\nt{
  La relación $\equiv$ entre estados es una \textit{relación de equivalecia}, por la que cumple las siguientes propiedades.
  \begin{itemize}
    \item Reflexividad: $\forall q \in Q,~ q \equiv q$
    \item Simetría: $\forall q, p \in Q;~ q \equiv p \Longrightarrow p \equiv q$
    \item Transitividad: $\forall q, p, s \in Q;~ q \equiv p \land p \equiv s \Longrightarrow q \equiv s$
  \end{itemize}
}

\nt{
  Además, la función $\delta$ es compatible con $\equiv$:
  \[q \equiv p \Longrightarrow \delta(q,a) \equiv \delta(p,a),~ \forall a \in \Sigma;~ q,p \in Q\]
}

De manera usual, la relación $\equiv$ induce una \textit{partición} del conjunto de estados dada por las clases de equivalencia de cada estado. Se tiene que
\[[q] = \{p \in Q \mid q \equiv p\}\]
Evidentemente, dicha partición cumple lo siguiente.
\begin{itemize}
  \item $\forall q \in Q,~ [q] \neq \varnothing$
  \item $\forall q,p \in Q,~ [q] = [p] \lor [q] \cap [p] = \varnothing$
  \item $\bigcup_{q \in Q}[q] = Q$
\end{itemize}

\section{Autómata Mínimo}

\dfn{Autómata Mínimo}{
  Dado $M = (Q, \Sigma, \delta, q_0, F)$ un \textit{AFD}, existe el \textbf{autómata mínimo} (también llamade \textbf{autómata cociente}) $M_{/\equiv}$ (también denotado como $M^{min}$) = ($Q_M, \Sigma, \delta_M, [q_0], F_M$) tal que
  \begin{itemize}
    \item $Q_M = \{[q] \mid q \in Q\}$
    \item $F_M = \{[p] \mid p \in F\}$
  \item $\delta_M([q], a) = [\delta(q,a)]$
  \end{itemize}
}

Con esta definición dada, lo siguiente es mostrar que ambos autómatas son equivalentes, es decir, $\mathcal L(M) = \mathcal L(M_{/\equiv})$ y que \textbf{no} existe un autómata equivalente a $M$ que tenga menos estados que $M_{/\equiv}$. Lo anterior se sigue de los siguientes lemas.

\mlemma{}{
  Sea $M = (Q, \Sigma, \delta, q_0, F)$ un \textit{AFD} y $M^{min}$ su autómata mínimo. Para cualesquiera $q \in Q,~ w \in \Sigma^*$ se cumple 
  \[\delta^*_{min}([q], w) = [\delta^*(q,w)]\]
}

\mlemma{}{
  Sean $M = (Q, \Sigma, \delta, q_0, F)$ un \textit{AFD} y $M^{min} = (Q_m, \Sigma, \delta_m, [q_0]_m, F_m)$ su autómata mínimo. Si $N = (R, \Sigma, \delta_N, r_0, F_N)$ es un \textit{AFD} tal que $\mid R \mid < ~\mid Q_m \mid$ entonces $\mathcal L(N) \neq \mathcal L(M)$. Es decir, si \textit{N} tiene menos estados que $M^{min}$, entonces \textit{M} y \textit{N} no son equivalentes.
}

Ambas demostraciones son tan evidentes que se omiten (son vacaciones, dsp lo escribo xd) $\blacksquare$

\section{K-equivalencia}
La demostración del último lema enunciado requiere de una relación de equivalencia que dependa de la longitud de una cadena.

\dfn{K-equivalencia}{
  Dado dos estados, \textit{q} y \textit{p} son \textbf{K-equivalentes} ($q \equiv_k p$) de la siguiente forma.
  \[\forall w \in \Sigma^*, ~ \mid w \mid ~ \le k \Longrightarrow \delta^*(q,w) \in F \Longleftrightarrow \delta^*(p,w) \in F\]
}

Se tiene que $\equiv_k$ es una relación de equivalencia cuyas clases se denotan con $[q]_k$ tal que
\[[q]_k = \{p \in Q \mid q \equiv_k p\}\]

Las siguientes son propiedades que cumple la relación anterior.

\begin{enumerate}
  \item $q \equiv p \Longleftrightarrow \forall k \in \mathbb N,~ q \equiv_k p$
  \item $q \equiv_0 p \Longleftrightarrow q, p \in F \lor q,p \in Q - F$
  \item $[q]_0 = F \Longleftrightarrow q \in F$
  \item $q \equiv_k p \Longrightarrow q \equiv_{k-1} p$
  \item $[q]_k \subseteq [q]_{k-1}$
  \item $q \equiv_k p \Longrightarrow \forall a \in \Sigma,~ \delta(q,a) \equiv_{k-1}\delta(p,a)$
  \item $q \equiv_k p \Longleftrightarrow q \equiv_{k-1} p \land \forall a \in \Sigma,~ \delta(q,a) \equiv_{k-1} \delta(p,a)$
  \item Sea $P_k = \{[q]_k \mid q \in Q\}$ la partición dada por la relación $\equiv_k$ para cualquier $k \in \mathbb N$. \\$P_k = P_{k-1} \Longrightarrow \forall m \ge k,~ P_k = Pm $
\end{enumerate}

\section{Autómata mínimo de un AFD}
Con todo lo anterior, se puede dar uno definición algoritmica que construya el \textit{autómata mínimo} dado un \textit{AFD}. Momentaneamente, se da solo un ejemplo de dicho algoritmo.

\ex{Autómata mínimo de un AFD}{
    Se minimiza el siguiente autómata.
    \begin{center}
        \centering
        \begin{tabular}{c|c|c}
            $\delta$ & $a$ & $b$ \\ \hline
            $*~\rightarrow q_0$ & $q_1$ & $q_3$ \\
            $q_1$ & $q_2$ & $q_4$ \\
            * $q_2$ & $q_1$ & $q_5$ \\
            $q_3$ & $q_4$ & $q_0$ \\
            $q_4$ & $q_5$ & $q_1$ \\
            $q_5$ & $q_4$ & $q_2$
        \end{tabular}
    \end{center}

    Primero podemos observar que no contiene estados inalcanzables, ya que de $q_0$ podemos llegar a $q_1$ y $q_3$; de $q_1$ a $q_2,q_4$; de $q_3$ a $q_4, q_0$; y de $q_4$ a $q_5, q_1$.

    Separamos los estados finales de los no finales:
    \begin{enumerate}
      \item Finales = $A = \{q_0, q_2\}$
      \item No finales = $B = \{q_1, q_3, q_4, q_5\}$
    \end{enumerate}
    \begin{center}
      \centering
      \begin{tabular}{c | c | c | c | c | c | c}
        & \multicolumn{2}{c|}{A} & \multicolumn{4}{c}{B} \\    
        \hline
        & $q_0$ & $q_2$ & $q_1$ & $q_3$ & $q_4$ & $q_5$   \\
        \hline
        a & B & B & A & B & B & B \\
        b & B & B & B & A & B & A
      \end{tabular}
    \end{center}

    Entonces obtenemos los siguiente grupos:
    \begin{enumerate}
      \item $A = \{q_0, q_2\}$
      \item $B = \{q_1\}$
      \item $C = \{q_3, q_5\}$
      \item $D = \{q_4\}$
    \end{enumerate}

    Partiendo de los anteriores se obtiene:
    \begin{center}
      \centering
      \begin{tabular}{c | c | c | c | c | c | c}
        & \multicolumn{2}{c|}{A} & \multicolumn{2}{c |}{C} & B & D \\    
        \hline
        & $q_0$ & $q_2$ & $q_3$ & $q_5$ & $q_1$ & $q_4$\\
        \hline
        a & B & B & D & D & A & C  \\
        b & C & C & A & A & D & B
      \end{tabular}
    \end{center}

    No se observan cambios, y por lo tanto el autómata minimo queda de la siguiente manera:
    \begin{center}
      \begin{tabular}{c | c | c}
      & a & b \\
      \hline
        *$\rightarrow$ A & B & C \\
        B & A & D \\
        C & D & A \\
        D & C & B
      \end{tabular}
    \end{center}
}

