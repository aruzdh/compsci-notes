\section{Autómatas Finitos No Deterministas}

\dfn{Autómata Finito No Determinista}{
  Un \textbf{Autómata Finito No Determinista (AFN)} es una \textit{5-tupla}
  \[M = (Q, \Sigma, \delta, q_0, F),\]
  donde
  \begin{itemize}
  \item \textit{Q} es un \textit{conjunto finito y no vacío de estados}.
    \item $\Sigma$ es el \textit{alfabeto} de entrada.
    \item $\delta$ es una \textit{función de transición} (función total) definida como
      \[\delta : Q \times \Sigma \longrightarrow \mathcal P(Q)\]
    \item $q_0$ es el \textit{estado inicial}.
    \item \textit{F} es el conjunto de \textit{estados finales} ($F \subseteq Q$).
  \end{itemize}
}

\nt{
  Cuando se hable de un \textit{AFD} y un \textit{AFN} se usará la notación $\delta_D$ y $\delta_N$ para denotar sus funciones de transición, respectivamente.
}

\ex{Autómata Finito Determinista}{
  Sea $\Sigma = \{0, 1\}$. Se busca construir un autómata que reconzca el lenguaje de cadenas sobre $\Sigma$ que tengan como prefijo \textit{000} \textbf{o} \textit{111}

  Un autómata que resuelve el problema anterior se puede ver de la siguiente manera.
  \begin{center}
      \begin{tikzpicture}[shorten >=1pt, node distance=2cm, on grid, auto]
          \tikzset{initial text={}}
          
          \node[state, initial] (qi) {$q_i$}; 
          \node[state] (q0) [above right=of qi] {$q_0$}; 
          \node[state] (q00) [right=of q0] {$q_{00}$}; 
          \node[state, accepting] (q000) [right=of q00] {$q_{000}$}; 

          \node[state] (q1) [below right=of qi] {$q_1$}; 
          \node[state] (q11) [right=of q1] {$q_{11}$}; 
          \node[state,accepting] (q111) [right=of q11] {$q_{111}$}; 
          \path[->]
            (qi) edge node {$0$} (q0)
            (q0) edge node {$0$} (q00)
            (q00) edge node {$0$} (q000)
            (qi) edge node {$1$} (q1)
            (q1) edge node {$1$} (q11)
            (q11) edge node {$1$} (q111);
          
          \node (start) [left=of qi] {};
          \draw[->] ([xshift=-.5cm] qi.west) -- (qi.west) node[midway, above] {};
      \end{tikzpicture}
  \end{center}

  La tabla de transiciones del autómata es la siguiente.
  \begin{center}
    \begin{tabular}{c | c | c}
    & 0 & 1 \\
    \hline
      *$\rightarrow$ $q_i$ & $q_0$ & $q_1$ \\
      $q_0$ & $q_{00}$ & $\varnothing$ \\
      $q_{00}$ & $q_{000}$ & $\varnothing$ \\
      $q_{000}$ & $q_{000}$ & $q_{000}$ \\
      $q_1$ & $\varnothing$ & $q_{11}$ \\
      $q_{11}$ & $\varnothing$ & $q_{111}$ \\
      $q_{111}$ & $q_{111}$ & $q_{111}$ \\
    \end{tabular}
  \end{center}
}

\nt{
  Ciertamente, el diagrama anterior tiene un abuso de notación, pero se mira lindo.
  Una versión más correcta está representada por su tabla de transiciones.
}

\subsection{Función extendida \texorpdfstring{$\delta^*$}{delta*}}

\dfn{Función extendida $\delta^*$}{
  Se define la \textbf{función extendida $\delta^*$} de la siguiente forma.
  \[ \delta^* : Q \times \Sigma^* \longrightarrow \mathcal P(Q)\]
  tal que
  \begin{itemize}
    \item $\delta^*(q, \varepsilon) = \{q\}$ 
    \item $\delta^*(q, wa) = \bigcup_{p \in \delta^*(q, w)}\delta(p,a)$
  \end{itemize}
}

\subsection{Lenguaje de un AFN}

\dfn{Lenguaje de un AFN}{
  Sea $M = (Q, \Sigma, \delta, q_0, F$ un \textit{AFN}, el \textbf{lenguaje aceptado} por \textit{M} se define como
  \[L(M) = \{w \in \Sigma^* \mid \delta^*(q_0,w) \cap F \neq \varnothing \}\]
}

\nt{
  Todo \textit{AFD} es a su vez un \textit{AFN} con la particularidad de que $\delta_D^*(p,w)$ consta de un \textbf{solo estado}, mientras que $\delta^*(p',w')$ es un \textbf{conjunto de estados}.
}

\newpage
\ex{AFN}{
    Sean $\Sigma = \{a,b\}$ y $L = \mathcal{L}(\alpha)$ donde $\alpha = (a + b)^*(aaa + bbb)(a + b)^*$.

    Se diseña un \textit{AFN $M$} que acepte a \textit{L}

    \begin{center}
      \begin{tikzpicture}[shorten >=1pt, node distance=2cm, on grid, auto]
        \tikzset{initial text={}}
        
        % INITIAL STATE
        \node[state, initial] (qi) {$q_i$}; 
        
        % STATES
        \node[state] (qa) [right=of qi] {$q_a$}; 
        \node[state] (qaa) [right=of qa] {$q_{aa}$}; 
        \node[state, accepting] (qf) [right=of qaa] {$q_{f}$}; 
        \node[state] (qb) [below=of qi] {$q_b$};
        \node[state] (qbb) [right=of qb] {$q_{bb}$};

        % TRANSITIONS
        \path[->]
            (qi) edge node {a} (qa) 
            (qi) edge node {b} (qb) 
            (qi) edge [loop above] node {a,b} (qi) 
            (qa) edge node {a} (qaa)
            (qaa) edge node {a} (qf)
            (qf) edge [loop above] node {a,b} (qf)
            (qb) edge node {b} (qbb)
            (qbb) edge node {b} (qf);
        
      \end{tikzpicture}
    \end{center}
    
    Ahora se muestra, usando $\delta^*$, la aceptación de la cadena $abbbb$\\
    Por un lado tenemos:
    \begin{align*}
      &\delta^*(q_i, abbbb) = \bigcup_{p \in \delta^*(q_i, abbb)} \delta(p,b) \\
      &\delta^*(q_i, abbb) = \bigcup_{p \in \delta^*(q_i, abb)} \delta(p,b)\\
      &\delta^*(q_i, abb) = \bigcup_{p \in \delta^*(q_i, ab)} \delta(p,b)\\
      &\delta^*(q_i, ab) = \bigcup_{p \in \delta^*(q_i, a)} \delta(p,b)\\
      &\delta^*(q_i, a) = \bigcup_{p \in \delta^*(q_i, \epsilon)} \delta(p,a)\\
    \end{align*}

    Sustituyendo de abajo hacia arriba:
    \begin{align*}
      &\delta^*(q_i, a) = \delta(qi,a) = \{q_i, q_a\}\\
      &\delta^*(q_i, ab) = \delta(q_i, b) \cup \delta(q_a,b) = \{q_i, q_b\}\\
      &\delta^*(q_i, abb) = \delta(q_i, b) \cup \delta(q_b,b) = \{q_i, q_b, q_{bb}\}\\
      &\delta^*(q_i, abbb) = \delta(q_i, b) \cup \delta(q_b, b) \cup \delta(q_{bb},b) = \{q_i, q_b, q_{bb},q_f\}\\
      &\delta^*(q_i, abbbb) = \delta(q_i,b) \cup \delta(q_b, b) \cup \delta(q_{bb},b) \cup \delta(q_f,b) = \{q_i, q_b, q_{bb}, q_f\}\\
    \end{align*}

    \textbf{Dado que hemos llegado al estado $q_f$ desde $q_i$, la cadena es aceptada.}
}

\newpage
\subsection{ELiminación del no-determinismo}

Es posible tener un \textit{AFN} y eliminar el \textit{no-determinismo} sin modificar el lenguaje aceptado.

\dfn{AFD a partir de un AFN}{
  Un \textit{AFD} \textbf{construido a partir} de un \textit{AFN} $M_N = (Q_N, \Sigma, \delta_N, q_0^N, F_N)$ es una \textit{5-tupla}
  \[M = (Q, \Sigma, \delta, q_0, F),\]
  donde
  \begin{itemize}
  \item $Q = \mathcal P(Q_N)$
    \item $\Sigma$ es el \textit{alfabeto} de entrada.
    \item $\delta(S,a)= \delta_N(S,a) =  \bigcup_{q \in S}\delta_N(q,a)$
    \item $q_0 = \{q_0^N\}$
    \item $F = \{S \subseteq Q \mid S \cup P \neq \varnothing\}$
  \end{itemize}
}

\nt {

  Dado un AFD $M_D$ construido a partir de un AFN $M_N$ usando la definición previa. Se tiene que \[L(M_D) = L(M_N)\]. Dicho de otra forma, $M_D$ y $M_N$ son \textit{autómatas equivalentes}.
}

\ex{AFD a partir de AFN}{
  Dado el autómata del ejemplo anterior. Se transforma \textit{M} a un \textit{AFD} mediante la construcción de subconjuntos.
  \begin{center}
    \begin{tabular}{c|c|c}
      \hline
      & a & b \\
      \hline
      $\rightarrow q_i$ & $q_i\,q_a$ & $q_i\,q_b$ \\
      $q_a$             & $q_{aa}$   & $\varnothing$ \\
      $q_{aa}$          & $q_f$      & $\varnothing$ \\
      $q_b$             & $\varnothing$ & $q_{bb}$ \\
      $q_{bb}$          & $\varnothing$ & $q_f$ \\
      $*q_f$             & $q_f$      & $q_f$ \\
      \hline
      $\,q_i\,q_a$     & $q_i\,q_a\,q_{aa}$ & $q_i\,q_b$ \\
      $\,q_i\,q_b$     & $q_i\,q_a$         & $q_i\,q_b\,q_{bb}$ \\
      $\,q_i\,q_a\,q_{aa}$ & $q_i\,q_a\,q_{aa}\,q_f$ & $q_i\,q_b$ \\
      $\,q_i\,q_b\,q_{bb}$ & $q_i\,q_a$         & $q_i\,q_b\,q_{bb}\,q_f$ \\
      $*\,q_i\,q_a\,q_{aa}\,q_f$    & $q_i\,q_a\,q_{aa}\,q_f$      & $q_i\,q_b\,q_f$ \\
      $*\,q_i\,q_b\,q_{bb}\,q_f$ & $q_i\,q_a\,q_f$ & $q_i\,q_b\,q_{bb}\,q_f$ \\
      $*\,q_i\,q_b\,q_f$ & $q_i\,q_a\,q_f$         & $q_i\,q_b\,q_{bb}\,q_f$ \\
      $*\,q_i\,q_a\,q_f$ & $q_i\,q_a\,q_{aa}\,q_f$         & $q_i\,q_b\,q_f$ \\
      \hline
    \end{tabular}
  \end{center}
}

\nt{
  Analizando la definición anterior junto con el ejemplo dado, se puede observar que \textit{no siempre} se necesita dar las transiciones para cada $S \in \mathcal P(Q)$. Notemos que parte de los estados son \textbf{inalcanzables}. Por tanto, no es necesario darlos.
}

\newpage
\ex{AFD a partir de un AFN}{
  Así, el diagrama del autómata para el ejemplo que venimos realizando queda de la siguiente manera
  \begin{center}
    \begin{tikzpicture}[->,>=stealth,node distance=3.2cm,on grid,auto]
      \node[state,initial] (qi) {$q_i$};
  
      \node[state] (qiqa)       [right=3cm of qi] {$q_i\,q_a$};
      \node[state] (qiqb)       [above=5cm of qiqa] {$q_i\,q_b$};
      \node[state] (qiqaqaa)    [right=3cm of qiqa] {$q_i\,q_a\,q_{aa}$};
      \node[state] (qiqbqbb)    [right=3cm of qiqb] {$q_i\,q_b\,q_{bb}$};
  
      \node[state,accepting] (qiqaqaaf) [right=3cm of qiqaqaa] {$q_i\,q_a\,q_{aa}\,q_f$};
      \node[state,accepting] (qiqbqbbf) [right=3cm of qiqbqbb] {$q_i\,q_b\,q_{bb}\,q_f$};
      \node[state,accepting] (qiqaqf)   [right=3cm of qiqbqbbf] {$q_i\,q_a\,q_f$};
      \node[state,accepting] (qiqbqf)   [right=3cm of qiqaqaaf] {$q_i\,q_b\,q_f$};
  
      \path
      (qi)   edge node {a} (qiqa)
              edge [bend left] node {b} (qiqb);
  
      \path
      (qiqa)       edge node {a} (qiqaqaa)
                    edge node {b} (qiqb)
      (qiqb)       edge [bend right=50] node {a} (qiqa)
                    edge node {b} (qiqbqbb)
      (qiqaqaa)    edge node {a} (qiqaqaaf)
                    edge [bend right] node {b} (qiqb)
      (qiqbqbb)    edge node {a} (qiqa)
                    edge node {b} (qiqbqbbf);
  
      \path
      (qiqaqaaf)   edge [loop below] node {a} (qiqaqaaf)
                    edge node {b} (qiqbqf)
      (qiqbqbbf)   edge node {a} (qiqaqf)
                    edge [loop above] node {b} (qiqbqbbqf)
      (qiqbqf)     edge [bend right=50] node {a} (qiqaqf)
                    edge [bend right] node {b} (qiqbqbbf)
      (qiqaqf)     edge node {a} (qiqaqaaf)
                    edge node {b} (qiqbqf);
    \end{tikzpicture}
  \end{center}
}
