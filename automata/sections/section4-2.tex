\section{Jerarquía de Chomsky}
Las gramáticas fueron clasificadas de acuerdo a sus propiedades por Chomsky.

\subsection{Tipo 0 - Lenguajes recursivamente enumerables}

Son aquellos lenguajes generados por una gramática sin restricciones adicionales. Estas gramáticas pueden incluir reglas de la forma 
\[\alpha \rightarrow \beta\]
Con esto, la gramática puede \textit{borrar} cadenas. Tales gramáticas son conocidas como \textbf{contraíbles}.

\ex{}{
  También la siguiente es una gramática de tipo 0, donde $\mathcal L(G) = \{ww \mid w \in \{0,1\}^*\}$

  \[S \rightarrow AT \qquad A \rightarrow 0A0 \qquad A \rightarrow 1A1 \qquad O0 \rightarrow 0O\]
  \[O1 \rightarrow 1O \qquad I0 \rightarrow 0I \qquad I1 \rightarrow 1I \qquad OT \rightarrow 0T\]
  \[IT \rightarrow 1T \qquad A \rightarrow \varepsilon \qquad T \rightarrow \varepsilon\]
}

\subsection{Tipo 1 - Lenguajes dependientes de contexto}
También son llamados \textit{sensibles al contexto}, y son aquellos generados por gramáticas con todas sus produciones de la forma
\[\alpha_1 A \alpha_2 \rightarrow \alpha_1 \beta \alpha_2\]
con $\alpha_1, \alpha_2 \in (V \cup T)^*,~ A \in V, \beta \neq \varepsilon$, adémas de que $| \alpha_1 \beta \alpha_2 | \ge |\alpha_1 A \alpha_2|$. Con la posible excepción de la regla $S \rightarrow \varepsilon$, en cuyo caso se prohibe la presencia de \textit{S} a la derecha de las producciones.

\vskip 0.2in

Lon autómatas que reconocen este tipo de lenguajes son los llamados \textit{autómatas acotados linealmente}.

\ex{gramática dependiente del contexto}{
\[L = \{a^ib^ic^i \mid i \ge 0\}\]
  \[S \rightarrow A \qquad A \rightarrow aABC \mid abC\]
  \[CB \rightarrow BC \qquad bB \rightarrow bb\]
  \[bC \rightarrow bc \qquad cC \rightarrow cc\]
}

\subsection{Tipo 2 - Lenguajes libres de contexto}
Son aquellos generados por gramáticas con todas sus producciones de la forma 
\[A \rightarrow a\]
con $A \in V,~ \alpha \in (V \cup T)^*$\\
Esta definiciíon incluye la regla $S \rightarrow \varepsilon$. Los autómatas que aceptan este tipo de lenguajes son llamados \textit{autómatas de pila}.

\subsection{Lenguajes regulares}
Aquellos generados por una gramática de una de las siguientes formas.

\begin{itemize}
  \item \textbf{Lineal por la derecha}: todas las producciones son de la forma
    \[A \rightarrow aB \qquad A \rightarrow a \qquad \varepsilon\]
    con $A,B \in V,~ a \in T$
  \item \textbf{Lineal por la izquierda}: todas las producciones de la forma
    \[A \rightarrow Ba \qquad A \rightarrow a \qquad A \rightarrow \varepsilon\]
    con $A,B \in V,~ a \in T$
  \item \textbf{No} se permite mezclar ambos tipos de producciones.
\end{itemize}
Lon \textit{autómatas finitos} son las que aceptan este tipo de lenguajes.

