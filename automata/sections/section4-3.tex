\section{Gramáticas Regulares}

\dfn{Gramática Regular}{
  Una \textbf{gramática regular} es una gramática lineal por la derecha o por la izquierda, donde no se permite una mezcla de ambos.
  \begin{itemize}
    \item \textbf{Lineal por la derecha}: todas las producciones son de la forma
      \[A \rightarrow aB \qquad A \rightarrow a \qquad \varepsilon\]
      con $A,B \in V,~ a \in T$
    \item \textbf{Lineal por la izquierda}: todas las producciones de la forma
      \[A \rightarrow Ba \qquad A \rightarrow a \qquad A \rightarrow \varepsilon\]
      con $A,B \in V,~ a \in T$
  \end{itemize}
}

\dfn{Lenguaje regular}{
  Decimos que un lenguaje \textit{L} es regular si \textit{existe una gramática regular G que lo genere}. Es decir, si $L = \mathcal L(G)$
}

\ex{Lenguaje regular}{
  El lenguaje $L = (a + b)^*$ es generade por la siguiente gramática.
  \begin{align*}
    &S \rightarrow aS \mid bC\\
    &C \rightarrow bC \mid aS \mid \varepsilon
  \end{align*}
}

\subsection{Equivalencia entre gramáticas y lenguajes regulares}
Esta equivalencia involucra a los \textit{autómatas finitos}.

\begin{enumerate}
  \item Autómata Finito $\Longrightarrow$ Gramática regular.

    Dado un autómata finito \textit{M}, existe una gramática regular \textit{G} tal que $\mathcal L(M) = \mathcal L(G)$. Asî, todo lenguaje regular es generado por una gramática regular.

    \vskip 0.2in
    Lo construción de una gramática regular \textit{G} dado un autómata finito \textit{M} es de la siguiente manera.

    \begin{itemize}
      \item Suponemos, \textit{sin pérdida de generalidad}, que no hay $\varepsilon$-transciones en \textit{M} (\textit{M} puede ser \textit{no-determinista}).
      \item Se definen las partes de \textit{G}, $V = Q,~ T = \Sigma, ~ S = q_0$.
      \item Las reglas de producción se definen de la siguiente forma.
        \begin{itemize}
          \item Si $p \in \delta(q,a)$, entonces agregamos $q \rightarrow ap$ a \textit{P}.
          \item Si $q_f \in \delta(q,a)$, con $q_f \in F$, entonces agregamos $q \rightarrow a$.
          \item Si $q_0 \in F$, entonces agregamos $q_0 \rightarrow \varepsilon$
        \end{itemize}
    \end{itemize}

  \item Gramática Regular $\Longrightarrow$ Autómata Finito

    Dada una gramática regular \textit{G}, existe un autómata finito \textit{M} tal que $\mathcal L(M) = \mathcal L(G)$. Así, todo lenguaje generado por una gramática regular es regular.

    \vskip 0.2in
    Definimos al autómata \textit{M} siguiendo lo siguiente.

    \begin{itemize}
      \item Suponemos, \textit{sin pérdida de generalidad}, que \textit{G} es lineal derecha.
      \item Se definen los siguientes conjuntos: $\Sigma = T,~ Q = V \cup \{q_F\}, ~ q_0 = S,~ F = \{q_F\}$
      \item La función de transición es la siguiente.
        \begin{itemize}
          \item Si $A \rightarrow aB \in P$, entonces $B \in \delta(A,a)$
          \item Si $A \rightarrow a \in P$, entonces $q_F \in \delta(A,a)$
          \item Si $A \rightarrow \varepsilon \in P$, entonces $q_F \in \delta(A,\varepsilon)$
        \end{itemize}
    \end{itemize}
\end{enumerate}

\ex{Equivalencia}{
  Considerando el siguiente autómata.
  \begin{center}
      \begin{tikzpicture}[shorten >=1pt, node distance=2.5cm, on grid, auto]
          \tikzset{initial text={}}
          
          % Estados
          \node[state, initial] (q0) {$q_0$}; 
          \node[state] (q1) [right=of q0] {$q_1$}; 
          \node[state, accepting] (q2) [below=of q0] {$q_2$};
          
          % Transiciones
          \path[->]
              (q0) edge [loop above] node {0} (q0)
              (q1) edge [loop above] node {1} (q1)
              (q0) edge node {1} (q1)
              (q0) edge node [left] {0} (q2)
              (q2) edge [bend left] node {1} (q1)
              (q1) edge [bend left] node {0} (q2);
      \end{tikzpicture}
  \end{center}
  La gramática correspondiente obtenida con el método antorior es la siguiente.
  \begin{align*}
    &q_0 \rightarrow 0q_0 \mid 1q_1\\
    &q_1 \rightarrow 1q_1 \mid 0q_2 \mid 0\\
    &q_2 \rightarrow 0q_0 \mid 1q_1
  \end{align*}
}

\subsection{Equivalencia entre gramáticas lineales}
Se puede probar que toda gramática lineal por la izquierda es equivalente a una gramática lineal por la derecha. Veamos cómo convertir gramáticas lineales por la izquierda a lineales por la derecha y viceversa, para ello utilizamos las transformaciones a autómatas finitos.

\begin{enumerate}
    \item Dada una gramática lineal por la derecha se debe construir un autómata finito que reconozca al mismo lenguaje.
    \item Para construir la gramática lineal por la izquierda se debe obtener un autómata dual:
    \begin{itemize}
        \item intercambiar los estados inicial y final
        \item invertir las transiciones
    \end{itemize}
    \item Obtener una nueva gramática del autómata nuevo y después reordenar las reglas de producción $A \rightarrow aB$ en $A \rightarrow Ba$.
\end{enumerate}

El método anterior se puede usar cuando el autómata tiene un sólo estado final. Para la transformación de lineal izquierda a lineal derecha se siguen los mismos pasos que antes.

