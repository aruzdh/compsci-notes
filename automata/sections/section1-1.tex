\section{Cadenas}
\subsection{Conceptos Básicos}
\dfn{Símbolo}{
  Un \textbf{símbolo} o \textbf{caracter} es una entidad indivisible.
}

\ex{Símbolos}{
  a, b, c, 1, 0, z, @, *
}

\dfn{Alfabeto}{
  Un \textbf{alfabeto} es un conjunto \textit{finito} y \textit{no vacío} de símbolos. Normalmente es denotado por $\Sigma$.
}

\ex{Alfabetos}{
  \begin{itemize}
    \item $\Sigma_1 = \{0,1\}$
    \item $\Sigma_2 = \{a,b,c\}$
    \item $\Sigma_3 = \{a,b,c\}$
    \item $\Sigma_4 = \{0,1,\cdots, 9\}$
  \end{itemize}
}

\ex{Alfabetos}{
  El codigo \textit{ASCII} es un alfabeto.
}

\dfn{Cadena}{
  Una \textbf{cadena} es una sucesión finita de símbolos de un alfabeto.
}

\ex{Cadenas}{
  Sea $\Sigma = \{0, 1\}$ un alfabeto. Las siguientes son cadenas sobre $\Sigma$: \textit{0,1, 00, 10, 101010, 101010100}
}

\dfn{Cadena vacía}{
Denotamos por $\varepsilon$ a la \textbf{cadena vacía.}
}

\nt{
  $\varepsilon$ us una cadena de \textbf{cualquier alfabeto}
}

\dfn{Estrella de Klenne}{
  El \textit{conjunto de todas las cadenas} sobre un alfabeto $\Sigma$ es infinito y se denota por $\Sigma^*$, llamado \textbf{Estrella de Klenne}.
}

\ex{Estrella de Klenne}{
  Sea $\Sigma = \{a\}$. Entonces $\Sigma^* = \{a_i\cdots a_n \mid n \in \mathbb N\}$
}

\subsection{Operaciones de cadenas}

\dfn{Longitud}{
  Sea $w = a_i\cdots a_n$ una cadena. La \textbf{longitud} de \textit{w}, denotada por $\mid w \mid$, es el número de símbolos que la forman.
}

\ex{Longitud}{
  Sea $w = 010101$. Su longitud es $\mid w \mid = \mid 010101 \mid = 6$
}

\ex{Longitud}{
  Sea $w = \varepsilon$. Su longitud es $\mid \varepsilon \mid = 0$
}

\dfn{Concatenación}{
  Sea $w = a_i\cdots a_n,~ x = b_1 \cdots b_m$ cadenas. La \textbf{concatenación} de \textit{w} con \textit{x} es la cadena que resulta de pegar \textit{w} con \textit{x}. Es decir \[wx = a_1a_2\cdots a_n b_1b_2 \cdots b_m\]
}

\nt{
  Se usará la notación $w \cdot x$ y $wx$ de manera indistinta para denotar la concatenación de dos cadenas.
}

\ex{Concatenación}{
  Sea $w = bronquio, ~ x = saurio$ cadenas. La concatenación de \textit{w} con \textit{x} es $wx = bronquiosaurio$.
}

Las siguientes son propiedades de la concatenación de cadenas.
\begin{itemize}
  \item Asociatividad \[(xy) z = x(yz)\]
  \item Elemento identidad \[w\varepsilon = w = \varepsilon w\]
  \item Longitud \[\mid wx\mid = \mid w \mid + \mid x \mid \]
  \item Potencia \[w^n = w \cdot w \cdot w \cdots w,~ \text{ n veces}\]
\end{itemize}
\nt {
  Sea \textit{a} un símbolo arbitrario. Se tiene que $a^0 = \varepsilon$.
}
\nt{
  La conmutatividad \textbf{no} es una propiedad de la concatenación.
}

\dfn{Reversa}{
  Sea $w = a_1a_2\cdots a_n$ una cadena. La \textbf{reversa} de \textit{w}, denotada por $w^R$, es la cadena que resulta de invertir los símbolos de \textit{w}. Es decir, \[w^R = a_na_{n-1}\cdots a_2a_1\]
}

\ex{Reversa}{
  Sea $w = carlos$. Se tiene que $w^R = solrac$.
}
Las siguientes son propiedades de la reversa de cadenas.
\begin{itemize}
  \item $(w^R)^R = w$
  \item $(wx)^R = x^R w^R$
\end{itemize}

\subsection{Subcadenas, prefijos y sufijos}

\dfn{Subcadenas}{
  Sea \textit{w} una cadena sobre $\Sigma$ tal que $w = xyz$ donde $x,y,z \in \Sigma^*$. Entonces \textit{x,y,z} son \textbf{subcadenas} de \textit{w}.
}
\nt{
  Dada \textit{w} una cadena arbitraria. Una cadena de \textit{x} es subcadena de \textit{w} \textbf{si y solo si} \textit{x} está formada de símbolos \textbf{contiguos} de \textit{w}.
}
\nt{
  Dada \textit{w} una cadena arbitraria. Se tiene que $\varepsilon$ es una subcadena de \textit{w}.
}

\ex{SUbcadenas}{
  Sea $w = patosaurio$. Una descomposición en subcadenas de \textit{w} es la siguiente: $x = pato, y = sau,~ z = rio$. Por otra lado, $n = pario$ \textbf{no} es subcadena.
}

\dfn{Prefijos}{
  Sea \textit{w} una cadena tal que $w = xyz$ donde $x,y,z \in \Sigma^*$. Se tiene que \textit{x} es un \textbf{prefijo} de \textit{w}. Es decir, \textit{x} es una subcadena que contiene el inicio de \textit{w} (incluyendo a $\varepsilon$).
}

\dfn{Prefijos propios}{
  Sea \textit{w} una cadena. Todo prefijo \textit{x} de \textit{w} es un \textbf{prefijo propio} si $x \neq w$
}

\ex{Prefijos}{
  Sea $w = patosaurio$. Se tienen los siguientes prefijos.
  \begin{itemize}
    \item $\varepsilon$
    \item p
    \item pa
    \item pat
    \item pato
    \item patos
    \item patosa
    \item patosau
    \item patosaur
    \item patosauri
    \item patosaurio
  \end{itemize}
  Donde el último prefijo listado (\textit{x = patosaurio = w}) no es un prefijo propio.
}

\dfn{Sufijos}{
  Sea \textit{w} una cadena tal que $w = xyz$ donde $x,y,z \in \Sigma^*$. Se tiene que \textit{z} es un \textbf{sufijo} de \textit{w}. Es decir, \textit{z} es una subcadena que contiene el final de \textit{w} (incluyendo a $\varepsilon$).
}

\dfn{Prefijos propios}{
  Sea \textit{w} una cadena. Todo sufijo \textit{z} de \textit{w} es un \textbf{sufijo propio} si $z \neq w$
}

\ex{Prefijos}{
  Sea $w = patosaurio$. Se tienen los siguientes prefijos.
  \begin{itemize}
    \item $\varepsilon$
    \item o
    \item io
    \item rio
    \item urio
    \item aurio
    \item saurio
    \item osaurio
    \item tosaurio
    \item atosaurio
    \item patosaurio

  \end{itemize}
  Donde el último sufijo listado (\textit{z = patosaurio = w}) no es un sufijo propio.
}

\subsection{Definiciones Recursivas}

\dfn{Def. Recursiva de Estrella de Klenne}{
  Dado un alfabeto $\Sigma$. Definimos $\Sigma^*$ de la siguient manera.
  \begin{itemize}
    \item $\varepsilon \in \Sigma^*$
    \item Sea $a \in \Sigma$ y $w \in \Sigma^*$. Entonces $wa \in \Sigma^*$
    \item Solo las anteriores son cadenas de $\Sigma^*$
  \end{itemize}
}

\dfn{Def. Recursiva de la Concatenación}{
  \begin{itemize}
    \item $w \varepsilon = w$
    \item $w(xa) = (wx)a$
  \end{itemize}
}

\ex{Concatenación Recursiva}{
  Sea $x = 01$ y $y = 210$. Se tiene que
  \begin{align*}
    (01)\cdot(210) &= (01\cdot 21)0\\
               &= ((01 \cdot 2)1)0\\
               &= (((01 \cdot \varepsilon)2)1)0\\
               &= (((01)2)1)0\\
               &= ((012)1)0\\
               &= (0121)0\\
               &= 01210\\
  \end{align*}
}

\dfn{Def. Recursiva de la Longitud}{
  \begin{itemize}
    \item $\mid \varepsilon \mid = 0$
    \item $\mid w a \mid = \mid w \mid + 1$
  \end{itemize}
}

\ex{Longitud Recursiva}{
  Sea $x = 010$. Se tiene que 
  \begin{align*}
    \mid x \mid = \mid 010 \mid &= \mid 01 \mid + 1\\
                    &= (\mid 0 \mid + 1) + 1\\
                    &= ((\mid \varepsilon \mid + 1) + 1) + 1\\
                    &= ((0 + 1) + 1) + 1\\
                    &= (1 + 1) + 1\\
                    &= 2 + 1\\
                    &= 3\\
  \end{align*}
}

\dfn{Def. Recursiva de Reversa}{
  \begin{itemize}
    \item $\varepsilon^R = \varepsilon$
    \item $(wa)^R = a \cdot w^R$
  \end{itemize}
}

\ex{Reversa Recursiva}{
  Sea $w = 012$. Se tiene que
  \begin{align*}
    w^R = (012)^R &= 2 \cdot (01)^R\\
                &= 2 \cdot (1 \cdot 0^R)\\
                &= 2 \cdot (1 \cdot (0 \cdot \varepsilon^R))\\
                &= 2 \cdot (1 \cdot (0 \cdot \varepsilon))\\
                &= 2 \cdot (1 \cdot (0))\\
                &= 2 \cdot (10)\\
                &= 210\\
  \end{align*}
}

\subsection{Demostraciones por Inducción}
Las propiedades enunciadas en las secciones anteriores pueden ser demostradas mediante el \textit{Principo de Inducción}.

\dfn{Principo de Inducción}{
  Para demostrar que una propiedad \textit{P} se cumple para toda $w \in \Sigma^*$, es decir, \textit{P(w)}, se procede de la siguiente forma.
  \begin{enumerate}[a)]
    \item Caso Base. Se demuestra $P(\varepsilon)$
    \item Hipotesis Inductiva. Se toma $w \in \Sigma^*$ arbitraria y se supone \textit{P(w)}.
    \item Paso Inductivo. Se toma $a \in \Sigma,~ w \in \Sigma^*$ y se demuestra \textit{P(wa)} \textbf{usando la Hipotesis Inductiva} del paso anterior.
  \end{enumerate}
  Hecho lo anterior, se concluye que \[\forall w \in \Sigma^*,~ P(w)\]
}

\nt{
  Sea \textit{w} una cadena arbitraria. Se usa de manera indistinta \textit{P(w)} y \textit{P(w) verdadera} para indicar que \textit{w} cumple la propiedad \textit{P}.
} 
