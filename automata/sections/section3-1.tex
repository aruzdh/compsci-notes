\section{Autómatas Finitos Deterministas}

\dfn{Autómata Finito Determinista}{
  Un \textbf{Autómata Finito Determinista (AFD)} es una \textit{5-tupla}
  \[M = (Q, \Sigma, \delta, q_0, F),\]
  donde
  \begin{itemize}
  \item \textit{Q} es un \textit{conjunto finito y no vacío de estados}.
    \item $\Sigma$ es el \textit{alfabeto} de entrada.
    \item $\delta$ es una \textit{función de transición} (función total) definida como
      \[\delta : Q \times \Sigma \longrightarrow Q\]
    \item $q_0$ es el \textit{estado inicial}.
    \item \textit{F} es el conjunto de \textit{estados finales} ($F \subseteq Q$).
  \end{itemize}
}

Dado un \textit{Autómata Finito (AF)}, podemos representarlo de las siguientes formas.
\begin{enumerate}[a)]
  \item Formal: defininendo $\delta$
  \item Diagrama de transiciones.
  \item Tabla de transiciones.
\end{enumerate}

\ex{Autómata Finito Determinista}{
  Sea $\Sigma = \{0, 1\}$, y $r = ((0 + 1)(0 + 1))^*$ la expresión regular tal que $L(r)$ es el conjunto de cadenas de longitud par sobre $\Sigma$.

  El Diagrama de transiciones un autómata que representa a \textit{r} es
  \begin{center}
      \begin{tikzpicture}[shorten >=1pt, node distance=2cm, on grid, auto]
          \tikzset{initial text={}}
          
          \node[state, initial, accepting] (q0) {$q_0$}; 
          \node[state] (q1) [right=of q0] {$q_1$}; 

          \path[->]
            (q0) edge [bend left] node {$0,1$} (q1)
            (q1) edge [bend left] node {$0,1$} (q0);
          
          \node (start) [left=of q0] {};
          \draw[->] ([xshift=-.5cm] q0.west) -- (q0.west) node[midway, above] {};
      \end{tikzpicture}
  \end{center}
}
\nt{
  Dado un \textit{AFD} mediante su diagrama. Se tiene que \textbf{solo existe una transición por cada símbolo del alfabeto}.
}
\newpage
\nt{
  Dado un \textit{AF} mediante su diagrama. Si alguna transición no está difinida explicitamenete (no hay "flecha" de un estado a otro para algún símbolo), se da por entendido que si el AF lee el símbolo correspondiente a la transición, este se detiene y no acepta la cadena actual.
}

Dado el autómata del ejemplo anterior, su tabla de transiciones es la siguiente.
\begin{table}[h]
\centering
  \begin{tabular}{c | c | c}
  & 0 & 1 \\
  \hline
    *$\rightarrow$ $q_0$ & $q_1$ & $q_1$ \\
    $q_1 $& $q_0$ & $q_0$ \\
  \end{tabular}
\end{table}

\nt{
  Cuando damos la tabla de transiciones de un \textit{AF}, usamos \textit{*} para denotar los \textit{estados finales} del autómata, y $\rightarrow$ para identificar al \textit{estado inicial}.
}

\subsection{Función extendida \texorpdfstring{$\delta^*$}{delta*}}

\dfn{Función extendida $\delta^*$}{
  Se define la \textbf{función extendida $\delta^*$} de la siguiente forma.
  \[ \delta^* : Q \times \Sigma^* \longrightarrow Q\]
  tal que
  \begin{itemize}
    \item $\delta^*(q, \varepsilon) = q$ 
    \item $\delta^*(q, wa) = \delta(\delta^*(w),a)$
  \end{itemize}
}

\subsection{Lenguaje de un AFD}

\dfn{Lenguaje de un AFD}{
  Sea $M = (Q, \Sigma, \delta, q_0, F$ un \textit{AFD}, el \textbf{lenguaje aceptado} por \textit{M} se define como
  \[L(M) = \{w \in \Sigma^* \mid \delta^*(q_0,w) \in F\}\]
}

Definido lo anterior, podemos procesar \textit{cadenas} y saber si una cadena es aceptada o no.

\ex{Función extendida $\delta^*$}{
  Consideremos el autómata del ejemplo pasado, y $w = 101$ un cadena. Se tiene que
  \begin{align*}
    &\delta^*(q_0, 101) = \delta(\delta^*(q_0, 10),1)\\
    &\delta^*(q_0, 10) = \delta(\delta^*(q_0, 1),0)\\
    &\delta^*(q_0, 1) = \delta(\delta^*(q_0, \varepsilon),1)\\
    &\delta^*(q_0, \varepsilon) = q_0\\
  \end{align*}

  Sustituyendo en sentido contrario obtenemos
  \begin{align*}
    &\delta^*(q_0, 1) = \delta(q_0,1) = q_1\\
    &\delta^*(q_0, 10) = \delta(q_1,0) = q_0\\
    &\delta^*(q_0, 101) = \delta(q_0,1) = q_1\\
  \end{align*}
  Dado que el autómata terminó en el estado $q_1$ y este \textbf{no} es un estado final, se concluye que \textit{101} no pertenece al lenguaje del autómata, y por tanto, la cadena no es aceptada.
}

\subsection{Diseño de AFD por Complemento}

Dado un lenguaje \textit{L} a veces es más facil diseñar un autómata para el complemento $\overline L = \Sigma^* - L$. Entonces, si $M = (Q, \Sigma, \delta, q_0, F)$ reconoce a \textit{L}, es decir $L(M) = L$, entonces $M^c = (Q, \Sigma, \delta, q_0, Q - F)$ reconoce a $\Sigma^* - L$, es decir $L(M^c) = \overline L$

\nt{
  El método anterior solo funciona para \textit{AFD}
}

\ex{Diseño por Complemento}{
  Sea $\Sigma = \{0,1\}$. Se busca crear un  \textit{AFD} para el lenguaje de cadenas sobre $\Sigma$ que \textbf{no} terminen en \textit{11}. Siguiendo el método descrito anteriormente se tiene que

  \begin{center}
      \begin{tikzpicture}[shorten >=1pt, node distance=2cm, on grid, auto]
          \tikzset{initial text={}}
          
          \node[state, initial] (q0) {$q_0$}; 
          \node[state] (q1) [right=of q0] {$q_1$}; 
          \node[state, accepting] (q11) [right=of q1] {$q_{11}$}; 

          \path[->]
            (q0) edge [bend left] node {1} (q1)
            (q0) edge [loop above] node {0} (q0)
            (q1) edge [bend left] node {0} (q0)
            (q11) edge [bend left=50] node {0} (q0)
            (q1) edge node {$1$} (q11);
          
          \node (start) [left=of q0] {};
          \draw[->] ([xshift=-.5cm] q0.west) -- (q0.west) node[midway, above] {};
      \end{tikzpicture}
  \end{center}
  este autómata acepta las cadenas que \textbf{si} terminan en \textit{11}.
  Por tanto, el complemento del autómata cumple con el requisito inicial
  \begin{center}
      \begin{tikzpicture}[shorten >=1pt, node distance=2cm, on grid, auto]
          \tikzset{initial text={}}
          
          \node[state, initial, accepting] (q0) {$q_0$}; 
          \node[state, accepting] (q1) [right=of q0] {$q_1$};
          \node[state] (q11) [right=of q1] {$q_{11}$}; 

          \path[->]
            (q0) edge [bend left] node {1} (q1)
            (q0) edge [loop above] node {0} (q0)
            (q1) edge [bend left] node {0} (q0)
            (q11) edge [bend left=50] node {0} (q0)
            (q1) edge node {$1$} (q11);
          
          \node (start) [left=of q0] {};
          \draw[->] ([xshift=-.5cm] q0.west) -- (q0.west) node[midway, above] {};
      \end{tikzpicture}
  \end{center}
  Notemos que el único cambio es el conjuto de \textit{estados finales}.
}

\nt{
  Si el \textit{AFD} no tiene todas sus transiciones de forma explicita, es casi seguro que para su complemento serán necesarias.
}

\newpage
\subsection{Modularización}
A veces la especificación de un lenguaje tiene una forma lógica que permite descomponerlo en lenguajes más sencillos.

\dfn{Modularización}{
  Sea $M_1 = (Q_1, \Sigma, \delta_1, q_0, F_1)$ y $M_2 = (Q_2, \Sigma, \delta_2, p_0, F_2)$ dos autómatas que aceptan los lenguajes $L_1$ y $L_2$, respectivamente. Sea $M = (Q, \Sigma, \delta, r_0, F)$ un autómata donde $Q = Q_1 \times Q_2$, el estado inicial es $r_0 = (q_0, p_0)$, la función de transición se define como
  \[\delta((q,p), a) = (\delta_1(q,a), \delta_2(p,a))\]
  y los estados finales de la siguiente manera.

  \begin{itemize}
    \item Si $F = \{(q,p) \mid q \in F_1 \lor p \in F_2\}$ entonces \textit{M} acepta el lenguajes $L_1 \cup L_2$
    \item Si $F = \{(q,p) \mid q \in F_1 \land p \in F_2\}$ entonces \textit{M} acepta el lenguajes $L_1 \cap L_2$
    \item Si $F = \{(q,p) \mid q \in F_1 \land p \notin F_2\}$ entonces \textit{M} acepta el lenguajes $L_1 - L_2$
  \end{itemize}
}
