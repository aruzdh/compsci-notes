
\documentclass{article}
\usepackage[utf8]{inputenc}
\usepackage{amssymb}
\usepackage{enumitem}
\usepackage[fleqn]{amsmath}
\usepackage[top=2cm, bottom=2cm, left=2.5cm, right=2.5cm]{geometry}

\title{Autom\'atas y Lenguajes Formales 25-2 \\ Tarea 2: Expresiones Regulares}
\author{Hernández Vázquez Carlos Arturo }
\date{Martes 25 de Febrero del 2025}

\begin{document}

\maketitle

\begin{enumerate}
    \item (1pt) A continuación se presenta una definicion recursiva de un subconjunto de $\{a, b\}^*$. Proporcione una descripción con palabras adem\'as de la expresión regular que define a tal lenguaje.
    \begin{enumerate}[label=\roman*.]
        \renewcommand{\labelenumi}{\Roman{enumi}}
        \item $a \in L$
        \item Para cada $x \in L,~ ax \in L$ y $xb \in L$
        \item Son todas
    \end{enumerate}
    \begin{itemize}[label=-]
        \item Este lenguaje, en primer lugar, cuenta con $a$ como cadena. Por ii) sabemos que podemos construir cualquier sucesión de $a$'s que termine en $a$ o en $b$, con la condición de que las $b$'s siempre estén juntas al final de la cadena. Dicho de otra manera: siempre que concatenamos por el lado izquierdo será una $a$, mientras que por la derecha será una $b$.
        Adem\'as, $\varepsilon$ no pertenece a $L$. \\ Ej: $a, aa, aaa, ab, aab ,abb, aabb, aabbbb, ...$
        \item Expresi\'on regular: $aa^*b^*$
    \end{itemize}

    \item (5pts) Considere $\Sigma = \{a, b, c\}$, para cada uno de los siguientes lenguajes proporcione una expresión regular que genere el mismo lenguaje, primero trate de dar una descripción por $extension$ del lenguaje:
    \begin{enumerate}[label=\alph*)] % Letras minúsculas con paréntesis
        \item $L = \{w \in \{a,b\}^* \mid w~ \mbox{no tiene la subcadena $bb$ y termino en $a$}\}$
        \begin{itemize}
            \item $L =\{a, ba, baa, aba, baba, babaa, ababa, aaba, abaa, baaba, abaaba, abababa, aababababaa, ...\}$
            \item $(\varepsilon + b)(aa^*(\varepsilon + b))^*a$
        \end{itemize}
        \item $L = \{w \in \Sigma^* \mid w~ \mbox{tiene a lo m\'as tres $a$ consecutivas}\}$
        \begin{itemize}
            \item $L = \{\varepsilon, a, b, c, aa, ab, ac, ba, ca, aaa, abc, bca, cba, aaab, baa, cbaaabcabaccccccaaab, ...\}$
            \item $((\varepsilon + a)(b + c)^*(\varepsilon + a)(b + c)^*(\varepsilon + a)(b + c)(b + c)^*)^*(\varepsilon + a + aa + aaa)$
        \end{itemize}
        \item $L = \{w \in \Sigma^* \mid \#_b(w) = 4n + 1, n \in \mathbb{N} \}$
        \begin{itemize}
            \item $L = \{b, ab, cb, abc, bca, cba, bbbbbbbbb, acbbbabcbaabbbcb, \}$
            \item $(a + c)^*(b(a + c)^*b(a+c)^*b(a + c)^*b(a + c)^*)^*b(a + c)^*$ 
        \end{itemize}
        \item $L = \{a^nb^m \mid n \ge 4, m \le 3; ~n,m \in \mathbb{N} \}$
        \begin{itemize}
            \item $L = \{aaaa, aaaab, aaaabb, aaaabbb, aaaaa, aaaaab, aaaaabb, aaaaabbb, aaaaaa, ...\}$
            \item $(aaaaa^*)(\varepsilon + b +bb+bbb)$
        \end{itemize}
        \item $L = \{w \in \Sigma^* \mid |w| = 3n, ~ n \in \mathbb{N}\}$
        \begin{itemize}
            \item $L = \{\varepsilon, aaa, bbb, ccc, abc, bca, cba, acb, aaaaaa, bbbbbb, cccccc, abcabc, bcabca, cbacba, acbacb, aaabac, ...\}$
            \item $((a+b+c)(a + b + c)(a + b +c))^*$
        \end{itemize}
        \item $L = \{w \in \Sigma^* \mid abc~ \mbox{no es subcadena de } w\}$
        \begin{itemize}
            \item $L = \{\varepsilon, a, b, c, aa, ab, ac, ba, bb, bc, ca, cb, cc, acb, cba, cab, bac, bca, abac, cabb, aabbcc, ...\}$
            \item $((b + c)+ a((a +c) + b((a + b)(a + b)^*))^*)^*$
        \end{itemize}
        \item $L = \{uwu \in \Sigma^* \mid |u| = 2\}$
        \begin{itemize}
            \item $L = \{aaa, aaaa, bbb, bbbb, ccc, cccc, acaaaac, bbcacbdabbb, ...\}$
            \item $aa (a + b + c)^* aa + ab (a + b + c)^* ab + ac (a + b + c)^* ac + ba (a + b + c)^* ba + bb (a + b + c)^* bb + bc (a + b + c)^* bc + ca (a + b + c)^* ca + cb (a + b + c)^* cb + cc (a + b + c)^* cc$
        \end{itemize}
        \item $L = \{a^nb^m \mid n,m \ge 1, n \cdot m \ge 3 \}$
        \begin{itemize}
            \item $L = \{aaab,abbb, aabbbb, aaabbb, ... \}$
            \item $(aaaa^*bb^*) + (aa^*bbbb^*)$
        \end{itemize}
        \item $L = \{ab^nw \mid n \ge 3 ~\mbox{y } w ~\mbox{empieza con } aba\}$
        \begin{itemize}
            \item $L = \{abbbaba, abbbabaa, abbbbbababcc, abbbbabaababbab, ...\}$
            \item $abbbb^*aba(a + b + c)^*$
        \end{itemize}
        \item $L = \{a^n \mid n = 3 + 2j; ~n, j \in \mathbb{N} \}$
        \begin{itemize}
            \item $L=\{aaa, aaaaa, aaaaaaa, a^3a^{2j},...\}$
            \item $aaa(aa)^*$
        \end{itemize}
    \end{enumerate}
    
    \item (1pt) Simplifique lo m\'as posible las siguientes expresiones regulares mediante equivalencias y describa con palabras el lenguaje correspondiente. Se puede suponer que $\Sigma = \{a,b\}$
    \begin{enumerate}[label=\alph*)]
        \item $a(\varepsilon + (b + aa)(aa + b)^*) + a(b + aa)^*$
        \begin{align*}
            a(\varepsilon + (b + aa)(aa + b)^*) + a(b + aa)^* &= a(\varepsilon + (b + aa)(b + aa)^*) + a(b + aa)^* \mbox{--- por conmutatividad de +}\\
                          &= a(b + aa)^* + a(b + aa)^* \mbox{--- por $r^* = \varepsilon + rr^*$} \\
                          &= a((b + aa)^* + (b + aa)^*) \mbox{--- por distributividad} \\ 
                          &= a(b + aa)^* \mbox{--- por $r = r + r$} \\ 
        \end{align*}
        Son las cadenas que siempre tienen como prefijo una $a$, seguida de la posibilidad de tener $b^n$ ($n \in \mathbb{N}$), o pares de $a$'s. Esto ultimo puede darse en cualquier orden gracias a la unión entre estas dos subexpresiones \\
        Ej: $a, ab, aaa, abaa, aaabbaabbb, ...$
        \item $(\varepsilon + (aa + b)(b + aa)^*)a + (aa + b)^*a$
        \begin{align*}
            (\varepsilon + (aa + b)(b + aa)^*)a + (aa + b)^*a &= (\varepsilon + (aa + b)(aa + b)^*)a + (aa + b)^*a \mbox{--- por conmutatividad de +}\\
                          &= (aa + b)^*a + (aa + b)^*a \mbox{--- por $r^* = \varepsilon + rr^*$} \\
                          &= ((aa + b)^* + (aa + b)^*)a \mbox{--- por distributividad} \\
                          &= (aa + b)^*a \mbox{--- por $r = r + r$} \\
        \end{align*}
        Son las cadenas que tienen como sufijo a $a$. Adem\'as tenemos pares de $a$'s o cualquier cantidad de $b$'s (en cualquier orden debido a la union y a *) antes del sufijo mencionado. \\
        Ej: $a, ba, aaa, aabaa, bba, baaaabaaa, aabbaaba, ...$
        
    \end{enumerate}
    
    \item (1pt) Escriba una definición recursiva de una función $rev$ tal que $rev(\alpha)$ devuelve la expresión regular correspondiente $L(\alpha)^R$.
    
    Sean $a,\lambda, s$ expresiones
    \begin{enumerate}[label=\roman*.]
        \item $rev(\varnothing) = \varnothing$
        \item $rev(\varepsilon) = \varepsilon$
        \item $rev(a) = a$
        \item $rev((\lambda)^*) = (rev(\lambda))^*$
        \item $rev((\lambda + \beta)) = (rev(\lambda) + rev(\beta))$
        \item $rev((\lambda \cdot \beta)) = (rev(\beta) \cdot rev(\lambda))$
        \item $rev(\lambda s) = rev(s) \cdot rev(\lambda)$
    \end{enumerate}
    
    \item (0.5pts) Para la siguiente expresión regular $\alpha$ escribe una expresión regular para la reversa del lenguaje $L(\alpha)$ usando la definición proporcionada en el ejercicio anterior.
    \begin{enumerate}[label=\alph*)]
        \item $\alpha = (aab + bbaba)^*baba$
    \end{enumerate}
    \begin{align*}
        rev((aab + bbaba)^*baba)&= rev(baba)rev((aab + bbaba)^*) \mbox{ --- def. vii}\\
          &= rev(baba)(rev(aab + bbaba))^* \mbox{ --- def. iv} \\
          &= rev(baba)(rev(aab) + rev(bbaba))^* \mbox{ --- def. v} \\
          &= rev(a)rev(bab)(rev(aab) + rev(bbaba))^* \mbox{ --- def. vii} \\
          &= rev(a)rev(b)rev(a)rev(b)(rev(aab) + rev(bbaba))^* \mbox{ --- def. vii} \\
          &= abab(rev(aab) + rev(bbaba))^* \mbox{ --- def. iii} \\
          &= abab(rev(b)rev(a)rev(a) + rev(bbaba))^* \mbox{ --- def. vii} \\
          &= abab(baa + rev(bbaba))^* \mbox{ --- def. iii} \\
          &= abab(baa + rev(a)rev(b)rev(a)rev(b)rev(b))^* \mbox{ --- def. vii} \\
          &= abab(baa + ababb)^* \mbox{ --- def. iii} \\
    \end{align*}
    \begin{itemize}
        \item Expresion final: $\boxed{abab(baa + ababb)^*}$
    \end{itemize}
    
    
    \item (1.5pts) Decidir si las siguientes expresiones regulares son equivalentes, en case de serlo, demostrarlo, y si no, justifique su respuesta:
    \begin{enumerate}[label=\alph*)]
        \item $((01 + 1)^*(10 + 0)^* + \varnothing^*)^* = (0 + 1)^*(01 + 10)$ 
        
        En particular, la cadena $0$ se puede formar con la primera expresión:

        $0 = ((01 + 1)^0(10 + 0)^1 + \varnothing ^ 0)^1 = (01 + 1)^0(10 + 0) = \varepsilon(10 + 0) = 10 + 0 = 0$ 

        Por otro lado, la expresión derecha son todas las cadenas formadas por $0$'s y $1$'s que tienen como sufijo a $01$ o $10$, entonces $0$ no pertenece al lenguaje generado por la expresión.
        
        Por tanto, no son equivalentes.
            \item $1^*01((01)^* + 1)^* = (((11\varnothing^*) + 1^*)^*1^*)^*01((01)^* + 1)^*$
            
            Con la expresión del lado derecho podemos formar a $\varepsilon$:

            $(((11\varnothing^*) + 1^*)^*1^*)^*01((01)^* + 1)^0 = \varepsilon$

            Pero $\varepsilon$ no se puede formar con la expresión del lado izquierdo, porque siempre tenemos la subcadena $01$: $1^001((01)^* + 1)^0 = \varepsilon 01 \varepsilon = 01$, siendo $01$ la cadena mas pequeña.
            
            Por tanto, no son equivalentes.
    \end{enumerate}
\end{enumerate}

\end{document}
