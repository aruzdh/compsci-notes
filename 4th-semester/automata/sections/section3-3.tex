\section{Autómatas Finitas No Deterministas con \texorpdfstring{$\varepsilon$-transiciones}{ε-transiciones}}

\dfn{Autómata Finito No Determinista con $\varepsilon$-transiciones}{
  Un \textbf{Autómata Finito No Determinista con $\varepsilon$-transiciones }(\textit{AF$N_{\varepsilon}$}) es una \textit{5-tupla}
  \[M = (Q, \Sigma, \delta, q_0, F),\]
  donde
  \begin{itemize}
  \item \textit{Q} es un \textit{conjunto finito y no vacío de estados}.
    \item $\Sigma$ es el \textit{alfabeto} de entrada.
    \item $\delta$ es una \textit{función de transición} definida como
      \[\delta : Q \times (\Sigma \cup \{\varepsilon\}) \longrightarrow \mathcal P(Q)\]
    \item $q_0$ es el \textit{estado inicial}.
    \item \textit{F} es el conjunto de \textit{estados finales} ($F \subseteq Q$).
  \end{itemize}
}

\nt{
  Se observa que la única diferencia entre un \textit{AFN} y un \textit{AF$N_{\varepsilon}$} es el el dominio de $\delta$. Es decir, las transiciones de la forma $\delta(q, \varepsilon) = p$ son validas y significa que el autómata puede cambiar de estado \textbf{sin} leer algún símbolo.
}

\nt{
  Aunque lo anterior es bastante útil, a su vez introduce un \textit{no-determinismo} más complicado de modelar matemáticamente.
}

Algunas caracteristicas de estos autómatas son las siguientes.
\begin{itemize}
  \item Se permiten múltiples cómputos para una cadena de entrada.
  \item Puede existir cómputos bloqueados o "muertos".
  \item A diferencia de los \textit{AFD} y los \textit{AFN}, pueden existir cómputos infititos, es decir, surge la \textbf{no-terminación}.
  \item La presencia de transiciones $\varepsilon$ permite mayor libertad en el diseño del autómata.
\end{itemize}

\subsection{\texorpdfstring{$\varepsilon$-cerradura}{ε-cerradura}}

\dfn{$\varepsilon$-cerradura}{
  Dado un estado \textit{q}, se define la \textbf{$\varepsilon$-cerradura} de \textit{q} como el conjunto de \textit{estados alcanzables} desde \textit{q mediante 0 o más transiciones $\varepsilon$}. Es decir
  \[Cl_{\varepsilon}(q) = \{s \in Q \mid \exists p_1,\cdots, p_n \text{ con } p_1 = q,~ p_n = s,~ p_i \in \delta(p_{i - 1}, \varepsilon)\}\]

  Recursivamente se tiene 

  \begin{itemize}
    \item $q \in Cl_{\varepsilon}(q)$
    \item Si $r \in Cl_{\varepsilon}(q)$ y $\delta(r, \varepsilon) = s$, entonces $s \in Cl_{\varepsilon}(q)$
  \end{itemize}

  Esta definición se \textit{extiende a conjuntos de estados} como 
  \[Cl_{\varepsilon}(S) = \bigcup_{q \in S}Cl_{\varepsilon}(q)\]
}

\subsection{Función extendida \texorpdfstring{$\delta^*$}{delta*}}

\dfn{Función extendida $\delta^*$}{
  Se define la \textbf{función extendida $\delta^*$} de la siguiente forma.
  \[ \delta^* : Q \times (\Sigma^* \cup \{\varepsilon\}) \longrightarrow \mathcal P(Q)\]
  tal que
  \begin{itemize}
    \item \[\delta^*(q, \varepsilon) = Cl_{\varepsilon}(q)\]
  \item \[\delta^*(q, wa) = Cl_{\varepsilon}(\bigcup_{p \in \delta^*(q,w)}\delta(q,a))\]
  \end{itemize}
}

\subsection{Lenguaje de un AFN\texorpdfstring{$_{\varepsilon}$}{ε}}
\dfn{Lenguaje de un AF$N_{\varepsilon}$}{
  Sea $M = (Q, \Sigma, \delta, q_0, F$ un \textit{AF$N_{\varepsilon}$}, el \textbf{lenguaje aceptado} por \textit{M} se define como
  \[L(M) = \{w \in \Sigma^* \mid \delta^*(q_0,w) \cap F \neq \varnothing \}\]
}

\ex{AF$N_{\varepsilon}$}{
  Dado \textit{M} el autómata siguiente:
  \begin{center}
      \begin{tabular}{c|c c c}
          & $\varepsilon$ & $a$ & $b$ \\
          \hline
          $\rightarrow q_1$ & $\emptyset$ & $\{q_2\}$ & $\emptyset$ \\
          $q_2$ & $\{q_3\}$ & $\{q_2\}$ & $\{q_4\}$ \\
          $q_3$ & $\emptyset$ & $\{q_4\}$ & $\{q_3, q_4\}$ \\
          $q_4$ & $\{q_5\}$ & $\{q_4\}$ & $\emptyset$ \\
          * $q5$ & $\emptyset$ & $\emptyset$ & $\emptyset$
      \end{tabular}
  \end{center}

  Se calcula paso a paso el reconocimiento de la siguiente cadena y se determina si es o no aceptada: $\delta^*(q_1, abba)$\\
  Por un lado tenemos:
  \begin{align*}
    &\delta^*(q_1, abba) = Cl_{\varepsilon}(\bigcup_{q' \in \delta^*(q_1, abb)} \delta(q',a)) \\
    &\delta^*(q_1, abb) = Cl_{\varepsilon}(\bigcup_{q' \in \delta^*(q_1, ab)} \delta(q',b)) \\
    &\delta^*(q_1, ab) = Cl_{\varepsilon}(\bigcup_{q' \in \delta^*(q_1, a)} \delta(q',b)) \\
    &\delta^*(q_1, a) = Cl_{\varepsilon}(\bigcup_{q' \in \delta^*(q_1, \varepsilon)} \delta(q',a)) \\
    &\delta^*(q_1, \varepsilon) = Cl_{\varepsilon}(q_1) \\
  \end{align*}

  Sustituyendo de abajo hacia arriba:
  \begin{align*}
    &\delta^*(q_1, \varepsilon) = Cl_{\varepsilon}(q_1) = \{q_1\} \\
    &\delta^*(q_1, a) = Cl_{\varepsilon}(\delta(q_1,a)) = Cl_{\varepsilon}(\{q_2\}) = \{q_2, q_3\} \\
    &\delta^*(q_1, ab) = Cl_{\varepsilon}(\bigcup_{q' \in\{q_2, q_3\}}\delta(q',b)) = Cl_{\varepsilon}(\delta(q_2,b) \cup \delta(q_3,b)) = Cl_{\varepsilon}(\{q_3,q_4\}) = \{q_3, q_4, q_5\} \\
    &\delta^*(q_1, abb) = Cl_{\varepsilon}(\bigcup_{q' \in\{q_3, q_4, q_5\}}\delta(q',b)) = Cl_{\varepsilon}(\delta(q_3,b) \cup \delta(q_4,b) \cup \delta(q_5, b)) = Cl_{\varepsilon}(\{q_3,q_4\}) = \{q_3, q_4,q_5\} \\
    &\delta^*(q_1, abba) = Cl_{\varepsilon}(\bigcup_{q' \in\{q_3, q_4, q_5\}}\delta(q',a)) = Cl_{\varepsilon}(\delta(q_3,a) \cup \delta(q_4,a) \cup \delta(q_5, a)) = Cl_{\varepsilon}(\{q_4\}) = \{q_4,q_5\}
  \end{align*}
  \underline{La cadena $abba$ \textbf{si} es aceptada.}
}

\subsection{Eliminación de \texorpdfstring{$\varepsilon$-transiciones}{ε-transiciones}}
Esta eliminación implica que un \textit{AFN} es un \textit{AF$N_{\varepsilon}$} con la particularidad de que no tiene $\varepsilon$-transiciones.

\dfn{AFN a partir de un AF$N_{\varepsilon}$}{
  Un \textit{AFN} \textbf{construido a partir} de un \textit{AF$N_{\varepsilon}$} $M_{\varepsilon}= (Q_{\varepsilon}, \Sigma, \delta_{\varepsilon}, q_0^{\varepsilon}, F_{\varepsilon})$ es una \textit{5-tupla}
  \[M = (Q, \Sigma, \delta, q_0, F),\]
  donde
  \begin{itemize}
  \item $Q = Q_{\varepsilon}$
    \item $\Sigma$ es el \textit{alfabeto} de entrada.
    \item $\delta(q,a)= \delta_{\varepsilon}^*(q,a) =  CL_{\varepsilon}(\bigcup_{p \in Cl_{\varepsilon}(q)}\delta_{\varepsilon}(p,a))$
    \item $q_0 = q_0^{\varepsilon}$
    \item $
      F = \begin{cases}
        F_1 \cup \{q_0\} \text{ si } Cl_{\varepsilon}(q_0) \cap F \neq \varnothing\\
        F_1 \text{ en caso contrario}\\
          \end{cases}
        $
  \end{itemize}
}

\ex{Eliminación de $\varepsilon$-transiciones}{
  Dado el autómata del ejemplo anterior.
  Se eliminan las transiciones $\varepsilon$ para obtener un AFN $M'$ tal que $\mathcal{L}(M) = \mathcal{L}(M')$\\
  Autómata con $\varepsilon$-transiciones:
  \begin{center}
      \begin{tikzpicture}[shorten >=1pt, node distance=2cm, on grid, auto]
          \tikzset{initial text={}}

          % INITIAL STATE
          \node[state, initial] (q1) {$q_1$}; 

          % STATES
          \node[state] (q2) [right=of q1] {$q_2$}; 
          \node[state] (q3) [right=of q2] {$q_3$}; 
          \node[state] (q4) [below=of q3] {$q_4$};
          \node[state, accepting] (q5) [right=of q4] {$q_5$}; 

          % TRANSITIONS
          \path[->]
              (q1) edge node {a} (q2) 
              (q2) edge [loop above] node {a} (q2) 
              (q2) edge node {b} (q4)
              (q2) edge node {$\epsilon$} (q3)
              (q3) edge node {a,b} (q4)
              (q3) edge [loop above] node {b} (q3)
              (q4) edge node {$\varepsilon$} (q5)
              (q4) edge [loop below] node {a} (q4);

          \draw[->] ([xshift=-.5cm] qi.west) -- (qi.west) node[midway, above] {};

      \end{tikzpicture}
  \end{center}

  Autómata sin $\varepsilon$-transiciones:\\
  - Tabla con las $\varepsilon$-transiciones de cada estado:

  \begin{center}
      \begin{tabular}{c|c}
          & $Cl_{\varepsilon}(q_i)$ \\
          \hline
          $q_1$ & $\{q_1\}$ \\
          $q_2$ & $\{q_2,q_3\}$ \\
          $q_3$ & $\{q_3\}$ \\
          $q_4$ & $\{q_4,q_5\}$ \\
          $q_5$ & $\{q_5\}$ \\
      \end{tabular}
  \end{center}

  Aplicando la nueva delta del AFN definida como: 
  $$\delta(q,a) = \delta^*(q,a) = Cl_{\varepsilon}(\bigcup_{p \in Cl_{\varepsilon}(q)} \delta_{\varepsilon}(p,a))$$

  \begin{align*}
    &\delta(q_1, a) = Cl_{\varepsilon}(\bigcup_{p \in Cl_{\varepsilon}(q_1)} \delta_{\varepsilon}(p,a)) = Cl_\varepsilon(\delta_\varepsilon(q_1,a)) = Cl_\varepsilon(\{q_2\}) = \{q_2, q_3\} \\
    &\delta(q_1, b) = Cl_{\varepsilon}(\bigcup_{p \in Cl_{\varepsilon}(q_1)} \delta_{\varepsilon}(p,b)) = Cl_\varepsilon(\delta_\varepsilon(q_1,b)) = Cl_\varepsilon(\varnothing) = \varnothing \\
    &\delta(q_2, a) = Cl_{\varepsilon}(\bigcup_{p \in Cl_{\varepsilon}(q_2)} \delta_{\varepsilon}(p,a)) = Cl_\varepsilon(\delta_\varepsilon(q_2,a)\cup \delta_\varepsilon(q_3, a)) = Cl_\varepsilon(\{q_2,q_4\}) = \{q_2, q_3, q_4, q_5\} \\
    &\delta(q_2, b) = Cl_{\varepsilon}(\bigcup_{p \in Cl_{\varepsilon}(q_2)} \delta_{\varepsilon}(p,b)) = Cl_\varepsilon(\delta_\varepsilon(q_2,b)\cup \delta_\varepsilon(q_3, b)) = Cl_\varepsilon(\{q_3,q_4\}) = \{q_3, q_4, q_5\} \\
  \end{align*}
  \begin{align*}
    &\delta(q_3, a) = Cl_{\varepsilon}(\bigcup_{p \in Cl_{\varepsilon}(q_3)} \delta_{\varepsilon}(p,a)) = Cl_\varepsilon(\delta_\varepsilon(q_3, a)) = Cl_\varepsilon(\{q_4\}) = \{q_4, q_5\} \\
    &\delta(q_3, b) = Cl_{\varepsilon}(\bigcup_{p \in Cl_{\varepsilon}(q_3)} \delta_{\varepsilon}(p,b)) = Cl_\varepsilon(\delta_\varepsilon(q_3, b)) = Cl_\varepsilon(\{q_3,q_4\}) = \{q_3,q_4, q_5\} \\
    &\delta(q_4, a) = Cl_{\varepsilon}(\bigcup_{p \in Cl_{\varepsilon}(q_4)} \delta_{\varepsilon}(p,a)) = Cl_\varepsilon(\delta_\varepsilon(q_4, a)\cup \delta_\varepsilon(q_5,a)) = Cl_\varepsilon(\{q_4\}) = \{q_4, q_5\} \\
    &\delta(q_4, b) = Cl_{\varepsilon}(\bigcup_{p \in Cl_{\varepsilon}(q_4)} \delta_{\varepsilon}(p,b)) = Cl_\varepsilon(\delta_\varepsilon(q_4, b)\cup \delta_\varepsilon(q_5,b)) = Cl_\varepsilon(\varnothing) = \varnothing\\
    &\delta(q_5, a) = Cl_{\varepsilon}(\bigcup_{p \in Cl_{\varepsilon}(q_5)} \delta_{\varepsilon}(p,a)) = Cl_\varepsilon(\delta_\varepsilon(q_5,a)) = Cl_\varepsilon(\varnothing) = \varnothing \\
    &\delta(q_5, b) = Cl_{\varepsilon}(\bigcup_{p \in Cl_{\varepsilon}(q_5)} \delta_{\varepsilon}(p,b)) = Cl_\varepsilon(\delta_\varepsilon(q_5,b)) = Cl_\varepsilon(\varnothing) = \varnothing \\
  \end{align*}
  Con lo anterior se puede construir el siguiente $AFN$ sin $\varepsilon$-transiciones
  \begin{center}
      \begin{tikzpicture}[shorten >=1pt, node distance=2cm, on grid, auto]
          \tikzset{initial text={}}

          % INITIAL STATE
          \node[state, initial] (q1) {$q_1$}; 

          % STATES
          \node[state] (q2) [right=of q1] {$q_2$}; 
          \node[state] (q3) [right=of q2] {$q_3$}; 
          \node[state] (q4) [below=4cm of q3] {$q_4$};
          \node[state, accepting] (q5) [right=of q4] {$q_5$}; 

          % TRANSITIONS
          \path[->]
              (q1) edge node {a} (q2) 
              (q1) edge [bend left=60] node {a} (q3)
              (q2) edge [loop below] node {a} (q2)
              (q2) edge node {a,b} (q3)
              (q2) edge [bend right=15] node {a,b} (q4)
              (q2) edge node {a,b} (q5)
              (q3) edge [loop above] node {b} (q3)
              (q3) edge [bend left=50] node {a,b} (q4)
              (q3) edge [bend left=50] node {a,b} (q5)
              (q4) edge node {a} (q5)
              (q4) edge [loop below] node {a} (q4);

          \draw[->] ([xshift=-.5cm] qi.west) -- (qi.west) node[midway, above] {};
      \end{tikzpicture}\\
  \end{center}
}

\ex{AFD a partir de AFN}{
  Dado el \textit{AFN} obtenido en el ejemplo previo. 
  Se elimina el no-determinismo para encontrar un AFD $M''$ tal que $\mathcal{L}(M') = \mathcal{L}(M'')$\\
    Los conjuntos quedan definidos de la siguiente manera:

    \begin{center}
      \begin{tabular}{c|c|c}
        \hline
          & a & b \\
        \hline
        $\rightarrow q_1$ & $q_2\,q_3$ & $\varnothing$ \\
        $q_2$          & $q_2\,q_3\,q_4\,q_5$      & $q_3\,q_4\,q_5$ \\
        $q_3$             & $q_4\,q_5$ & $q_3\,q_4\,q_5$ \\
        $q_4$          & $q_4\,q_5$ & $\varnothing$ \\
        $*q_5$             & $\varnothing$      & $\varnothing$ \\
        \hline
        $q_2\,q_3$             & $q_2\,q_3\,q_4\,q_5$      & $q_3\,q_4\,q_5$ \\
        $*q_2\,q_3\,q_4\,q_5$             & $q_2\,q_3\,q_4\,q_5$      & $q_3\,q_4\,q_5$ \\
        $*q_3\,q_4\,q_5$             & $q_4\,q_5$      & $q_3\,q_4\,q_5$ \\
        $*q_4\,q_5$             & $q_4\,q_5$      & $\varnothing$ \\
        \hline
      \end{tabular}
    \end{center}

  Así, el autómata queda de la siguiente manera
  \begin{center}
      \begin{tikzpicture}[shorten >=1pt, node distance=2.5cm, on grid, auto]
          \tikzset{initial text={}}

          % Estados
          \node[state, initial] (q1) {$q_1$}; 
          \node[state] (q23) [right=of q1] {$q_2 q_3$}; 
          \node[state, accepting] (q2345) [right=of q23] {$q_2 q_3 q_4 q_5$};
          \node[state, accepting] (q345) [below=of q2345] {$q_3 q_4 q_5$};
          \node[state, accepting] (q45) [below=of q345] {$q_4 q_5$};

          % Transiciones
          \path[->]
              (q1) edge node {a} (q23)
              (q23) edge [loop above] node {a} (q23)
              (q23) edge node {b} (q345)
              (q23) edge node {a} (q2345)
              (q2345) edge [loop right] node {a} (q2345)
              (q2345) edge [bend left] node {b} (q345)
              (q345) edge [loop right] node {b} (q345)
              (q345) edge node {a} (q45)
              (q45) edge [loop right] node {a} (q45);
      \end{tikzpicture}
  \end{center}
}

\subsection{Equivalencias}
\textbf{Cualquier} tipo de \textit{autómata finito} puede convertirse en un \textit{AFD} y viceversa, es decir
\[AFD \Longleftrightarrow AFN \Longleftrightarrow AFN_{\varepsilon}\]
Los ejemplos dados a lo largo de esta sección son la prueba de esto.
