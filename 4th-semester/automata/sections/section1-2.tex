\section{Lenguajes}
\subsection{Concepto}
\dfn{Lenguaje}{
  Un \textbf{lenguaje} es un \textit{conjunto de cadenas} sbre un alfabeto $\Sigma$
}

\ex{Lenguaje}{
  Dado $\Sigma = \{m,u\}$. Los siguientes son lenguajes.
  \begin{itemize}
    \item $\varnothing$ es un lenguaje sobre $\Sigma$.
    \item $\Sigma^*$ es un lenguaje sobre $\Sigma$.
    \item $L_1 = \{w :~ \mid w \mid = 2n,~ n \ge 0 \} $
    \item $L_2 = \{w \mid w = w^R \land \mid w \mid = 3 \} $
  \end{itemize}
}
\subsection{Operaciones de Lenguajes}
Sean $L_1, L_2$ dos lenguajes sobre $\Sigma^*$ ($L_1, L_2 \subseteq \Sigma^*$). Las siguientes son operaciones entre lenguajes.
\begin{itemize}
  \item Unión: \[L_1 \cup L_2 = \{w \in \Sigma^* \mid w \in L_1 \lor w \in L_2\}\]
\item Intersección: \[L_1 \cap L_2 = \{w \in \Sigma^* \mid w \in L_1 \land w \in L_2\}\]
\item Diferencia: \[L_1 - L_2 = \{w \in \Sigma^* \mid w \in L_1 \land w \notin L_2\}\]
\item Complemento: \[\overline L = \{w \in \Sigma^* \mid w \notin L\}\]
  \item Concatenación
    \[L_1 \cdot L_2 = \{w = xy \mid x \in L_1 \land y \in L_2\}\]
  \item Reversa de un Lenguaje
    \[L^R = \{w^R \mid w \in L\}\]
\end{itemize}

\ex{Operaciones de Lenguajes}{
  Consideremos los lenguajes $L_1$ y $L_2$ del ejemplo anterior. Se tiene que
  \begin{itemize}
    \item $L_1 \cup L_2 = \{mum, umu, uuu, \varepsilon, mm, mu, um, \cdots\}$ 
    \item $L_1 \cap L_2 = \varnothing$
    \item $L_1 - L_2 = L_1$
    \item $\overline{L_1} = \{m, u, mmm, mmu, \cdots\} = \{w \in \{m,u\}^* :~ \mid w \mid = 2n + 1,~ n \ge 0\}$
    \item $L_1 \cdot L_2 = \{m^3, u^3, mum, umu, m^5, m^2u^3, m^2mum, \cdots\}$
  \end{itemize}
}

Las siguientes son propiedades de las operaciones anteriores. Sean \textit{L, M, N} lenguajes.
\begin{itemize}
  \item Asociatividad 
    \[(L \cup M) \cup N = L \cup (M \cup N)\] 
    \[(L \cap M) \cap N = L \cap (M \cap N)\]
  \item Conmutatividad de Unión e Intersección
    \[L \cup M = M \cup L\]
    \[L \cap M = M \cap L\]
  \item Elemento Neutro
    \[L \cup \varnothing = L\]
    \[L \cap \Sigma^* = L\]
  \item Casos especiales
    \[L \cap \varnothing = \varnothing\]
    \[L \cup \Sigma^* = \Sigma^*\]
  \item Distributividad
    \[L \cup(M \cap N) = (L \cup M) \cap (L \cup N)\]
    \[L \cap(M \cup N) = (L \cap M) \cup (L \cap N)\]
  \item De Morgan
    \[\overline{L \cap M} = \overline L \cup \overline M\]
    \[\overline{L \cup M} = \overline L \cap \overline M\]
  \item Complemento como una diferencia
    \[\overline L = \Sigma^* - L\]
  \item Concatenación con $\varnothing$
    \[L \cdot \varnothing = \varnothing \cdot L = \varnothing\]
  \item Concatenación con $\{\varepsilon\}$
    \[L \cdot \{\varepsilon\} = \{\varepsilon\} \cdot L = L\]
  \item Conmutatividad de la Concatenación
    \[(LM) N = L(MN)\]
  \item Distributividad de la Concatenación y Unión
    \[L(M \cup N) = LM \cup LN \neq (M \cup N) L = ML \cup NL\]
  \item Potencia
    \[L^n = \underbrace{L \cdot L \cdot \cdots \cdot L}_{\text{n veces}}\]
  \item Reversa de la Reversa de L
    \[(L^R)^R = L\]
  \item Reversa de la Concatenación
    \[(LM)^R = M^R L^R\]
  \item Reversa de la Unión
    \[(L \cup M)^R = L^R \cup M^R\]
  \item Reversa de la Intersección
    \[(L \cap M)^R = L^R \cap M^R\]
\end{itemize}

\subsection{Cerradura de Klenne y Cerradura Positiva}

\dfn{Cerradura de Klenne}{
  La \textbf{Cerradura o Estrella de Klenne} de un \textit{lenguaje} es el \textit{conjunto de todas las potencias de L}, y se denota por
  \[L^* = \bigcup_{i = 0}^{\infty}L^i\]
}

\ex{Cerradura de Klenne}{
  Sea $L = \{aa, b\}$. Se tiene que
  \begin{align*}
    &L^0 = \{\varepsilon\}\\
    &L^1 = \{aa, b\}\\
    &L^2 = \{aaaa, aab, baa, bb\}\\
    &L^3 = \{a^6, a^2a^2b, a^2ba^2, a^2b^2, \cdots\}\\
    &\cdots
  \end{align*}
}

\nt{
  Dado \textit{L} un lenguaje arbitrario. Se tiene que
  \begin{itemize}
    \item $L^0 = \{\varepsilon\}$
    \item $L^1 = L$
  \end{itemize}
}

\dfn{Cerradura Positiva}{
  La \textbf{Cerradura Positiva} de un lenguaje \textit{L} es denotada por
  \[L^+ = \bigcup_{i = 1}^{\infty}L^i\]
}

\nt{
  Se tiene que
  \begin{itemize}
    \item $\varnothing^+ = \varnothing$
    \item $\{\varepsilon\}^+ = \{\varepsilon\}$
  \end{itemize}
}

Las siguientes son propiedades de la Cerradura de Klenne / Positiva.

\begin{enumerate}
  \item \[L^* = \{\varepsilon\} \cup L^+\]
  \item \[L^+ = L^* \cdot L\]
  \item \[L^* = L^+ \text{si y solo si } \varepsilon \in L\]
  \item \[(L^*)^* = L^*\]
  \item \[(L^+)^* = (L^*)^+ = L^*\]
  \item \[L^* \cdot L^* = L^*\]
  \item \[(L^+)^+ = L^+\]
  \item \[L^+ \cdot L^+ \subseteq L^+\]
\end{enumerate}

