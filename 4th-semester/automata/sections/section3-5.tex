\section{Teorema de Klenne}
\thm{Teorema de Klenne}{
  Un lenguaje es \textbf{regular} \textit{si y solo si es aceptado por un autómata finito}.
}
\begin{myproof}
  La demostración consta de dos partes.
  \begin{enumerate}
    \item Síntesis: Dado un lenguaje regular \textit{L}, existe un autómata finito \textit{M} tal que $L = \mathcal L(M)$.\\
      Esta parte es constructiva y se hará mediante inducción sobre las expresiones regulares.

      \textbf{Base de Inducción}\\
      Caso $\alpha = \varnothing$. El siguiente autómata reconoce a $\mathcal L(\alpha)$.
      \begin{center}
          \begin{tikzpicture}[shorten >=1pt, node distance=2cm, on grid, auto]
              \tikzset{initial text={}}
              
              \node[state, initial] (qi) {$q_i$}; 

              \node (start) [left=of qi] {};
              \draw[->] (start) -- (qi);

          \end{tikzpicture}
      \end{center}

      Caso $\alpha = \varepsilon$. El siguiente autómata reconoce a $\mathcal L(\alpha)$.
      \begin{center}
          \begin{tikzpicture}[shorten >=1pt, node distance=2cm, on grid, auto]
              \tikzset{initial text={}}
              
              \node[state, initial, accepting] (qi) {$q_i$}; 
              
              \node (start) [left=of qi] {};
              \draw[->] (start) -- (qi);

          \end{tikzpicture}
      \end{center}

      Caso $\alpha = a$ con $a \in \Sigma$. El siguiente autómata reconoce a $\mathcal L(\alpha)$.
      \begin{center}
          \begin{tikzpicture}[shorten >=1pt, node distance=2cm, on grid, auto]
              \tikzset{initial text={}}
              
              \node[state, initial] (qi) {$q_i$}; 
              \node[state] (qa) [right=of qi] {$q_a$}; 

              \path[->]
                      (qi) edge node {a} (qa);
              
              \node (start) [left=of qi] {};
              \draw[->] (start) -- (qi);

          \end{tikzpicture}
      \end{center}

      \textbf{Hipótesis de Inducción}\\
      Sean $M, N$ dos autómatas que reconocen los lenguajes $L(\beta)$ y $L(\gamma)$, respectivamente.

      \textbf{Paso Inductivo}\\
      Caso $\alpha = \beta + \gamma$. El siguiente autómata reconoce a $\mathcal L(\alpha)$.
      \begin{center}
        \begin{tikzpicture}[shorten >=1pt, node distance=2cm, on grid, auto]
            \tikzset{initial text={}}
            
            \node[state, initial] (qi) {$q_i$}; 
            \node[state, rectangle] (M) [above right=of qi] {$M$}; 
            \node[state, rectangle] (N) [below right=of qi] {$N$}; 

            \path[->]
                (qi) edge node {$\varepsilon$} (M)
                (qi) edge node [below left] {$\varepsilon$} (N);
            
            \node (start) [left=of qi] {};
            \draw[->] (start) -- (qi);

        \end{tikzpicture}
      \end{center} 
      donde las $\varepsilon$-transiciones van hacia los estados iniciales de \textit{M} y \textit{N}.

      Caso $\alpha = \beta \gamma$. El siguiente autómata reconoce a $\mathcal L(\alpha)$.
      \begin{center}
        \begin{tikzpicture}[shorten >=1pt, node distance=2cm, on grid, auto]
            \tikzset{initial text={}}
            
            \node[state, rectangle, initial] (M) {$M$}; 
            \node[state, rectangle] (N) [right=of M] {$N$}; 

            \path[->]
                (M) edge node {$\varepsilon$} (N);
            
            \node (start) [left=of M] {};
            \draw[->] (start) -- (M);

        \end{tikzpicture}
      \end{center} 
      donde el estado inicial es el de \textit{M} y las $\varepsilon$-transiciones van de los estados finales de \textit{M} hacia el inicial de \textit{N}.

      Caso $\alpha = \beta^*$. El siguiente autómata reconoce a $\mathcal L(\alpha)$.
      \begin{center}
          \begin{tikzpicture}[shorten >=1pt, node distance=2cm, on grid, auto]
              \tikzset{initial text={}}
              
              \node[state, initial] (qi) {$q_i$}; 
              \node[state, rectangle] (M) [right=of qi] {$M$}; 

              \path[->]
                (qi) edge [bend left] node {$\varepsilon$} (M)
                (M) edge [bend left] node {$\varepsilon$} (qi);
              
              \node (start) [left=of qi] {};
              \draw[->] (start) -- (qi);

          \end{tikzpicture}
      \end{center}
      donde las $\varepsilon$-transiciones conectan el estado inicial y los finales con el nuevo estado inicial.

    \item Análisis: Dado un autómata finito \textit{M}, existe una expresión regular $\alpha$ tal que $\mathcal L(M) = \mathcal L(\alpha)$ ($\mathcal L(M)$ es regular).

      Esta parte se demuestra usando el \textit{Lema de Arden}.
  \end{enumerate}
\end{myproof}

\section{Lema de Arden}

\dfn{Ecuaciones lineales}{
  Sean $A, B \subseteq \in \Sigma^*$ y \textit{X} una variable. Una \textbf{ecuación lineal derecha} de \textit{X} es una expresión de la forma
  \[X = AX + B\]
  De manera completamente análoga, una \textbf{ecuación lineal izquierda} de \textit{X} es de la forma
  \[X = XA + B\]
  donde \textit{+} representa la \textit{unión de lenguajes}.
}

\mlemma{Lema de Arden}{
  Sean $A, B \subseteq \Sigma^*$ dos lenguajes y $X = AX + B$ una ecuación lineal derecha.
  \begin{enumerate}
    \item $A^*B$ es una \textit{solución} de $X = AX + B$.
    \item Si \textit{C} es otra solución, entonces $A^*B \subseteq C$. Es decir, $A^*B$ es la solución mínima.
    \item Si $\varepsilon \notin A$, entonces $A^*B$ es única.
  \end{enumerate}
}

Usando el \textit{Lema de Arden} y la definición de \textit{lenguaje de un estado} se puede mostrar que un \textit{autómata finito} implica una \textit{expresión regular}. Para lograrlo, se hace por medio de un sistema de ecuaciones a partir de un \textit{AF} (preferentmente uno determinista). A continuación se presenta un ejemplo. Además, para ejemplificar la conexión entre cada sección, Además, para ejemplificar la conexión entre los temas de cada sección, se parte de un \textit{AF$N_\varepsilon$}.

\ex{Conexión entre temas}{
  Se muestra un autómata finito que acepta el siguiente Lenguaje: $\alpha = (0 + 1)^* (01 + 110)$.\\
  Usando el \textit{Teorema de Klenne}.
  Por un lado tenemos $0 + 1$:
    \begin{center}
        \begin{tikzpicture}[shorten >=1pt, node distance=2cm, on grid, auto]
            \tikzset{initial text={}}
            
            \node[state, initial] (qi) {$q_i$}; 
            \node[state] (q1) [above right=of qi] {$q_1$}; 
            \node[state] (q0) [below right=of qi] {$q_0$}; 

            \node[state, accepting] (q2) [right=of q1] {$q_2$};
            \node[state, accepting] (q3) [right=of q0] {$q_3$};

            \path[->]
                (qi) edge node {$\varepsilon$} (q1)
                (qi) edge node {$\varepsilon$} (q0)
                (q1) edge node {1} (q2)
                (q0) edge node {0} (q3);
            
            \node (start) [left=of qi] {};
            \draw[->] (start) -- (qi);

        \end{tikzpicture}
    \end{center}

    Aplicando * a la expresión anterior:
    \begin{center}
        \begin{tikzpicture}[shorten >=1pt, node distance=2cm, on grid, auto]
            \tikzset{initial text={}}
            
            \node[state, initial] (qi) {$q_i$}; 
            \node[state] (q1) [above right=of qi] {$q_1$}; 
            \node[state] (q0) [below right=of qi] {$q_0$}; 

            \node[state, accepting] (q2) [right=of q1] {$q_2$};
            \node[state, accepting] (q3) [right=of q0] {$q_3$};

            \path[->]
                (qi) edge node {$\varepsilon$} (q1)
                (qi) edge node [below] {$\varepsilon$} (q0)
                (qi) edge[out=100, in=120, looseness=1.2] node {$\varepsilon$} (q2)
                (qi) edge[out=260, in=240, looseness=1.2] node [below] {$\varepsilon$} (q3)
                (q1) edge node {1} (q2)
                (q0) edge node {0} (q3)
                (q3) edge [bend right=.5cm] node {$\varepsilon$} (qi)
                (q2) edge [bend left=.5cm] node {$\varepsilon$} (qi);
            
            \node (start) [left=of qi] {};
            \draw[->] (start) -- (qi);

        \end{tikzpicture}
    \end{center}

    Por otro lado tenemos $01 + 110$:
    \begin{center}
        \begin{tikzpicture}[shorten >=1pt, node distance=2cm, on grid, auto]
            \tikzset{initial text={}}
            
            \node[state, initial] (qi) {$q_i$}; 
            \node[state] (q1) [above right=of qi] {$q_1$}; 
            \node[state] (q0) [below right=of qi] {$q_0$}; 

            \node[state] (q2) [right=of q1] {$q_2$};
            \node[state] (q4) [right=of q2] {$q_4$};
            \node[state, accepting] (q5) [right=of q4] {$q_5$};
            \node[state] (q3) [right=of q0] {$q_3$};
            \node[state, accepting] (q6) [right=of q3] {$q_6$};

            \path[->]
                (qi) edge node {$\varepsilon$} (q1)
                (qi) edge node {$\varepsilon$} (q0)
                (q1) edge node {1} (q2)
                (q2) edge node {1} (q4)
                (q4) edge node {0} (q5)
                (q3) edge node {1} (q6)
                (q0) edge node {0} (q3);
            
            \node (start) [left=of qi] {};
            \draw[->] (start) -- (qi);

        \end{tikzpicture}
    \end{center}

    Concatenando los ultimos dos autómatas obtenemos el correspondiente para $(0 + 1)^*(01 + 110)$:

    \begin{center}
        \begin{tikzpicture}[shorten >=1pt, node distance=2cm, on grid, auto]
            \tikzset{initial text={}}
            
            % Definición de nodos del primer autómata
            \node[state, initial] (q0) {$q_0$}; 
            \node[state] (q1) [above right=of q0] {$q_1$}; 
            \node[state] (q2) [below right=of q0] {$q_2$}; 
            \node[state] (q3) [right=of q1] {$q_3$};
            \node[state] (q4) [right=of q2] {$q_4$};

            % Transiciones del primer autómata
            \path[->]
                (q0) edge node {$\varepsilon$} (q1)
                (q0) edge node [below] {$\varepsilon$} (q2)
                (q0) edge[out=100, in=120, looseness=1.2] node {$\varepsilon$} (q3)
                (q0) edge[out=260, in=240, looseness=1.2] node [below] {$\varepsilon$} (q4)
                (q1) edge node {1} (q3)
                (q2) edge node {0} (q4)
                (q4) edge [bend right=.5cm] node {$\varepsilon$} (q0)
                (q3) edge [bend left=.5cm] node {$\varepsilon$} (q0);

            % Definición de nodos del segundo autómata con nombres cambiados
            \node[state] (q5) [right=5cm of q0] {$q_5$}; 
            \node[state] (q6) [above right=of q5] {$q_6$}; 
            \node[state] (q7) [below right=of q5] {$q_7$}; 
            \node[state] (q8) [right=of q6] {$q_8$};
            \node[state] (q9) [right=of q8] {$q_9$};
            \node[state, accepting] (q10) [right=of q9] {$q_{10}$};
            \node[state] (q11) [right=of q7] {$q_{11}$};
            \node[state, accepting] (q12) [right=of q11] {$q_{12}$};

            % Transiciones del segundo autómata
            \path[->]
                (q5) edge node {$\varepsilon$} (q6)
                (q5) edge node {$\varepsilon$} (q7)
                (q6) edge node {1} (q8)
                (q8) edge node {1} (q9)
                (q9) edge node {0} (q10)
                (q11) edge node {1} (q12)
                (q7) edge node {0} (q11);

            % Conexión entre los autómatas
            \path[->]
                (q3) edge node {$\varepsilon$} (q5)
                (q4) edge node {$\varepsilon$} (q5);
            
            % Flecha de inicio
            \node (start) [left=of q0] {};
            \draw[->] (start) -- (q0);

        \end{tikzpicture}
    \end{center}

    Se minimiza el autómata anterior.
    Como primer paso, sacamos la $\varepsilon$-cerradura de cada estado:
    \begin{center}
      \centering
        \begin{tabular}{c|c}
              & $Cl_\varepsilon$ \\ \hline
            $q_0$ & $q_0,q_1,q_2,q_3,q_4,q_5,q_6,q_7$  \\
            $q_1$ & $q_1$  \\
            $q_2$ & $q_2$  \\
            $q_3$ & $q_0,q_1,q_2,q_3,q_4,q_5,q_6,q_7$  \\
            $q_4$ & $q_0,q_1,q_2,q_3,q_4,q_5,q_6,q_7$  \\
            $q_5$ & $q_5,q_6,q_7$  \\
            $q_6$ & $q_6$  \\
            $q_7$ & $q_7$  \\
            $q_8$ & $q_8$  \\
            $q_9$ & $q_9$  \\
            $q_{10}$ & $q_{10}$  \\
            $q_{11}$ & $q_{11}$  \\
            $q_{12}$ & $q_{12}$  \\
        \end{tabular}
    \end{center}

    Asi, el AFN sin $\varepsilon$-transiciones queda definido por las siguiente transiciones:
    \begin{align*}
      \delta(q_0, 0) &= Cl_{\varepsilon}(\bigcup_{p \in Cl_{\varepsilon}(q_0)} \delta_{\varepsilon}(p,0)) \\
        &= Cl_\varepsilon(\delta_\varepsilon(q_0,0) \cup \delta_\varepsilon(q_1,0) \cup \delta_\varepsilon(q_2, 0) \cup \delta_\varepsilon(q_3, 0) \cup \delta_\varepsilon(q_4, 0) \cup \delta_\varepsilon(q_5,0) \cup \delta_\varepsilon(q_6, 0) \cup \delta_\varepsilon(q_7,0)) \\
        &= Cl_\varepsilon(\{q_4,q_{11}\}) = \{q_0,q_1,q_2,q_3,q_4,q_5,q_6,q_7,q_{11}\}\\
    \end{align*}
    \begin{align*}
      \delta(q_0, 1) &= Cl_{\varepsilon}(\bigcup_{p \in Cl_{\varepsilon}(q_0)} \delta_{\varepsilon}(p,1)) \\
        &= Cl_\varepsilon(\delta_\varepsilon(q_0,1) \cup \delta_\varepsilon(q_1,1) \cup \delta_\varepsilon(q_2, 1) \cup \delta_\varepsilon(q_3,1) \cup \delta_\varepsilon(q_4, 1) \cup \delta_\varepsilon(q_5,1) \cup \delta_\varepsilon(q_6, 1) \cup \delta_\varepsilon(q_7,1)) \\
        &= Cl_\varepsilon(\{q_3,q_8\}) = \{q_0,q_1,q_2,q_3,q_4,q_5,q_6,q_7,q_8\}\\
    \end{align*}
    \begin{align*}
      \delta(q_1, 0) &= Cl_{\varepsilon}(\bigcup_{p \in Cl_{\varepsilon}(q_1)} \delta_{\varepsilon}(p,0)) = Cl_\varepsilon(\delta_\varepsilon(q_1,0)) = Cl_\varepsilon(\varnothing) = \varnothing
    \end{align*}
    \begin{align*}
      \delta(q_1, 1) &= Cl_{\varepsilon}(\bigcup_{p \in Cl_{\varepsilon}(q_1)} \delta_{\varepsilon}(p,1)) = Cl_\varepsilon(\delta_\varepsilon(q_1,1)) = Cl_\varepsilon(\{q_3\}) = \{q_0,q_1,q_2,q_3,q_4,q_5,q_6,q_7\}
    \end{align*}
    \begin{align*}
    \delta(q_2, 0) &= Cl_{\varepsilon}(\bigcup_{p \in Cl_{\varepsilon}(q_2)} \delta_{\varepsilon}(p,0)) = Cl_\varepsilon(\delta_\varepsilon(q_2,0)) = Cl_\varepsilon(\{q_4\}) = \{q_0,q_1,q_2,q_3,q_4,q_5,q_6,q_7\} \\
    \end{align*}
    \begin{align*}
    \delta(q_2, 1) &= Cl_{\varepsilon}(\bigcup_{p \in Cl_{\varepsilon}(q_2)} \delta_{\varepsilon}(p,1)) = Cl_\varepsilon(\delta_\varepsilon(q_2,1)) = Cl_\varepsilon(\varnothing) = \varnothing \\
    \end{align*}
    \begin{align*}
      \delta(q_3, 0) &= Cl_{\varepsilon}(\bigcup_{p \in Cl_{\varepsilon}(q_3)} \delta_{\varepsilon}(p,0)) \\
        &= Cl_\varepsilon(\delta_\varepsilon(q_0,0) \cup \delta_\varepsilon(q_1,0) \cup \delta_\varepsilon(q_2, 0) \cup \delta_\varepsilon(q_3, 0) \cup \delta_\varepsilon(q_4, 0) \cup \delta_\varepsilon(q_5,0) \cup \delta_\varepsilon(q_6, 0) \cup \delta_\varepsilon(q_7,0)) \\
        &= Cl_\varepsilon(\{q_4,q_{11}\}) = \{q_0,q_1,q_2,q_3,q_4,q_5,q_6,q_7,q_{11}\}\\
    \end{align*}
    \begin{align*}
      \delta(q_3, 1) &= Cl_{\varepsilon}(\bigcup_{p \in Cl_{\varepsilon}(q_3)} \delta_{\varepsilon}(p,1)) \\
        &= Cl_\varepsilon(\delta_\varepsilon(q_0,1) \cup \delta_\varepsilon(q_1,1) \cup \delta_\varepsilon(q_2, 1) \cup \delta_\varepsilon(q_3,1) \cup \delta_\varepsilon(q_4, 1) \cup \delta_\varepsilon(q_5,1) \cup \delta_\varepsilon(q_6, 1) \cup \delta_\varepsilon(q_7,1)) \\
        &= Cl_\varepsilon(\{q_3,q_8\}) = \{q_0,q_1,q_2,q_3,q_4,q_5,q_6,q_7,q_8\}\\
    \end{align*}
    \begin{align*}
      \delta(q_4, 0) &= Cl_{\varepsilon}(\bigcup_{p \in Cl_{\varepsilon}(q_4)} \delta_{\varepsilon}(p,0)) \\
        &= Cl_\varepsilon(\delta_\varepsilon(q_0,0) \cup \delta_\varepsilon(q_1,0) \cup \delta_\varepsilon(q_2, 0) \cup \delta_\varepsilon(q_3, 0) \cup \delta_\varepsilon(q_4, 0) \cup \delta_\varepsilon(q_5,0) \cup \delta_\varepsilon(q_6, 0) \cup \delta_\varepsilon(q_7,0)) \\
        &= Cl_\varepsilon(\{q_4,q_{11}\}) = \{q_0,q_1,q_2,q_3,q_4,q_5,q_6,q_7,q_{11}\}\\
    \end{align*}
    \begin{align*}
      \delta(q_4, 1) &= Cl_{\varepsilon}(\bigcup_{p \in Cl_{\varepsilon}(q_4)} \delta_{\varepsilon}(p,1)) \\
        &= Cl_\varepsilon(\delta_\varepsilon(q_0,1) \cup \delta_\varepsilon(q_1,1) \cup \delta_\varepsilon(q_2, 1) \cup \delta_\varepsilon(q_3,1) \cup \delta_\varepsilon(q_4, 1) \cup \delta_\varepsilon(q_5,1) \cup \delta_\varepsilon(q_6, 1) \cup \delta_\varepsilon(q_7,1)) \\
        &= Cl_\varepsilon(\{q_3,q_8\}) = \{q_0,q_1,q_2,q_3,q_4,q_5,q_6,q_7,q_8\}\\
    \end{align*}
    \begin{align*}
      \delta(q_5, 0) &= Cl_{\varepsilon}(\bigcup_{p \in Cl_{\varepsilon}(q_5)} \delta_{\varepsilon}(p,0)) = Cl_\varepsilon(\delta_\varepsilon(q_5,0) \cup \delta_\varepsilon(q_6,0) \cup \delta_\varepsilon(q_7, 0)) = Cl_\varepsilon(\{q_{11}\}) = \{q_{11}\}\\
    \end{align*}
    \begin{align*}
      \delta(q_5, 1) &= Cl_{\varepsilon}(\bigcup_{p \in Cl_{\varepsilon}(q_5)} \delta_{\varepsilon}(p,1)) = Cl_\varepsilon(\delta_\varepsilon(q_5,1) \cup \delta_\varepsilon(q_6,1) \cup \delta_\varepsilon(q_7,1)) = Cl_\varepsilon(\{q_8\}) = \{q_8\}\\
    \end{align*}
    \begin{align*}
      \delta(q_6, 0) &= Cl_{\varepsilon}(\bigcup_{p \in Cl_{\varepsilon}(q_6)} \delta_{\varepsilon}(p,0)) = Cl_\varepsilon(\delta_\varepsilon(q_6,0)) = Cl_\varepsilon(\varnothing) = \varnothing\\ 
    \end{align*}
    \begin{align*}
      \delta(q_6, 1) &= Cl_{\varepsilon}(\bigcup_{p \in Cl_{\varepsilon}(q_6)} \delta_{\varepsilon}(p,1)) = Cl_\varepsilon(\delta_\varepsilon(q_6,1)) = Cl_\varepsilon(\{q_8\}) = \{q_8\}\\
    \end{align*}
    \begin{align*}
      \delta(q_7, 0) &= Cl_{\varepsilon}(\bigcup_{p \in Cl_{\varepsilon}(q_7)} \delta_{\varepsilon}(p,0)) = Cl_\varepsilon(\delta_\varepsilon(q_7,0)) = Cl_\varepsilon(\{q_{11}\}) = \{q_{11}\}\\
    \end{align*}
    \begin{align*}
      \delta(q_7, 1) &= Cl_{\varepsilon}(\bigcup_{p \in Cl_{\varepsilon}(q_7)} \delta_{\varepsilon}(p,1)) = Cl_\varepsilon(\delta_\varepsilon(q_7,1)) = Cl_\varepsilon(\varnothing) = \varnothing\\
    \end{align*}
    \begin{align*}
      \delta(q_8, 0) &= Cl_{\varepsilon}(\bigcup_{p \in Cl_{\varepsilon}(q_8)} \delta_{\varepsilon}(p,0)) = Cl_\varepsilon(\delta_\varepsilon(q_8,0)) = Cl_\varepsilon(\varnothing) = \varnothing\\
    \end{align*}
    \begin{align*}
      \delta(q_8, 1) &= Cl_{\varepsilon}(\bigcup_{p \in Cl_{\varepsilon}(q_8)} \delta_{\varepsilon}(p,1)) = Cl_\varepsilon(\delta_\varepsilon(q_8,1)) = Cl_\varepsilon(\{q_9\}) = \{q_9\}\\
    \end{align*}
    \begin{align*}
      \delta(q_9, 0) &= Cl_{\varepsilon}(\bigcup_{p \in Cl_{\varepsilon}(q_9)} \delta_{\varepsilon}(p,0)) = Cl_\varepsilon(\delta_\varepsilon(q_9,0)) = Cl_\varepsilon(\{q_{10}\}) = \{q_{10}\}\\
    \end{align*}
    \begin{align*}
      \delta(q_9, 1) &= Cl_{\varepsilon}(\bigcup_{p \in Cl_{\varepsilon}(q_9)} \delta_{\varepsilon}(p,1)) = Cl_\varepsilon(\delta_\varepsilon(q_9,1)) = Cl_\varepsilon(\varnothing) = \varnothing\\
    \end{align*}
    \begin{align*}
      \delta(q_{10}, 0) &= Cl_{\varepsilon}(\bigcup_{p \in Cl_{\varepsilon}(q_{10})} \delta_{\varepsilon}(p,0)) = Cl_\varepsilon(\delta_\varepsilon(q_{10},0)) = Cl_\varepsilon(\varnothing) = \varnothing\\
    \end{align*}
    \begin{align*}
      \delta(q_{10}, 1) &= Cl_{\varepsilon}(\bigcup_{p \in Cl_{\varepsilon}(q_{10})} \delta_{\varepsilon}(p,1)) = Cl_\varepsilon(\delta_\varepsilon(q_{10},1)) = Cl_\varepsilon(\varnothing) = \varnothing\\
    \end{align*}
    \begin{align*}
      \delta(q_{11}, 0) &= Cl_{\varepsilon}(\bigcup_{p \in Cl_{\varepsilon}(q_{11})} \delta_{\varepsilon}(p,0)) = Cl_\varepsilon(\delta_\varepsilon(q_{11},0)) = Cl_\varepsilon(\varnothing) = \varnothing\\
    \end{align*}
    \begin{align*}
      \delta(q_{11}, 1) &= Cl_{\varepsilon}(\bigcup_{p \in Cl_{\varepsilon}(q_{11})} \delta_{\varepsilon}(p,1)) = Cl_\varepsilon(\delta_\varepsilon(q_{11},1)) = Cl_\varepsilon(\{q_{12}\}) = \{q_{12}\}\\
    \end{align*}
    \begin{align*}
      \delta(q_{12}, 0) &= Cl_{\varepsilon}(\bigcup_{p \in Cl_{\varepsilon}(q_{12})} \delta_{\varepsilon}(p,0)) = Cl_\varepsilon(\delta_\varepsilon(q_{12},0)) = Cl_\varepsilon(\varnothing) = \varnothing\\
    \end{align*}
    \begin{align*}
      \delta(q_{12}, 1) &= Cl_{\varepsilon}(\bigcup_{p \in Cl_{\varepsilon}(q_{12})} \delta_{\varepsilon}(p,1)) = Cl_\varepsilon(\delta_\varepsilon(q_{12},1)) = Cl_\varepsilon(\varnothing) = \varnothing\\
    \end{align*}

    Usando lo anterior, el AFD queda de la siguiente manera:
    \begin{center}
      \centering
        \begin{tabular}{c|c|c}
            $\delta$ & $0$ & $1$ \\ \hline
            $\rightarrow q_0$ & $q_0q_1q_2q_3q_4q_5q_6q_7q_{11}$ & $q_0q_1q_2q_3q_4q_5q_6q_7q_8$ \\
            $q_0q_1q_2q_3q_4q_5q_6q_7q_{11}$ & $q_0q_1q_2q_3q_4q_5q_6q_7q_{11}$& $q_0q_1q_2q_3q_4q_5q_6q_7q_8q_{12}$\\
            $q_0q_1q_2q_3q_4q_5q_6q_7q_8$ & $q_0q_1q_2q_3q_4q_5q_6q_7q_{11}$& $q_0q_1q_2q_3q_4q_5q_6q_7q_8q_9$\\
            *$q_0q_1q_2q_3q_4q_5q_6q_7q_8q_{12}$ & $q_0q_1q_2q_3q_4q_5q_6q_7q_{11}$ & $q_0q_1q_2q_3q_4q_5q_6q_7q_8q_9$\\
            $q_0q_1q_2q_3q_4q_5q_6q_7q_8q_9$ & $q_0q_1q_2q_3q_4q_5q_6q_7q_{10}q_{11}$& $q_0q_1q_2q_3q_4q_5q_6q_7q_8q_9$ \\
            *$q_0q_1q_2q_3q_4q_5q_6q_7q_{10}q_{11}$ & $q_0q_1q_2q_3q_4q_5q_6q_7q_{11}$& $q_0q_1q_2q_3q_4q_5q_6q_7q_8q_{12}$
        \end{tabular}
    \end{center}

    Renombrando los estados:
    \begin{center}
      \centering
        \begin{tabular}{c|c|c}
            $\delta$ & $0$ & $1$ \\ \hline
            $\rightarrow p_0$ & $p_1$ & $p_2$ \\
            $p_1$ & $p_1$ & $p_3$\\
            $p_2$ & $p_1$ & $p_4$\\
            *$p_3$ & $p_1$ & $p_4$\\
            $p_4$ & $p_5$ & $p_4$ \\
            *$p_5$ & $p_1$ & $p_3$ 
        \end{tabular}
    \end{center}

    Comenzando a minimizar. Por la construcción del AFD, todos los estados son alcanzables. Asi que se procede separando los estados finales y los no finales:
    \begin{enumerate}
      \item Finales = $A = \{p_3, p_5\}$
      \item No finales = $B = \{p_0, p_1, p_2, p_4\}$
    \end{enumerate}

    Con ello, comienza la primera etapa:
    \begin{center}
      \centering
      \begin{tabular}{c | c | c | c | c | c | c}
        & \multicolumn{2}{c|}{A} & \multicolumn{4}{c}{B} \\    
        \hline
        & $p_3$ & $p_5$ & $p_0$ & $p_1$ & $p_2$ & $p_4$   \\
        \hline
        0 & B & B & B & B & B & A  \\
        1 & B & A & B & A & B & B
      \end{tabular}
    \end{center}

    Entonces obtenemos los siguiente grupos:
    \begin{enumerate}
      \item $A = \{p_3\}$
      \item $B = \{p_5\}$
      \item $C = \{p_0, p_2\}$
      \item $D = \{p_1\}$
      \item $E = \{p_4\}$
    \end{enumerate}

    Continuamos con las clases de equivalencia obtenidas:
    \begin{center}
      \centering
      \begin{tabular}{c | c | c | c | c | c | c}
        & \multicolumn{1}{c|}{A} & \multicolumn{1}{c|}{B} & \multicolumn{2}{c|}{C} & D & E \\    
        \hline
        & $p_3$ & $p_5$ & $p_0$ & $p_2$ & $p_1$ & $p_4$   \\
        \hline
        0 & D & D & D & D & D & B \\
        1 & E & A & C & E & A & E
      \end{tabular}
    \end{center}
        
    Observando la tabla, se puede apreciar que cada estado original quedó separando de los demás. Es decir, el AFD obtenido en primera instancia ya era el mínimo. El diagrama correspondiente queda de la siguiente forma:

    \begin{center}
        \begin{tikzpicture}[shorten >=1pt, node distance=2cm, on grid, auto]
            \tikzset{initial text={}}
            
            \node[state, initial] (p0) {$p_0$}; 
            \node[state] (p1) [right=of p0] {$p_1$}; 
            \node[state] (p2) [below =of p0] {$p_2$}; 
            \node[state, accepting] (p3) [right=of p1] {$p_3$}; 
            \node[state] (p4) [right=of p2] {$p_4$}; 
            \node[state, accepting] (p5) [right=of p4] {$p_5$}; 

            \path[->]
                (p0) edge node {0} (p1)
                (p0) edge node [left] {1} (p2)
                (p1) edge [loop above] node {0} (p1)
                (p1) edge [bend left] node {1} (p3)
                (p2) edge node {0} (p1)
                (p2) edge node [below] {1} (p4)
                (p3) edge node {0} (p1)
                (p3) edge node [near end, left] {1} (p4)
                (p4) edge [loop below] node {1} (p4)
                (p4) edge node [below] {0} (p5)
                (p5) edge node [right] {1} (p3)
                (p5) edge node [near start] {0} (p1);
            
            \node (start) [left=of qi] {};
            \draw[->] (start) -- (qi);

        \end{tikzpicture}
    \end{center}
}

\ex{Lema de Arden}{
  Utilizando el Lema de Arden, se calcula una expresión regular para cada una de los siguientes autómatas, muestrando las ecuaciones y los pasos para obtener la expresión.

  \underline{Primer autómata}
  \begin{center}
    \centering
      \begin{tabular}{c|c|c}
          $\delta$ & $a$ & $b$ \\ \hline
          $\rightarrow q_0$ & $q_1$ & $q_3$ \\
          * $q_1$ & -- & $q_2$ \\
          $q_2$ & $q_1$ & $q_2$ \\
          $q_3$ & $q_1$ & --
      \end{tabular}
  \end{center}

  \setcounter{equation}{0}
  De la tabla anterior tenemos las siguiente ecuaciones:
  \begin{align}
    L(q_0) &= aL(q_1) + bL(q_3) \\
    L(q_1) &= bL(q_2) + \varepsilon \\
    L(q_2) &= aL(q_1) + bL(q_2) \\
    L(q_3) &= aL(q_1) 
  \end{align}

  De $3$ obtenemos
  \begin{align}
    L(q_2) = aL(q_1) + bL(q_2) = b^*(aL(q_1))
  \end{align}

  Sustituyendo $5$ en $2$
  \begin{align}
    L(q_1) &= bb^*aL(q_1) + \varepsilon = (bb^*a)^*\varepsilon = (bb^*a)^*
  \end{align}

  Sustituyendo $6$ en $4$
  \begin{align}
    L(q_3) &= a(bb^*a)^*
  \end{align}

  Por $6, 7$ en $1$
  \begin{align}
    L(q_0) &= a(bb^*a)^* + ba(bb^*a)^* = (a + ba)(bb^*a)^*
  \end{align}

  Entonces, la expresión regular para el autómata dado es
  $$\boxed{a(bb^*a)^* + ba(bb^*a)^* = (a + ba)(bb^*a)^*}$$

  \underline{Segundo autómata:}
  \begin{center}
    \centering
      \begin{tabular}{c|c|c}
          $\delta$ & $a$ & $b$ \\ \hline
          *$\rightarrow q_0$ & $q_0$ & $q_1$ \\
        * $q_1$ & $q_2$ & $q_1$ \\
        * $q_2$ & $q_0$ & $q_3$ \\
        * $q_3$ & $q_3$ & $q_3$
      \end{tabular}
  \end{center}

  \setcounter{equation}{0}
  Tenemos las siguiente ecuaciones
  \begin{align}
    L(q_0) &= aL(q_0) + bL(q_1) + \varepsilon \\
    L(q_1) &= aL(q_2) + bL(q_1) + \varepsilon \\
    L(q_2) &= aL(q_0) + bL(q_3) + \varepsilon \\
    L(q_3) &= aL(q_3) + bL(q_3) + \varepsilon 
  \end{align}

  De $3$ obtenemos
  \begin{equation}
  \begin{aligned}
    L(q_3) &= (a + b)L(q_3) + \varepsilon \\
          &= (a + b)^* (\varepsilon) \\
          &= (a + b)^* 
  \end{aligned}
  \end{equation}

  Sustituyendo $5$ en $3$
  \begin{equation}
  \begin{aligned}
    L(q_2) &= aL(q_0) + b(a+b)^* + \varepsilon \\
  \end{aligned}
  \end{equation}

  Sustituyendo $6$ en $2$
  \begin{equation}
  \begin{aligned}
    L(q_1) &= a(aL(q_0) + b(a+b)^* + \varepsilon) + bL(q_1) + \varepsilon \\
          &= aaL(q_0) + ab(a+b)^* + a + bL(q_1) + \varepsilon \\
          &= b^* (aaL(q_0) + ab(a+b)^* + a + \varepsilon )\\
          &= b^*aaL(q_0) + b^*ab(a+b)^* + b^*a + b^* \\
  \end{aligned}
  \end{equation}

  Sustituyendo $7$ en $1$
  \begin{equation}
  \begin{aligned}
    L(q_0) &= aL(q_0) + b(b^*aaL(q_0) + b^*ab(a+b)^* + b^*a + b^* ) + \varepsilon \\
          &= aL(q_0) + bb^*aaL(q_0) + bb^*ab(a+b)^* + bb^*a + bb^* + \varepsilon \\
          &= (a + bb^*aa)L(q_0) + bb^*ab(a+b)^* + bb^*a + bb^* + \varepsilon \\
          &= (a + bb^*aa)^* (bb^*ab(a+b)^* + bb^*a + bb^* + \varepsilon)\\
          &= (a + b^+aa)^* (b^+ab(a+b)^* + b^+a + b^+ + \varepsilon)\\
  \end{aligned}
  \end{equation}

  Entonces, la expresión regular para el autómata es 
  $$\boxed{(a + b^+aa)^* (b^+ab(a+b)^* + b^+a + b^+ + \varepsilon)}$$
}
