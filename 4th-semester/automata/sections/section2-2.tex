\section{Expresiones Regulares}
\dfn{Expresiones Regulares}{
  Las \textbf{expresiones regulares} se definen recursivamente de la siguiente manera.

  \begin{enumerate}[a)]
    \item $\varnothing,~ \varepsilon$ son expresiones regulares.
    \item Si $a \in \Sigma$, entonces \textit{a} es una expresión regular.
    \item Si \textit{r} y \textit{s} son expresiones regulares, entonces $r + s,~rs, ~ r^*$ nos expresiones regulares (considerese los casos análogos).
    \item Solo las anteriores son expresiones regulares.
  \end{enumerate}
}

\dfn{Lenguaje de una Expresión Regular}{
  El \textbf{lenguaje de una expresión regular} \textit{r} se denota $L(r)$. Dicho lenguaje corresponde a uno de los siguientes casos.
  \begin{enumerate}
    \item Si $r = \varnothing,~ L(r) = \varnothing$
    \item Si $r = \varepsilon, ~ L(r) = \{\varepsilon\}$
    \item Si $r = a$, con $a \in \Sigma,~ L(r) = \{a\}$.
    \item Si $r = s + t,~ L(r) = L(s) \cup L(t)$
    \item Si $r = st,~ L(r) = L(s)L(t)$
    \item Si $r = s^*,~ L(r) = (L(s))^*$
  \end{enumerate}
  donde \textit{r,s,t} son expresiones regulares.
}

\ex{Lenguaje de una Expresión Regular}{
  Sea \textit{L} el lenguaje de cadenas que contiene exactamente una \textit{b}. Entonces \[a^*ba^*\] es la expresión regular que representa a cualquier cadena de \textit{L}. Por ende, \[L(a^*ba^*) = \{b, ab, ba, aab, aabaaa, \cdots\}\]
}

\ex{Lenguaje de una Expresión Regular}{
  Sea \textit{L} el lenguaje de cadenas que contiene la subcadena \textit{bb}. Entonces \[\Sigma^*bb\Sigma^* = (a+b)^* bb (a+b)^*\] es la expresión regular que representa a cualquier cadena de \textit{L}.
}

\qs{Expresiones Regulares}{
  Sea $\Sigma = \{a,b\}$. Diseña expresiones regulares para los siguientes lenguajes.
  \begin{itemize}
    \item EL lenguaje de cadenas que inician y terminan con \textit{bb}.
    \item El lenguaje de cadenas de longitud par.
    \item El lenguaje de cadenas cuyo penúltimo  símbolo es \textit{a}.
    \item El lenguaje de cadenas que no contienen dos \textit{a's} consecutivas.
  \end{itemize}
}

\mprop{}{
  Un lenguaje \textit{M} es \textbf{regular} \textit{si y solo si existe} una expresión regular \textit{r} tal que \[L(r) = M\]
}

Las siguientes son propiedades de las expresiones regulares.\\
Sean \textit{r,s,t} exp. reg.
\begin{enumerate}
  \item \[r + s = s + r\]
  \item \[r + \varnothing = r = \varnothing + r\]
  \item \[ r + r = r\]
  \item \[\varepsilon + r = r + \varepsilon = r\]
  \item \[(r + s) + t = r + (s + t)\]
  \item \[r \varepsilon = r = \varepsilon r\]
  \item \[r \varnothing = \varnothing r = \varnothing\]
  \item \[(rs)t = r(st)\]
  \item \[r(s + t) = rs + rt\] \[(s + t)r = sr + tr\]
  \item \[r^* = (r^*)^* = r^* r^* = (\varepsilon + r)^* = r^*(r + \varepsilon) = (r + \varepsilon)r^* = \varepsilon + rr^*\]
  \item \[(r + s)^* = (r^* + s^*)^* = (r^*s^*)^* = (r^*s)r^* = r^*(sr^*)^*\]
  \item \[r(sr)^* = (rs)^*r\]
  \item \[(r^*s)^* = \varepsilon + (r + s)^*s\]
  \item \[(r s^*)^* = \varepsilon + r(r + s)^*\]
  \item \[s(r + \varepsilon)^* (r + \varepsilon) + s = s r^*\]
  \item \[rr^* = r^*r\]
\end{enumerate}

Las propiedades enunciadas anteriormente pueden ser demostradas mediante \textit{argumentación} o \textit{análisis}.
