\section{Linear dependence and linear independence}

\dfn{}{
  A $S \subseteq V$ is called \textbf{linearly dependent} if there exist a finite number o distinct vectors $u_1, u_2, \cdots, u_n \in S$ and scalars $\lambda_1, \lambda_2, \cdots \lambda_n \in K$, not all zero, such that $$\lambda_1 u_1 + \lambda_2 u_2 + \cdots \lambda_n u_n = \overline0$$
  In this case we also say that the vectors of \textit S are linearly dependent.
}

For any vectors $u_1, u_2, \cdots, u_n$, we have $\lambda_1 u_1 + \lambda_2 u_2 + \cdots \lambda_n u_n = \overline0$ if $\lambda_1 = \lambda_2 = \cdots = \lambda_n = 0$. We call this the \textbf{trivial repesentation} of $\overline0$ as a linear combination of $u_1, u_2, \cdots, u_n$. Thus, for a set to be linearly dependent, there must exist a nontrivial representation of $\overline0$ as a linear combination of vectors in the set.

\nt{
  Any subset of a vector space that contains the zero vector is linearly dependent, because $\overline0 = 1 \times 0$ is a nontrivial repesentation of $\overline0$ as a linear combination of vectors in the set.
}

\dfn{}{
  A $S \subseteq V$ that isn't linearly dependent is called \textbf{linearly independent}. As before, we also say that the vectors of \textit S are linearly independent.
}

The following facts about linearly independent sets are true in any vector space.
\begin{enumerate}
  \item The empty set is linearly independent, for linearly dependent sets must be nonempty.
  \item A set consisting of a single nonzero vector is linearly independent. For if $\{u\}$ is linearly dependent, then $\lambda u = \overline 0$ for some nonzero scalar $\lambda$. Thus $u = \lambda ^{-1}(\lambda u) = \lambda^{-1}\overline0 = \overline0$
  \item A set is linearly independent if and only if the only repesentations of $\overline0$ as linear combination of its vector are trivial repesentations.
\end{enumerate}

\noindent The condition in item 3 provides a useful method for determining whether a finite set is linearly independent.

The following important results are immediate consequences of the definitions of linear dependent and linear independence.

\thm{}{
  Let \textit V be a vector space, and let $S_1 \subseteq S_2 \subseteq V$. If $S_1$ is linearly dependen, then $S_2$ is linearly dependent.
}

\cor{}{
  Let \textit V be a vector space, and let $S_1 \subseteq S_2 \subseteq V$. If $S_2$ is linearly independent, then $S_1$ is linearly independent.
}

\thm{}{
  Let \textit S be a linearly independent subset of a vector space \textit V, and let $v \in V \mid v \notin S$. Then $S \cup \{v\}$ is linearly dependent if and only if $v \in$ span(\textit S).
}

Linearly independent generating sets are investigated in detail in the coming section.
