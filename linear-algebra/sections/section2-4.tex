\section{Invertibily and Isomorphisms}
We see in this section that the inverse of a linear transformation is also linear. This result greatly aids us  in the study of \textit{inverses} of matrices.

\dfn{Inverse of a function}{
  Let \textit{V} and \textit{W} be vector spaces, and let $T : V \longrightarrow W$ be linear. A function $U : W \longrightarrow V$ is said to be an \textbf{inverse} of \textit{T} if $TU = I_W$ and $UT = I_V$. If \textit{T} has an inverse, then \textit{T} is said to be \textbf{invertible}. Moreover, if \textit{T} is invertible, then the inverse of \textit{T} is unique and is denoted by $T^{-1}$
}

The following facts hold for invertible functions \textit{T} and \textit{U}.
\begin{enumerate}
  \item $(TU)^{-1} = U^{-1} T^{-1}$
  \item $(T^{-1})^{-1} = T$; in particular, $T^{-1}$ is invertible.
  \item Let $T : V \longrightarrow W$ be a linear transformation, where \textit{V} and \textit{W} are finite-dimensional spaces of equal dimension. Then \textit{T} is invertible \textit{if and only if} rank(\textit{T}) $=$ dim(\textit{V}).
\end{enumerate}

\thm{}{
  Let \textit{V} and \textit{W} be vector spaces, and let $T : V \longrightarrow W$ be linear and invertible. Then $T^{-1} : W \longrightarrow V$ is linear.
}

It now follows immediately that if \textit{T} is a linear transformation between vector spaces of equal (finite) dimension, then the conditions of being invertible, one-to-one, and onto are all equivalent.

\dfn{Inverse of a matrix}{
  Let \textit{A} be a $n \times n$ matrix. Then \textit{A} is \textbf{invertible} if there exists and $n \times n$ matrix \textit{B} such that $AB = BA = I$.
}

\ex{inverse of a matrix}{
  It's easy to verify that the invese of
  \[
  \begin{pmatrix}
    5 & 7\\
    2 & 3\\
  \end{pmatrix}
  \qquad \text{ is } \qquad
  \begin{pmatrix}
    3 & -7\\
    -2 & 5\\
  \end{pmatrix}
  \]
}

At this point, we develop a number of results that relate the invese of matrices to the inverses of linear transformations.

\mlemma{}{
  Let \textit{T} be an invertible linear transformation from \textit{V} to \textit{W}. Then \textit{V} is finite-dimensional \textit{if and only if} \textit{W} is finite-dimensional. In this case, dim(\textit{V}) $=$ dim(\textit{W}).
}

\thm{}{
  Let \textit{V} and \textit{W} be finite-dimensional vector spaces with ordered bases $\beta$ and $\gamma$, respectively. Let $T : V \longrightarrow W$ be linear. Then \textit{T} is invertible \textit{if and only if} $[T]_\beta^\gamma$ is invertible. Furthermore, $[T^{-1}]_\gamma^\beta = ([T]_\beta^\gamma)^{-1}$.
}

\cor{}{
  Let \textit{V} be a finite-dimensional vector space with and ordered basis $\beta$, and let $T : V \longrightarrow V$ be linear. Then \textit{T} is invertible \textit{if and only if} $[T]_\beta$ is invertible. Furthermore, $[T^{-1}]_\beta = ([T]_\beta)^{-1}$.
}

\cor{}{
  Let \textit{A} be an $n \times n$ matrix. Then \textit{A} is invertible \textit{if and only if} $L_A$ is invertible. Furthermore, $(L_A)^{-1} = L_{A^{-1}}$
}

The notion of invertibility may be used to formalize what we've been already observerd, that is, that certain vector spaces strongly resemble one another except for the form of their vectors.

\dfn{Isomorphism}{
  Let \textit{V} and \textit{W} be vector spaces. We say that \textit{V} is \textbf{isomorphic} to \textit{W} if there exists a linear transformation $T : V \rightarrow W$ that's invertible. Such a linear transformation is called an \textbf{Isomorphism} from \textit{V} onto \textit{W}.
}

\ex{Isomorphism}{
  Define $T : F^2 \rightarrow P_1(F)$ by $T(a_1, a_2) = a_1 + a_2x$. It's easily checked that \textit{T} us an Isomorphism; so $F^2$ is isomorphic to $P_1(F).$
}

\typeout{Engine: \ifxetex XeLaTeX\else\ifluatex LuaLaTeX\else pdfTeX\fi\fi}
