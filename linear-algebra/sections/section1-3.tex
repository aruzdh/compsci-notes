\section{Linear combinations}

Expressions of the form $su + tv$, where \textit s and \textit t are scalars and \textit u and \textit v are vectors, play a central role in the theory of vector spaces. The appropiate generalization of such expressions is presented in the following definitions.

\dfn{}{
  Let \textit V be a vector space and $S \neq \varnothing \subseteq V$. A vector $v \in V$ is called a \textbf{linear combination} of vectors of \textit S if there exist a finite number of vectors $u_1, u_2, \cdots, u_n \in S$ and scalars $\lambda_1, \lambda_2, \cdots, \lambda_n \in K$ such that $$v = \lambda_1 u_1 + \lambda_2 u_2 + \cdots + \lambda_n u_n$$
  In this case we also say that \textit v is a linear combination of $u_1, u_2, \cdots, u_n$ and call $\lambda_1, \lambda_2, \cdots \lambda_n$ the \textbf{coefficients} of the linear combination.
}

\noindent Observe that in any vector space \textit V, $\forall v \in V, ~0v = \overline 0$ . Thus the zero vector is a linear combination of any nonempty subset of \textit V.

Throughout this course, we form the set of all linear combinations of some set of vectors. We now name such a set of linear combinations.

\dfn{} {
  Let $S \neq \varnothing \subseteq V$. The \textbf{span} of \textit S, denoted span(\textit S), is the set consisting of all linear combinations of the vectors in \textit S. For convenience, we define span($\varnothing) = \{\overline 0\}$.
}

In $R^3$, for instance, the span of the set $\{(1,0,0), (0,1,0)\}$ consists of all vectors in $R^3$ that have the form $\lambda(1,0,0) + \beta(0,1,0) = (\lambda, \beta, 0)$ for some scalars $\lambda$ and $\beta$. Thus the span of $\{(1,0,0), (0,1,0)\}$ consists all the points in the \textit{xy}-plane. In this case, the span of the set is a subspace of $R^3$. This fact is true in general.

\thm{}{
  The span of any $S \subseteq V$ is a subspace of \textit V. Moreover, any subspace of \textit V that contains \textit S must also contain the span of \textit S.
}

\dfn{}{
  A $S \subseteq V$ \textbf{generates} (or \textbf{spans}) \textit V if span(\textit S) = \textit V. In this case, we also say that the vectors of \textit S generate (or span) \textit V.
}

Usually there are many different subsets that generate a subspace \textit W. It's natural to seek a subset of \textit W that generates \textit W and is as small as possible. In the next section we explore the circumstances under which a vector can be removed from a generating set to obtain a smaller generating set.

